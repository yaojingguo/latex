\documentclass{article}

\usepackage{tikz}

\usetikzlibrary{intersections}
\usetikzlibrary {arrows.meta}
\usetikzlibrary {positioning,shapes.misc}
\usetikzlibrary {calc}
\usetikzlibrary {graphs}

\begin{document}

\tikzset{terminal/.style={
                      % The shape:
                      rounded rectangle,
                      minimum size=6mm,
                      % The rest
                      very thick,draw=black!50,
                      top color=white,bottom color=black!20,
                  font=\ttfamily}}

\tikzset{nonterminal/.style={
      % The shape:
      rectangle,
      % The size:
      minimum size=6mm,
      % The border:
      very thick,
      draw=red!50!black!50,         % 50% red and 50% black,
                                    % and that mixed with 50% white
      % The filling:
      top color=white,              % a shading that is white at the top...
      bottom color=red!50!black!20, % and something else at the bottom
      % Font
      font=\itshape
  }}

\usetikzlibrary {positioning,shapes.misc}
\begin{tikzpicture}[node distance=5mm and 5mm]
  \node (ui1)   [nonterminal]                     {unsigned integer};
  \node (dot)   [terminal,right=of ui1]           {.};
  \node (digit) [terminal,right=of dot]           {digit};
  \node (E)     [terminal,right=of digit]         {E};
  \node (plus)  [terminal,above right=of E]       {+};
  \node (minus) [terminal,below right=of E]       {-};
  \node (ui2)   [nonterminal,below right=of plus] {unsigned integer};
\end{tikzpicture}

\begin{tikzpicture}[node distance=5mm and 5mm]
  \node (E)     [terminal]                                   {E};
  \node (plus)  [terminal,above right=of E,xshift=5mm]       {+};
  \node (minus) [terminal,below right=of E,xshift=5mm]       {-};
  \node (ui2)   [nonterminal,below right=of plus,xshift=5mm] {unsigned integer};
\end{tikzpicture}

\begin{tikzpicture}[node distance=5mm and 5mm,terminal/.append style={rectangle,rounded corners=3mm}]
  \node (E)     [terminal]                        {E};
  \node (plus)  [terminal,above right=of E]       {+};
  \node (minus) [terminal,below right=of E]       {-};
  \node (ui2)   [nonterminal,below right=of plus] {unsigned integer};
\end{tikzpicture}

\begin{tikzpicture}[node distance=5mm and 5mm]
  \node (dot)   [terminal]                        {.};
  \node (digit) [terminal,right=of dot]           {digit};
  \node (E)     [terminal,right=of digit]         {E};

  \path (dot)   edge[->] (digit)  % simple edges
        (digit) edge[->] (E);

  \draw [->]
     % start right of digit.east, that is, at the point that is the
     % linear combination of digit.east and the vector (2mm,0pt). We
     % use the ($ ... $) notation for computing linear combinations
     ($ (digit.east) + (2mm,0) $)
     % Now go down
     -- ++(0,-.5)
     % And back to the left of digit.west
     -| ($ (digit.west) - (2mm,0) $);
\end{tikzpicture}

\begin{tikzpicture}[node distance=5mm and 5mm,
    skip loop/.style={to path={-- ++(0,-.5) -| (\tikztotarget)}}]
  \node (dot)   [terminal]                        {.};
  \node (digit) [terminal,right=of dot]           {digit};
  \node (E)     [terminal,right=of digit]         {E};

  \path (dot)   edge[->]           (digit)  % simple edges
        (digit) edge[->]           (E)
        ($ (digit.east) + (2mm,0) $)
                edge[->,skip loop] ($ (digit.west) - (2mm,0) $);
\end{tikzpicture}

\begin{tikzpicture}[node distance=5mm and 5mm,
    skip loop/.style={to path={-- ++(0,#1) -| (\tikztotarget)}}]
  \node (dot)   [terminal]                        {.};
  \node (digit) [terminal,right=of dot]           {digit};
  \node (E)     [terminal,right=of digit]         {E};

  \path (dot)   edge[->]                (digit)  % simple edges
        (digit) edge[->]                (E)
        ($ (digit.east)!.5!(E.west) $)
                edge[->,skip loop=-5mm] ($ (digit.west)!.5!(dot.east) $);
\end{tikzpicture}

\begin{tikzpicture}
  \matrix[row sep=1mm,column sep=5mm] {
    % First row:
      & & & & \node [terminal] {+}; & \\
    % Second row:
    \node [nonterminal] {unsigned integer}; &
    \node [terminal]    {.};                &
    \node [terminal]    {digit};            &
    \node [terminal]    {E};                &
                                            &
    \node [nonterminal] {unsigned integer}; \\
    % Third row:
      & & & & \node [terminal] {-}; & \\
  };
\end{tikzpicture}

\begin{tikzpicture}[point/.style={circle,inner sep=0pt,minimum size=2pt,fill=red},
                   skip loop/.style={to path={-- ++(0,#1) -| (\tikztotarget)}}]
  \matrix[row sep=1mm,column sep=2mm] {
    % First row:
    & & & & & & &  & & & & \node (plus) [terminal] {+};\\
    % Second row:
    \node (p1) [point]  {}; &    \node (ui1)   [nonterminal] {unsigned integer}; &
    \node (p2) [point]  {}; &    \node (dot)   [terminal]    {.};                &
    \node (p3) [point]  {}; &    \node (digit) [terminal]    {digit};            &
    \node (p4) [point]  {}; &    \node (p5)    [point]  {};                      &
    \node (p6) [point]  {}; &    \node (e)     [terminal]    {E};                &
    \node (p7) [point]  {}; &                                                    &
    \node (p8) [point]  {}; &    \node (ui2)   [nonterminal] {unsigned integer}; &
    \node (p9) [point]  {}; &    \node (p10)   [point]       {};\\
    % Third row:
    & & & & & & &  & & & & \node (minus)[terminal] {-};\\
  };

  \path (p4) edge [->,skip loop=-5mm] (p3)
        (p2) edge [->,skip loop=5mm]  (p6);
\end{tikzpicture}

\def \matrixcontent {
    % First row:
    & & & & & & &  & & & & \node (plus) [terminal] {+};\\
    % Second row:
    \node (p1) [point]  {}; &    \node (ui1)   [nonterminal] {unsigned integer}; &
    \node (p2) [point]  {}; &    \node (dot)   [terminal]    {.};                &
    \node (p3) [point]  {}; &    \node (digit) [terminal]    {digit};            &
    \node (p4) [point]  {}; &    \node (p5)    [point]  {};                      &
    \node (p6) [point]  {}; &    \node (e)     [terminal]    {E};                &
    \node (p7) [point]  {}; &                                                    &
    \node (p8) [point]  {}; &    \node (ui2)   [nonterminal] {unsigned integer}; &
    \node (p9) [point]  {}; &    \node (p10)   [point]       {};\\
    % Third row:
    & & & & & & &  & & & & \node (minus)[terminal] {-};\\
}


\begin{tikzpicture}[point/.style={circle,inner sep=0pt,minimum size=2pt,fill=red},
  skip loop/.style={to path={-- ++(0,#1) -| (\tikztotarget)}},
                    hv path/.style={to path={-| (\tikztotarget)}},
                    vh path/.style={to path={|- (\tikztotarget)}}]
  \matrix[row sep=1mm,column sep=2mm] { 
    % First row:
    & & & & & & &  & & & & \node (plus) [terminal] {+};\\
    % Second row:
    \node (p1) [point]  {}; &    \node (ui1)   [nonterminal] {unsigned integer}; &
    \node (p2) [point]  {}; &    \node (dot)   [terminal]    {.};                &
    \node (p3) [point]  {}; &    \node (digit) [terminal]    {digit};            &
    \node (p4) [point]  {}; &    \node (p5)    [point]  {};                      &
    \node (p6) [point]  {}; &    \node (e)     [terminal]    {E};                &
    \node (p7) [point]  {}; &                                                    &
    \node (p8) [point]  {}; &    \node (ui2)   [nonterminal] {unsigned integer}; &
    \node (p9) [point]  {}; &    \node (p10)   [point]       {};\\
    % Third row:
    & & & & & & &  & & & & \node (minus)[terminal] {-};\\
  };

  \graph {
    (p1) -> (ui1) -- (p2) -> (dot) -- (p3) -> (digit) -- (p4)
         -- (p5)  -- (p6) -> (e) -- (p7) -- (p8) -> (ui2) -- (p9) -> (p10);
    (p4) ->[skip loop=-5mm]  (p3);
    (p2) ->[skip loop=5mm]   (p5);
    (p6) ->[skip loop=-11mm] (p9);
    (p7) ->[vh path]         (plus)  -> [hv path] (p8);
    (p7) ->[vh path]         (minus) -> [hv path] (p8);
  };
\end{tikzpicture}

\usetikzlibrary {arrows.meta,graphs,shapes.misc}
\begin{tikzpicture}[point/.style={circle,inner sep=0pt,minimum size=2pt,fill=red},
  skip loop/.style={to path={-- ++(0,#1) -| (\tikztotarget)}},
  >={Stealth[round]},thick,black!50,text=black,
                    every new ->/.style={shorten >=1pt},
                    graphs/every graph/.style={edges=rounded corners},
                    hv path/.style={to path={-| (\tikztotarget)}},
                    vh path/.style={to path={|- (\tikztotarget)}}]
  \matrix[column sep=4mm] { 
    % First row:
    & & & & & & &  & & & & \node (plus) [terminal] {+};\\
    % Second row:
    \node (p1) [point]  {}; &    \node (ui1)   [nonterminal] {unsigned integer}; &
    \node (p2) [point]  {}; &    \node (dot)   [terminal]    {.};                &
    \node (p3) [point]  {}; &    \node (digit) [terminal]    {digit};            &
    \node (p4) [point]  {}; &    \node (p5)    [point]  {};                      &
    \node (p6) [point]  {}; &    \node (e)     [terminal]    {E};                &
    \node (p7) [point]  {}; &                                                    &
    \node (p8) [point]  {}; &    \node (ui2)   [nonterminal] {unsigned integer}; &
    \node (p9) [point]  {}; &    \node (p10)   [point]       {};\\
    % Third row:
    & & & & & & &  & & & & \node (minus)[terminal] {-};\\
  };

  \graph [use existing nodes] {
    p1 -> ui1 -- p2 -> dot -- p3 -> digit -- p4 -- p5  -- p6 -> e -- p7 -- p8 -> ui2 -- p9 -> p10;
    p4 ->[skip loop=-5mm]  p3;
    p2 ->[skip loop=5mm]   p5;
    p6 ->[skip loop=-11mm] p9;
    p7 ->[vh path] { plus, minus } -> [hv path] p8;
  };
\end{tikzpicture}



\end{document}
