\documentclass[a4paper]{article}
\usepackage{fixltx2e} % LaTeX patches, \textsubscript
\usepackage{cmap} % fix search and cut-and-paste in Acrobat
\usepackage{ifthen}
\usepackage[T1]{fontenc}
\usepackage[utf8]{inputenc}

\usepackage{fancyhdr}
\pagestyle{fancy}

%%% Custom LaTeX preamble
% PDF Standard Fonts
\usepackage{mathptmx} % Times
\usepackage[scaled=.90]{helvet}
\usepackage{courier}
\usepackage{CJKutf8}


%%% User specified packages and stylesheets
%%% Fallback definitions for Docutils-specific commands

% providelength (provide a length variable and set default, if it is new)
\providecommand*{\DUprovidelength}[2]{
  \ifthenelse{\isundefined{#1}}{\newlength{#1}\setlength{#1}{#2}}{}
}

% lineblock environment
\DUprovidelength{\DUlineblockindent}{2.5em}
\ifthenelse{\isundefined{\DUlineblock}}{
  \newenvironment{DUlineblock}[1]{%
    \list{}{\setlength{\partopsep}{\parskip}
            \addtolength{\partopsep}{\baselineskip}
            \setlength{\topsep}{0pt}
            \setlength{\itemsep}{0.15\baselineskip}
            \setlength{\parsep}{0pt}
            \setlength{\leftmargin}{#1}}
    \raggedright
  }
  {\endlist}
}{}

% hyperlinks:
\ifthenelse{\isundefined{\hypersetup}}{
  \usepackage[colorlinks=true,linkcolor=blue,urlcolor=blue]{hyperref}
  \urlstyle{same} % normal text font (alternatives: tt, rm, sf)
}{}


%%% Body
\begin{document}
\begin{CJK}{UTF8}{gkai}

\section*{个人信息}

\begin{DUlineblock}{0em}
\item[] 姓名:么敬国
\item[] 性别:男
\item[] 手机:139-1033-4380
\item[] Email:yaojingguo@gmail.com
\end{DUlineblock}


\section*{教育}
\begin{itemize}
\item[] 北京理工大学,硕士,计算机应用技术,2004年4月
\item[] 北京理工大学,学士,计算机科学技术,2001年7月
\end{itemize}


\section*{自我评价}
\begin{itemize}
\item 10年企业级平台架构和研发经验。
\item 有以下企业级系统的架构经验:企业内容管理平台,企业流程管理平台,企业主数据管理平台,企业统一用户管理平台,CRM平台,数据仓库,大数据分析平台,统一消息平台。
\item 精通主流大数据平台的架构与系统建设,熟练掌握数据分析技术和数据挖掘技术。丰富的Hadoop平台(HBase, Zookeeper, Cassandra,Pig和Hive)上数据挖掘应用开发经验。
\item 精于计算机系统构建理论和技术,并发系统设计和分布式系统构建专家。
\item 精通Java(J2EE),C/C++,Python,Go, Lisp和Bash。熟练使用JavaScript,Haskell和Ruby。
\item 丰富的主流数据库系统使用和调优经验:Oracle, DB2, SQL Server和MySQL。
\item 精通Linux环境先的系统开发,精通TCP/IP协议,熟练掌握基本算法和数据结构。熟悉云计算虚拟化相关技术。
\end{itemize}

\section*{英语}
CET6优秀,口语流利,多年和海外团队合作经验。

\section*{教育,认证和培训}
\begin{itemize}
\item MIT-6.006 Introduction to Algorithms
\item MIT-6.033 Computer Systems Engineering
\item MIT-6.830 Database Systems
\item MIT-6.828 Operating System Engineering
\item MIT-6.824 Distributed Systems
\item IBM Certified Solution Designer DB2 Content Manager V8
\end{itemize}

\section*{工作经历}
\subsection*{海南天涯在线网络科技有限公司 (2010/08 -{}- 现在)}

\subsubsection*{天涯大数据部}
\begin{DUlineblock}{0em}
\item[] 总监
\item[] Hadoop,HBase,Pig, Mahout,C++, Python
\item[] 09/2012 -{}- 至今
\end{DUlineblock}
在天涯企业数据仓库的基础上开发天涯大数据分析平台:
\begin{itemize}
\item 天涯分布式锁系统:实现部分Google Chubby分布式锁功能,实现了Paxos算法和Replicated State Machine架构。
\item 开发天涯用户标签系统:根据用户在天涯上的行为,给用户打上标签,对客户进行细分。
\item 开发天涯推荐引擎:对用户行为数据和内容数据进行挖掘,提供个性化的推荐服务。
\item 开发天涯词库系统:业务人员使用的领域内关键词管理平台。
\item 数据挖掘,信息过滤,搜索,BI和会员团队管理。
\end{itemize}


\subsubsection*{天涯Hadoop数据分析平台}
\begin{DUlineblock}{0em}
\item[] 技术经理
\item[] Java, Lisp,Hadoop,Hive, JavaScript,MySQL,Oracle ODS, Oracle ODI,Oracle BIEE
\item[] 02/2011 -{}- 08/2012
\end{DUlineblock}
%
\begin{itemize}
\item 架构并带领团队建设新一代基于Hadoop/Hive的企业数据仓库来替换已有基于Oracle Data Warehouse的解决方案。
\item 用Hadoop实现协同过滤推荐算法。
\item 调查研究Hadoop平台技术,并向Apache提交Patch。
\end{itemize}

\subsubsection*{天涯虚拟化平台}
\begin{DUlineblock}{0em}
\item[] 资深系统架构师
\item[] Xen,KVM, QEMU
\item[] 12/2010 -{}- 01/2011
\end{DUlineblock}
天涯虚拟化平台的前期调研和设计工作。

\subsubsection*{Memlink Project}
\begin{DUlineblock}{0em}
\item[] Committer
\item[] C, Libevent
\item[] 09/2010 -{}- 11/2010
\end{DUlineblock}
\url{http://code.google.com/p/memlink}/是一个列表数据存储引擎。我设计实现了主从同步部分。


\subsection*{IBM中国软件开发中心 (2004/11 -{}- 2010/08)}

\subsubsection*{ECM/BPM CoE}
\begin{DUlineblock}{0em}
\item[] 技术经理
\item[] IBM ECM,IBM BPM,SPSS Data Modler,Oracle Database,LDAP
\item[] 06/2010 -{}- 08/2010
\end{DUlineblock}
ECM/BPM COE的职责是促进IBM ECM/BPM软件产品在中国的成长。我们的主要任务是为ECM/BPM建立重要的样板客户。
我作为COE Leader参加了银行业,保险,制造和核电行业的架构设计方案与实施。其中保险分析项目使用SPSS进行交叉
销售,使用了关联规则和决策树。

\subsubsection*{大数据查询语言Jaql}
\begin{DUlineblock}{0em}
\item[] 开发组长
\item[] 07/2009 -{}- 05/2010
\item[] Java,C++,Hadoop,Pig,SPSS Statistics,R,IBM BigData
\end{DUlineblock}
Jaql (\url{http://code.google.com/p/jaql})是IBM Hadoop大数据平台上的查询语言, 实现类似Hadoop Pig的功能。在这个
项目中,我和IBM Almaden Research Center的科研人员合作。 我是中国方面的technial leader, 同时是
Jaql项目的committer。我设计实现了Jaql catalog server,SPSS与Jaql的集成和JSON与CSV的转换。
Catalog Server提供一个key-value存储机制。SPSS与Jaql的集成集成的目标是结合SPSS的统计数据挖掘能力和Hadoop
平台的扩展性和并行计算机制。通过这个集成,可以平行的运行某些SPSS的算法。我还实现了JSON和CSV的转换,并做了
系统性能优化,达到了和Pig相当的性能。我还参与了VISA信用卡反欺诈项目。

\subsubsection*{泰康人寿主数据项目}
\begin{DUlineblock}{0em}
\item[] 架构师
\item[] 04/2009 -{}- 06/2009
\item[] IBM MDM,IBM DB2,Informatica
\end{DUlineblock}
我参与了主数据系统的架构设计和咨询。这个项目把泰康分散的客户数据和产品数据整合了在一起。整个项目分成3个阶段。第一阶段是ETL数据加载,
第二阶段是业务系统和主数据同步,第三阶段主数据系统开始提供对外服务。在泰康主数据平台基础之上,泰康开发实施了泰康分析型CRM系统,对
客户进行分类和个性化营销。

\subsubsection*{泰康人寿工作流项目}
\begin{DUlineblock}{0em}
\item[] 架构师/技术组长
\item[] 01/2009 -{}- 04/2009
\item[] J2EE,IBM ECM,IBM BPM,WebSphere,IBM MQ
\end{DUlineblock}
我带领了泰康人寿保全工作流系统的开发。开发团队有IBM实验室,IBM服务部门,IBM合作伙伴和泰康IT人员组成。
我架构了泰康核心业务系统的企业服务抽象。
我带领团队按时高标准的完成了泰康个人保险保全项目,并和泰康建立了深厚的客户关系。
在项目中,我设计实现了IBM Content Manager和Filenet BPM的集成。这个集成被当作最主要的功能点放入了IBM CM 8.4.1发行版,
并在美国IBM IOD大会上首发,获得非常大的反响。

\subsubsection*{泰康人寿影像中心项目}
\begin{DUlineblock}{0em}
\item[] 架构师
\item[] 08/2008 -{}- 12/2008
\item[] J2EE,IBM ECM,IBM MQ
\end{DUlineblock}
参与架构泰康人寿全集团企业级影像集中管理平台,基于企业级中间件IBM ECM和IBM MQ构建该解决方案,解决了网络环境不稳定的可靠性问题。

\subsubsection*{Rapid Industry Solution Enabler}
\begin{DUlineblock}{0em}
\item[] 开发组长
\item[] 06/2006 -{}- 07/2008
\item[] IBM ECM,UML,WebSphere Process Server,BPEL,Eclipse Plugin开发
\end{DUlineblock}
Rapid Industry Solution Enabler是一个代码生成工具,其目标是加速IBM ECM和BPM平台上的解决方
案开发。这个工具从UML数据模型和流程模型生成数据库schema,流程定义, Java代码和UI代码。所有这些构成了
一个可以运行的Web应用。这是一个IBM CDL和IBM Waston Research Center的合作项目。

\subsubsection*{ECM (企业内容管理)}
\begin{DUlineblock}{0em}
\item[] 软件工程师/开发组长
\item[] 11/2004 – 05/2006
\item[] Java,J2EE,IBM Rational Functional Tester
\end{DUlineblock}
这是一个IBM中国实验室和IBM Silicon Valley Lab的合作项目,为IBM ECM产品做开发和测试。我最初用Rational
Robot和Rational Functional Tester做产品界面测试,用ANT和JUnit做非界面测试。我后来加入了
Records Manager Enabler(一个ECM组件)的开发团队,设计实现了FIPS功能点。

\subsection*{北京理工大学 (09/2001 – 04/2004)}
研究方向包括Semantic Web和知识推理。我参加了几个基于J2EE的项目。

\section*{文章}
\begin{itemize}
\item Effective Java GUI automation on multiple platforms. IBM developerWorks.
\url{http://www-128.ibm.com/developerworks/rational/library/05/1004_yao/?ca=dgr-lnxw06JavaGUI4Win-Linux}.
\item Integrate FileNet BPM with IBM Content Manager,
Part 3: Implement the Component Integrator-based work performers.
IBM developerworks.
\url{http://www.ibm.com/developerworks/data/library/techarticle/dm-0805yao/index.html}.
\item 一个语义Web应用研究:旅游信息系统{[}J{]}。 北京理工大学学报(Ei核心刊物)。
\item 语义Web系统及其实现{[}J{]}。 北京理工大学学报(Ei核心刊物)。 2004,24(2):145-149。
\item 基于奇异值分解的中文Onotology自动学习技术{[}J{]}。 计算机工程(Ei Page One刊物),2003,29(9):137-139。
\item CRL:对语义Web上的Ontology表示语言DAML-OIL的一种扩充方案{[}J{]}。 计算机工程与应用(核心期刊),2003-23。
\item 一个语义Web架构及其实现{[}J{]}。 计算机工程与应用(核心期),2003-15。
\end{itemize}

\end{CJK}
\end{document}


