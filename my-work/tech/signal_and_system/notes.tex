\documentclass{book}

\usepackage{amsmath}
\usepackage{enumitem}
\usepackage{amsthm}
\usepackage{amssymb}
\usepackage{extarrows}
\usepackage{graphicx}
\usepackage{tikz}
\usetikzlibrary{
	cd,
	math,
	decorations.markings,
	decorations.pathreplacing,
	positioning,
	arrows.meta,
	circuits.logic.US,
	shapes,
	calc,
	fit,
	quotes}


\begin{document}

\chapter{Difference equations}

\section{Rabbits}

\begin{quote}
  Rather than growing approximately exponentially, this sequence is exactly periodic. Why? Furthermore, it has period 6.
  Why? How can this period be predicted without simulation?
\end{quote}

Answer:
\begin{center}
\begin{tabular}{ |c|c|c|c|c|c|c|c| } 
 \hline
           & 2 & 3 & 4  & 5  & 6  & 7 & 8 \\
\hline \hline
  $g[n-2]$ & 1 & 1 & 0  & -1 & -1 & 0 & 1 \\ 
  \hline
  $g[n-1]$ & 1 & 0 & -1 & -1 & 0  & 1 & 1 \\ 
 \hline
\end{tabular}
\end{center}

\subsection*{Prolem 1.2}
The reason is that the quesss of $f(n)$ is not correct when $n$ is large.

\subsection*{Problem 1.4}
For $h(n\tau)$, we have $|\frac{dh}{dt}| = \frac{h}{\tau}$. For $h[n]$, $|\frac{dh}{dt}|=\frac{h{0}}{\tau}$. Since $h \le h[0]$ in the first interval, the
formater is no larger that the latter. So the former is less than or equal to the later. The lower curve is $h[n]$.

\subsection*{Problem 1.5}
\begin{center}
  \begin{tabular}{|c|c|}
    \hline
    $T = \frac{\tau}{2}$ & $h[n]=h[0](\frac{1}{2})^n$ \\
    \hline
    $T = \tau$ & $h[n] = 0$ \\
    \hline
    $T = 2\tau$ & $h[n] = h[0](-1)^n$ \\
    \hline
    $T = 3\tau$ & $h[n] = h[0](-2)^n$ \\
    \hline
  \end{tabular}
\end{center}
It is obvious that $h[n] = h[0](1-\frac{T}{\tau})^n$ is not a good approximation of $h(n)$ if $T$ is big.

\end{document}
