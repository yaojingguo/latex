\documentclass{book}

\usepackage{clrscode3e}
\usepackage{upgreek}
\usepackage{listings}
\usepackage{color}
\usepackage{hyperref}
\usepackage{amsmath, amssymb, amsthm}
\usepackage{mathabx}
\usepackage{CJKutf8}
\usepackage[open]{bookmark}


\begin{document}
\begin{CJK}{UTF8}{gkai}
\part{Foundations}
\chapter{The Role of Algorithms in Computing}
\chapter{Get Started}
\chapter{Growth of Functions}

Why $F_i=\lfloor \frac{\phi ^ i}{\sqrt{5}} + \frac{1}{2} \rfloor$?

$\frac{|\hat{\phi}^i|}{\sqrt{5}} < \frac{1}{\sqrt{5}} < \frac{1}{2}$

$F_i = \frac{\phi ^ i}{\sqrt{5}} + \frac{1}{2} - (\frac{1}{2}+ \frac{\hat{\phi}^i}{\sqrt{5}})$

$F_i + (\frac{1}{2}+ \frac{\hat{\phi}^i}{\sqrt{5}})= \frac{\phi ^ i}{\sqrt{5}} + \frac{1}{2}$

$\frac{1}{2}+ \frac{\hat{\phi}^i}{\sqrt{5}} \leq \frac{1}{2}+ \frac{|\hat{\phi}^i|}{\sqrt{5}} < \frac{1}{2} + \frac{1}{2}=1$

$\frac{1}{2}+ \frac{\hat{\phi}^i}{\sqrt{5}} \geq \frac{1}{2} - \frac{|\hat{\phi}^i|}{\sqrt{5}} > \frac{1}{2} - \frac{1}{2}=0$

So $0<\frac{1}{2}+ \frac{\hat{\phi}^i}{\sqrt{5}}<1$

$F_i<F_i + (\frac{1}{2}+ \frac{\hat{\phi}^i}{\sqrt{5}})<F_i+1$

 $F_i<\frac{\phi ^ i}{\sqrt{5}} + \frac{1}{2}<F_i+1$
 
 So $F_i=\lfloor \frac{\phi ^ i}{\sqrt{5}} + \frac{1}{2} \rfloor$

\subsection*{3.2-6}

$\phi^2=(\frac{1+\sqrt{5}}{2})^2=\frac{3+\sqrt{5}}{2}$ and $\phi +
1=\frac{1+\sqrt{5}}{2} = \frac{3+\sqrt{5}}{2}$. So $\phi$ satisfies $x^2=x+1$

$\hat{\phi}^2=(\frac{1-\sqrt{5}}{2})^2=\frac{3-\sqrt{5}}{2}$ and $\hat{\phi} +
1=\frac{1-\sqrt{5}}{2} = \frac{3-\sqrt{5}}{2}$. So $\hat{\phi}$ satisfies $x^2=x+1$

\subsection*{3.2-7}
$F_0 = 0$ and $\frac{\phi^0 - \hat{\phi}^0}{\sqrt{5}}=\frac{1-1}{\sqrt{5}}=0$

$F_1=1$ and $\frac{\phi^1 - \hat{\phi}^1}{\sqrt{5}}=\frac{\frac{1+\sqrt{5}}{2}
-\frac{1-\sqrt{5}}{2}}{\sqrt{5}}$

Assume $F_i = \frac{\phi^i-\hat{\phi}^i}{\sqrt{5}}$ for $i \geq 1$.


\begin{align*}
  F_{i+1}  & = F_{i} + F_{i-1} \\
            &  = \frac{\phi^i - \hat{\phi}^i}{\sqrt{5}} + \frac{\phi^{i-1} -
\hat{\phi}^{i-1}}{\sqrt{5}} \\
&  = \frac{\phi^i - \hat{\phi}^i + \phi^{i-1}- \hat{\phi}^{i-1}}{\sqrt{5}} \\
            &  = \frac{(\phi^i + \phi^{i-1})- (\hat{\phi}^i + \hat{\phi}^{i-1})}{\sqrt{5}}
\end{align*}

Since $\phi^i + \phi^{i-1}=\phi^{i-1}(\phi+1)=\phi^{i-1}\cdot\phi^2=\phi^{i+1}$
and $\hat{\phi}^i +
\hat{\phi}^{i-1}=\hat{\phi}^{i-1}(\hat{\phi}+1)=\hat{\phi}^{i-1}\cdot\hat{\phi}^2=\hat{\phi}^{i+1}$.
We have $F_{i+1}=\frac{\phi^{i+1}-\hat{\phi}^{i+1}}{\sqrt{5}}$

\chapter{Divide-and-Conquer}
\subsection*{Unknown point list}
\begin{itemize}
\item solution to 4.3-i
\item permutation and fixed point (mentioned in solution to 5.2-4)
\end{itemize}

\section{The maximum-subarray problem}
\section{Strassen's algorithm for matrix multiplication}
\section{The substitution method for solving recurrences}

\subsection*{4.3-6}
Show that the solution to $T(n) = 2T(\lfloor n/2 \rfloor + 17) + n$ is $O(n \lg n)$.

Assume that $T(n) = 1$, for all $0 < n < 35$. We need to show that for some $n_0 > 0$, 
and some fixed constant $c > 0$,
\begin{align*}
T(n) \le cn\lg n
\end{align*}
for all $n \ge n_0$. Let $n_0 = 35$.

If $T(n) \le c(n-34)\lg(n-34) - n$ then $T(n) \le cn\lg n$ by the transitivity of 
inequalities because $c(n-34)\lg(n-34) -n \le cn\lg n$. 

\textbf{Base case:} let $n = 36$.
\begin{align*}
T(35) = 2T(34) + 35 = 2 + 35 = 37 \\
T(36) = 2T(35) + 36 = 2 * 37 + 36 = 110
\end{align*}
and $c(n-34)\lg(n-34)-n = c2\lg2 - 36 = 2c - 36 \ge 110$ when $c \ge 73$.

\textbf{Inductive step}: Suppose $T(n) \le c(n-34)\lg(n-34) - n$ for all $n\ge n_0$.
\begin{align*}
T(n)  & = 2T(\lfloor n/2 \rfloor + 17) + n \\
& \le 2(c((\lfloor n/2 \rfloor+17)-34)\lg((\lfloor n/2\rfloor+17)-34)-(\lfloor n/2 \rfloor + 17)) + n \\
& \le 2(c(n/2+17-34)\lg(n/2+17-34)-((n-1)/2+17)) + n \\
& = c(n-34)\lg((n-34)/2)-(n-1+34) + n \\
& = c(n-34)(\lg(n-34)-\lg2)-33 \\
& = c(n-34)(\lg(n-34)-1)-33 \\
& = c(n-34)\lg(n-34) -cn+34c-33 \\
& = c(n-34)\lg(n-34) - n + ((1-c)n + 34c - 33) \\
& \le c(n-34)\lg(n-34) -n 
\end{align*}
The last inequality holds when 
\begin{align*}
((1-c)n+34c-33) <= 0 \\
n\ge(34c-33)/(c-1) 
\end{align*}
Let $c = 73$. The above inequality results $n \ge 35$. Therefore, $T(n) = O(n\lg n)$.

\subsection*{4.3-b}
We guess by the means of recursion tree. The size of level $i$ is $n/3^i$. Solve 
$n/3^i = 1$, we have $\log_3 n$. So the depth of the recursion tree is $\log_3n$. 
Level cost is 
\begin{equation}
\frac{3^i (n/3^i)}{\lg(n/3^i)} = \frac{n}{\lg3(\log_3 n - i)}
\end{equation}

The sum for all levels is

\begin{align*}
\sum_{i=0}^{\log_3 n - 1} \frac{n}{\lg3(\log_3 n - i)} & = \frac{n}{\lg3}\sum_{i=1}^{\log_3 n} \frac{1}{i} \\
& = \frac{n}{\lg3} \Uptheta (\lg\log_3 n) \\
& = \Uptheta(n\lg\log_3 n) \\
& = \Uptheta(n\lg\lg n)
\end{align*}

We will show  $T(n)\le \log_3 2 \cdot n(1 + H_{\lfloor log_3 n \rfloor})$ and 
$T(n)\ge \log_3 2 \cdot n \cdot H_{\lceil \log_3 n \rceil}$,
where $H_k = 1 + 1/2 + 1/3 + \dotsi + 1/k$. We also define $H_0 = 0$.
Since $H_k = \Uptheta(\lg k)$, 
we have that $H_{\lfloor\log_3 n\rfloor} = \Uptheta(\lg\lfloor \log_3 n \rfloor)
= \Uptheta(\lg\log_3 n)$ and $H_{\lceil\log_3 n\rceil} = \Uptheta(\lg\lceil \log_3 n \rceil)
= \Uptheta(\lg\log_3 n)$.

The base case for the proof is for $n=1$, and we use $T(1) = 1$. Here, $log_3 n = 0$, 
so that $log_3 n = \lfloor\log_3 n\rfloor = \lceil\log_3 n\rceil$. Since $H_0 = 0$, we 
have $T(1) = 1 \le 1(1+H_0)$ and $T(1) = 1 \ge 0 = 1 \cdot H_0$.

For the upper bound of $T(n) \le n(1+H_{\lfloor \log_3 n \rfloor})$, we have

\begin{align*}
T(n) & = 3T(n/3) + n /\lg n \\
& \le 3\log_3 2 \cdot (n/3)(1+H_{\lfloor\log_3 \frac{n}{3} \rfloor}) + n/\lg n \\
& = \log_3 2 \cdot n(1+H_{\lfloor\log_3 n - 1\rfloor}) + \log_3 2 \cdot n/\log_3 n \\
& = \log_3 2 \cdot n(1+H_{\lfloor\log_3n -1\rfloor} + 1/\log_3 n) \\
& \le \log_3 2 \cdot n(1+H_{\lfloor \log_3 n \rfloor - 1} + 1/\lfloor \log_3 n \rfloor) \\
& = \log_3 2 \cdot n(1+H_{\lfloor \log_3 n \rfloor})  
\end{align*}
where the last line follows from the identity $H_k = H_{k-1} + 1/k$. 
The lower bound of $T(n)\ge \log_3 2 \cdot n \cdot H_{\lceil \log_3 n \rceil}$ is similar:
\begin{align*}
T(n) & = 3T(n/3) + n/\lg n \\
& \ge 3\log_3 2 \cdot (n/3)H_{\lceil\log_3\frac{n}{3}\rceil} + n/\lg n \\ 
& = \log_3 2 \cdot n(H_{\lceil\log_3 n - 1\rceil} + 1/\log_3 n) \\
& \ge \log_3 2 \cdot n(H_{\lceil\log_3 n\rceil - 1} + 1/\lceil\log_3 n \rceil) \\
& = \log_3 2 \cdot n \cdot H_{\lceil\log_3 n\rceil}
\end{align*}

\subsection*{4.3-d}
If there is not $-2$ in $n/3-2$, $T(n) = \Uptheta(n\lg n)$ from the master
theorem. So the guess is $T(n)=\Uptheta(n\lg n)$. 

The base case is $T(1)=T(2)=1$.

First, we prove that
$T(n)=O(n\lg n)$. We assume that $T(n) \le c\cdot (n+3)\lg(n+3)$.
\begin{align*}
T(n) & = 3T(n/3-2) + n/2 \\
& \le 3\cdot c\cdot ((n/3-2)+3)\lg((n/3-2)+3) \\
& = c(n+3)(\lg(n+3) - 1) + n/2 \\
& = c(n+3)\lg(n+3) + (1/2-c)n- 3c \\
& = c(n+3)\lg(n+3) - 3/2 \\
& \le c(n+3)\lg(n+3) 
\end{align*}
Choose $c=1/2$. $T(2) = 1 \le 3/2\cdot (2+3)\lg(2+3)$ and $T(1) = 1 \le
3/2\cdot (1+3)\lg(1+3)$.

Second, we prove that $T(n) = \Omega(n\lg n)$. We assume that 
$T(n) \ge c(n+3)\lg(n+3)$.
\begin{align*}
T(n) & \ge c(n+3)\lg(n+3) + (1/2-c)n - 3c \\
& = c(n+3)\lg(n+3) + \frac{49}{100}n-3/100 \\
& \ge c(n+3)\lg(n+3)
\end{align*}
Choose $c=1/100$. $T(2) = 1 \ge 1/100\cdot(2+3)\lg(2+3)$ and 
$T(1) = 1 \ge 1/100\cdot(1+3)\lg(1+3)$.

\subsection*{4.3-i uncomplete}
$T(n) = T(n-2) + 1/\lg n$. The depth is $(n-1)/2$.
\begin{align*}
T(n) &= \sum_{i=0}^{(n-1)/2-1} \frac{1}{\lg n} \\
& = \sum_{i=3/2}^{n}\frac{1}{1 + \lg n}
\end{align*}

\subsection*{4.3-g $T(n) = \sqrt{n}T(\sqrt{n}) + n$}
The size of level $i$ is $n^{\frac{1}{2_i}}$. The depth of the recursion tree is 
$\lg\lg n$. The cost for level $i$ ($0 \le i \le \lg\lg n $) is $n$ (assmue that 
$T(2) = 2$). So $T(n) = n\lg\lg n$.
\begin{align*}
T(n) &= \sqrt{n}\sqrt{n}\lg\lg{\sqrt{n}} + n \\
&= n\lg(\frac{1}{2}\lg{n}) + n \\
&= n(\lg\lg{n} - 1) + n \\
&= n\lg\lg{n}
\end{align*}

\section{The recursion-tree method for solving recurrences}
\section{The master method for solving recurrences}

\subsection*{4.5-5}
Let $g(n) = cn^{log_b a + \epsilon}$ and $\epsilon \le log_b 1/c$.Assume that 
when $n\ge n_0$, $af(n/b)=cf(n)$ and $c<1$. So for $n=b^dm$, we have $a(b^dm) 
\ge (a/c)^d f(m)$. We can decrease $c$ to satisfy $f(m) \ge g(m)$. The proof of 
$b^dm$ case:
\begin{align}
f(n) &= f(b^dm) \\
& \ge (\frac{a}{c})^d f(m) \\
& \ge (\frac{a}{c})^d g(m) \\
& \ge (ab^\epsilon)^d g(m) \\
& = (b^d)^{log_b a + \epsilon} g(m) \\
& = g(b^dm) \\
& = g(n)
\end{align}
(5) holds since $\epsilon \le log_b 1/c$.

\section{Proof of the master theorem}
\section{Problems}

\subsection*{4.5}
\begin{flushleft}
\begin{itemize}
\item[\textbf{a.}] Let $g$ be the number of good chips and $n-g$ be the number 
of bad chips. Since $n/2$ chips are bad, $n-g > g$. We can always find a set G 
of good chips and a set B of bad chips of equal sizes.  Now, assume that the set 
B of bad chips conspire to fool the professor in the following way: for any test 
made by the professor, they declare themselves as 'good' and the chips in G as 
'bad'.  Notice that the chips in G report correct answers and then exhibit a 
symmetric behaviour: they declare themselves as 'good' and the chips in B as 
'bad'. This implies that whichever is the strategy based on the kind of test 
considered and used by the professor, the behaviour of the chips in G is 
indistinguishable from the behaviour of the chips in B . This does not allow the 
professor to determine which chips are good.

\item[\textbf{b.}] Assume that $n$ is even. Then we can match chip in pairs 
($c_{2i-1}, c_{2i}$), for $i = 1, \dots, n/2$ and test all these pairs. Then we 
throw away all pairs of chips that do not say 'good, good'. In this way, the 
pairs that we have discarded contain at least a bad chip.  Finally we take a 
chip from each pair. In this way, the good chips and bad chips lelf are divided 
by two. So the final set contains at most n/2 chips and we still keeping more 
good chips than bad ones. 

Now, assume $n$ is odd. Then, we test $\lfloor n \rfloor$ pairs constructed as 
before; this time, however, we don't test one chip, say, $c_n$. Our final set 
will contain one chip for every pair tested that reported 'good, good'.  
Moreover, according to the number of such chips are odd or even, we discard chip 
$c_n$ or put it into our final set.

First, assume that the number is $2k+1$. There are at least $k+1$ good chips in 
this set because of the precondition that good chips are no less than bad chips 
in this set.  If there are $k$ good chips in this set, then the number of bad 
chips is $k+1$. So we discard $c_n$ in this case.

Second, assume that the number is $2k$. Assume that $c_n$ is good. In this case, 
after place $c_n$ in the set, there are at least $k+1$ good chips. Assume that 
$c_n$ is $bad$. There are at least 2 more good chips than bad chips in the chips 
we tested. So there are more good chips than bad chips in the set. As the number 
is even. There are at lease $k+2$ good chips. Place $c_n$ in the set, there are 
still more good chips than bad chips in the set.

\item[\textbf{c.}] If more that $n/2$ of the chips are good, using the answer to 
question b, we can compute a single good chip. Now, in order to find all good 
chips, we can just test the good chip with all the others: for each pair of this 
type, knowing that at least one of the two chips in the pair is good is enough 
to decide whether the other chip is good or not.  This last stage takes $n-1$ 
more tests. Then, in order to prove that the goof chips can be found with 
$\Uptheta(n)$ pairwise tests, we just need to show that the number of tests to 
find a single good chip is $\Uptheta(n)$. To show this, we can write the number 
of tests T(n) that we do in the answer to b as:

\begin{equation}
T(n) = T(\lceil n/2 \rceil) + \lfloor n/2 \rfloor \label{clever}
\end{equation}

We can use master theorem to solve this equation. $a = 1, b = 2, n^{\log_b a} = 
1, f(n) = \lfloor n/2 \rfloor$. $f(n) = \Omega(n^{0+1}), \epsilon=1$. $af(n/b) = 
\lfloor n/4 \rfloor \le \frac{1}{2} \lfloor n/2 \rfloor$. This inequality can be 
easily proved be writing $n=4k, n=4k+1, n=4k+2, n=4k+3$.
\end{itemize}
\end{flushleft}

\subsection*{4.6 Monge arrays}
\textbf{a.} First to prove 'only if' part. If A is a $m \times n$ monge array, 
then for all $i$, $j$, $k$ and $l$ such that $1\le i < k \le m$ and $1\le j < l 
\le n$, we have $A[i,j] + A[k,l] \le A[i,l] + A[k,j]$. Let $k=i+1, l=j+1$. We 
have the desired result.

Now to prove the 'if' part. The condition is for all $i = 1, 2, \dots, m$ and $j 
= 1,2,\dots,n$, we have
\begin{equation*}
A[i,j] + A[i+1,j+1] \le A[i,j+1] + A[i+1, j]
\end{equation*}
First to prove
\begin{equation}
A[i,j] + A[k,j+1] \le A[i,j+1] + A[k, j]
\end{equation}
$1\le i < k \le m$ and $j = 1,2,\dots,n-1$. The base case is $k=i+1$. It holds 
fron the condition. Now assume that the following equation holds.
\begin{equation}
A[i,j] + A[k-1,j+1] \le A[i,j+1] + A[k-1, j]
\end{equation}
And we have
\begin{equation}
A[k-1, j] + A[k, j+1] \le A[k-1,j+1] + A[k,j+1]
\end{equation}

Add (10) and (11), remove same items, we have
\begin{equation*}
A[i,j] + A[k,j+1] \le A[i,j+1] + A[k, j]
\end{equation*}

Now to prove the monge array condition. The base case is $l=j+1$. It holds from 
(10). Now assume that
\begin{equation}
A[i,j] + A[k,l-1] \le A[i,l-1] + A[k,j]
\end{equation}

And we have
\begin{equation*}
A[i,l-1] + A[k, l] \le A[i,l] + A[k, l-1]
\end{equation*}

Add (13) and (14),  and remove same items. We have
\begin{equation}
A[i,j] + A[k,l] \le A[i,l] + A[k,j]
\end{equation}

\textbf{b.} Increase 22 to 24 in the first row. Another solution to replace 7 
with 4 on the second row and the third column.

\textbf{c.} Let $f(k)$ be the index of the column containing the leftmost 
minimum element of row $k$ and $f(k+1)$ for $k+1$ row. And assume that $f(k) > 
f(k+1)$. We have $A[k, f(k+1)] > A[k, f(k)]$ by the defintion of $f(i)$. And we 
have $A[k+1, f(k)] \ge A[k+1, f(k+1)]$. Add these two equalities, we have
\begin{equation*}
A[k, f(k+1)] + A[k+1, f(k)] > A[k, f(k)] + A[k+1, f(k+1)]
\end{equation*}
This inequality is in contradiction with the definiton of monge array.

\textbf{d.}
The recursive solution to the problem of computing $f(i)$ is based on the 
observation that if A is a Monge array and A' is a subarray obtained by 
selecting a subset of rows of A, then A' is also Monge. Assuming we already 
computed the $f(i)$ for the even $i$, we can compute each odd index $f(2k+1)$ by 
observing that
$f(2k) \le f(2k + 1) \le f(2k + 2)$ by the previous problem.  Hence in order to 
compute $f(2k+1)$, we only need to check entries from $A[2k + 1, f(2k)]$ to 
$A[2k + 1, f(2k+2)]$. Therefore, the required computation can be done in
\begin{align*}
f(2) + \sum_{i=1}^{m/2} (f(2i+2)-f(2i)+1) & = f(m) + m/2 -1 \\
&  = O(m+n)
\end{align*}
because $f(m) \le n$. $f(2)$ appears as $f(2)$ entries have to be checked to 
compute $f(1)$.  Hence the worst-case running time is $O(m + n)$.  

\textbf{e.} Recursion equation:
\begin{equation*}
T(m,n) = T(m/2, n) + O(m+n)
\end{equation*}

We will prove $T(m,n) \le c(m+n\lg m)$.
\begin{align*}
T(m,n) & \le c(\frac{m}{2}+n\lg\frac{m}{2}) + d(m+n) \\
& = c(m+n\lg m) + (d-\frac{c}{2})m + (d-c)n \\
& \le c(m+n\lg m)
\end{align*}
The last line holds if $c\ge 2d$.

\chapter{Probabilistic Analysis and Randomized Algorithms}

\section{The hiring problem}

\subsection*{Balls and Bins}

\begin{align*}
  \text{Pr\{A bin is not hittedn in $m$ tosses\}} & = \prod_{i = 1}^{m} 
  \text{Pr\{A bin is not hitted in $i$th toss\}} \\
  & = \prod_{i = 1}^{m}\frac{n-1}{n} \\
  & = \left(\frac{n-1}{n}\right)^m
\end{align*}

\begin{align*}
  \text{Pr\{A bin is  hittedn in $m$ tosses\}} & = 1 - \text{Pr\{A bin is not 
  hitted in $m$ tosses\}} \\
  & = 1 - \left(\frac{n-1}{n}\right)^m \\
  & = \left(\frac{1}{n} + \frac{n-1}{n}\right)^m - \left(\frac{n-1}{n}\right)^m \\
  & = \sum_{k=0}^{m}\binom{n}{k}\left(\frac{1}{n}\right)^k 
  \left(\frac{n-1}{n}\right)^{m-k} - \left(\frac{n-1}{n}\right)^m \\
  & = \sum_{k=1}^{m}\binom{n}{k}\left(\frac{1}{n}\right)^k 
  \left(\frac{n-1}{n}\right)^{m-k}
\end{align*}

\subsection*{5.1-2}
The algorithm:
\begin{lstlisting}[language=Python]
import math
import random
def rand(a, b):
  d = math.floor(math.log(b, 2)) + 1
  n = 0;
  for i in range(d):
    n += (random.randint(0, 1) << i)
  if a <= n and n <= b:
    return n
  else:
    return rand(a, b)
\end{lstlisting}

The probability of $n$ hitting in the range $[a, b]$ is $\frac{b-a+1}{2^d}$. The 
invocation count of $rand$ function is a geometic distriction. The expect number 
is $\frac{1}{p} = \frac{2^d}{b-a+1}$.

\subsection*{5.1-3}
See \textbf{Instructor's Manual}

\subsection*{5.3-3}
This code does not produce a uniform random permutation. Some permuation will 
have a higher probability than others. Take an array with length 3 as an 
example. There are $3!=6$ permuations. But the code produces $3^3 = 27$ results.  
Since 27 can't be evently divide by 6, some permuations must correspond to more 
results than others. Here is the code to verify it:
%\lstinputlisting[language=C]{/data/code/algorithm/prog/perm.c}

\subsection*{5.3-5}
The number of different results are $n^3 \cdot n^3 \dots n^3$. The number of 
permuations of $RANDOM(1, n^3)$ is $n^3 \cdot (n^3-1) \dots (n^3-(n-1))$. So
\begin{align*}
Pr & = \frac{n^3 \cdot n^3 \dots n^3}{n^3 \cdot (n^3-1) \dots (n^3-(n-1))} \\
& = (1-\frac{1}{n^3}) \dots (1-\frac{n-1}{n^3}) \\
& > 1-\frac{1}{n^3}-\dots-\frac{n-1}{n^3} \\
& = 1-\frac{n(n-1)}{2n^3} \\
& > 1-\frac{1}{n}
\end{align*}

$(1-a_1)(1-a_2)\dots(1-a_n) > 1-a_1-a_2-\dots-a_n, a_i>0 (i=1,\dots,n)$ can be 
easily proved with the use of induction.

\subsection*{5.3-6}
If two or more priorities are the same, regenerate the priorities. But there is 
a problem, it is possible that the procedure will never end if the RANDOM always 
produce some identitcal priorities. Maybe there is some better solutions.

\subsection*{5.3-7}
Prove the following statement by induction. An $l$-element subset S of $A[1], 
\cdots, A[n-(m-l)]$ is created by invocation of $RANDOM-SAMPLE(l, n-(m-l))$ 
($l=1,\dots, m$). Such each $l$-subset is likely by probability 
$\frac{1}{\binom{n-(m-l)}{l}}$. 

The base case is $l=1$.  S is $1$-element subset whose only element is chosen 
from $A[1], \dots, A[n-(m-1)]$ by probability of 
$\frac{1}{(n-(m-1))}=\frac{1}{\binom{n-(m-1)}{1}}$. The statement holds.

Now assume that S is a $k-1$-element subset by the invocation of $RANDOM(k-1, 
n-(m-(k-1)))$.  And each $k-1$-subset is likely by 
$\frac{1}{\binom{n-(m-(k-1))}{k-1}}$.  Now we need to prove that S is a 
$k$-element subset by the invocation of $RANDOM(k, n-(m-k))$.  Such each 
$k$-subset is equally likely by probability $\frac{1}{\binom{n-(m-k)}{k}}$.

There are two situations. One situation is $k$-subset which includes $A[n-k]$.  
The other is one which does not include $A[n-k]$. Define the former event as 
$E_1$ and the later as $E_2$.

Define $E_0$ the event that any particular $k-1$-subset occurs.Define $E_{11}$ 
as the event that $i$ in line 4 is $A[n-k]$.  Define $E_{12}$ as the event that 
$i \in S_{k-1}$. $S_{k-1}$ is $k-1$-subset created by the invocation of 
$RANDOM-SAMPLE(k-1, n-(m-(k-1)))$.
\begin{align*}
Pr\{E_1\} & = Pr\{E_0\}\cdot Pr\{E_{11} \cup E_{12}\} \\
& = \frac{1}{\binom{n-(m-(k-1))}{k-1}}\cdot(\frac{1}{n-(m-k)} + 
\frac{k-1}{n-(m-k)}) \\
& = \frac{1}{\binom{n-(m-(k-1))}{k-1}}\cdot\frac{k}{n-(m-k)} \\
& = \frac{1}{\binom{n-(m-(k-1))}{k-1}}\cdot\frac{1}{\frac{n-(m-k)}{k}} \\
& = \frac{1}{\binom{n-(m-k)}{k}}
\end{align*}

To produce a particular $k$-subset which does not include $A[n-(m-k)]$. First,  
a $k-1$-subset must contain the $k-1$ elements chosen from the $k$-subset.  
Define such an event as $E_21$. Second, the 1 element left in $k$-subset must be 
chosen.
\begin{align*}
Pr\{E_1\} & = Pr\{E_{21}\}\cdot Pr\{E_{22}\} \\
& = \frac{\binom{k}{k-1}}{\binom{n-(m-(k-1))}{k-1}} \cdot \frac{1}{n-(m-k)} \\
& = \frac{1}{\binom{n-(m-(k-1))}{k-1}} \cdot \frac{1}{\frac{n-(m-k)}{k}} \\
& = \frac{1}{\binom{n-(m-k)}{k}}
\end{align*}

Now we can claim that every $k$-subset is likely by probability of 
$\frac{1}{\binom{n-(m-k)}{k}}$. Let $l=m$, we have that every $m$-subset is 
equally likely by probability $\frac{1}{\binom{n}{m}}$.

\subsection*{5.4-1}
\textbf{a} The probability that a person has the same birthday as me is $1/365$ 
(we don't take leap year into account). Define $E_1$ as the event that one 
person of $n$ persons has the same birthday as me. This is a $n$ Bernoulli 
trials, $p=1/365,, q=364/365$. Define $E_2$ as the event that none has the same 
birthday as me.
\begin{align*}
Pr\{E_1\} & = 1 - Pr\{E_2\} \\
& = 1 - \frac{\binom{n}{0}}{365} (\frac{364}{365})^n \\
& = 1 - (\frac{364}{365})^n\\
\end{align*}
Make the probability bigger than $1/2$, we have 253.
\textbf{b} This is also Bernoulli trials.  4. Make the following definitions:
\begin{description}
\item[$E_1$] at least two person have a birthday on July 4.
\item[$E_2$] no person has a birthday on July 4.
\item[$E_3$] only one person has a birthday on July 4.
\end{description}
\begin{align*}
Pr\{E_1\} & = 1 - P\{E_2\} - P\{E_3\} \\
& = 1 - \binom{n}{0}(\frac{364}{365})^n - 
\binom{n}{1}\frac{1}{365}(\frac{364}{365})^{n-1}
\end{align*}

Make the probability bigger than $1/2$, we can have the number.

\subsection*{5.4-2}
The following text is from 
\href{http://www.physicsforums.com/showthread.php?t=248541}{mikepol's solution}
First, I think the problem is stated a little ambiguously, so this is my 
understanding of it. Suppose I perform the following experiment: take b empty 
bins and start tossing balls into them at random. As soon as a ball lands in 
already occupied bin, I record the number of throws I had made. The problem asks 
to find the expected value of these recorded numbers, which will be the average 
over this experiment performed great many times. In other words, I need to find 
the expected number of tosses that result in a single collision on the last 
toss.

I have approached this from two paths.

First way is define a random variable X such that event X=k means that after k-1 
tosses, there were no collisions, but k'th toss resulted in a collission. In 
other words, X=k is the event that k'th toss results in a single bin having two 
balls, while all the rest have at most one. Then probability of X=k is:

\begin{align*}
P(X=k) = \frac{(b)_{k-1}(k-1)}{b^k}
\end{align*}

Where $(b)_{k}=\frac{b!}{(b-k)!}$. And the expected value that the problem is to 
compute is:

\begin{equation*}
E(X) = \sum_{k=2}^{b+1} kP(X=k) = \sum_{k=2}^{b+1} \frac{(b)_{k-1}(k-1)}{b^k}k
\end{equation*}

This is a solution in a form of a sum.

From the fact that X is a random variable and that it takes values from 2 to b+1, we also have:

\begin{equation*}
\sum_{k=2}^{b+1}P(X=k) = \sum_{k=2}^{b+1}\frac{(b)_{k-1}(k-1)}{b^k} = \frac{1}{b}\sum_{k=1}^{b}\frac{(b)_{k}k}{b^k}=1
\end{equation*}

A second way of getting an answer is define an indicator random variable $I_{k}$ 
such that the event $I_{k}=1$ means that there was k'th toss, and define 
$I_{k}=0$ otherwise. In other words, it means that previous tosses have resulted 
in no collisions, and therefore we made k'th toss, the result of which is 
unknown. The probability of this event is equals to the probability that 
previous k-1 tosses had no collisions. It is:

\begin{equation*}
P(I_{k}=1) = \frac{(b)_{k-1}}{b^{k-1}} = E(I_{k})
\end{equation*}

And random variable X can be expressed in terms of I's:

\begin{equation*}
X=\sum_{k=1}^{b+1}I_k
\end{equation*}

So by linearity of expectations we have:

\begin{equation*}
E(X)=\sum_{k=1}^{b+1}E(I_k) = \sum_{k=1}^{b+1}\frac{(b)_{k-1}}{b^{k-1}} = 
\sum_{k=0}^{b}\frac{(b)_k}{b^k}
\end{equation*}

\subsection*{5.4-3}
The birthday paradox holds even when the birthdays are only pairwise 
independent.  
$\mbox{Pr}\{b_i=b_j\}=\sum_{r=1}^n\mbox{Pr}\{b_i=r\quad\mbox{and}\quad b_j=r\}$
Obviously, it involves only two variables $b_i$ and $b_j$. Thus, pairwise 
independence is sufficient.

\subsection*{5.4-4}
\textbf{a. use probability: } Suppose that there are k people invited, and all 
years have $n=365$ days. $P_k = (n-k+1)/n$
$$\mbox{Pr\{there are exactly 3 people with the same birthday\}} = \frac{n \cdot C^{n-1}_{k-3} \cdot \frac{k!}{3!}}{n^k}$$
$$\mbox{Pr\{there are at least 3 people with the same birthday\}} = 1 - \frac{P^n_k}{n^k} - \frac{n \cdot C^{n-1}_{k-2} \cdot \frac{k!}{2!}}{n^k}$$

\textbf{b. use random variable: } The probability that $i$'s, $j$'s and $k$'s 
birthday all fall on day $r$ is
\begin{align*}
Pr\{b_i=r \;\text{and}\; b_j=r\;\text{and}\; b_k=r\} & = 
Pr\{b_i=r\}Pr\{b_j=r\}Pr\{b_k=r\} \\
& = 1/n_3.
\end{align*}

Thus, the probability that they all fall on the same day is
\begin{align*}
Pr\{b_i = b_j = b_k\} &= 
\sum_{r=1}^{n}Pr\{b_i=r\;\text{and}\;b_j=r\;\text{and}\;b_k=r\} \\
& = \sum_{r=1}^{n}(1/n^3) \\
& = 1/n^2
\end{align*}

For each pair $(i,j,k)$ of the $l$ people invited, we define the indicator 
variable $X_{ijk}$, for $1\le i \le j \le k \le l$, by

\begin{align*}
X_{ijk} & = \text{I\{person $i$, $j$ and $k$ have the same birthday\}} \\
& = \left\{
\begin{array}{rl}
  1 & \text{if person $i$, $j$ and $k$ has the same birthday,}  \\
  0 & \text{otherwise.}
\end{array} \right.
\end{align*}

Letting $X$ be the random variable that counts the number of pairs of people 
having the same birthday, we have
\begin{align*}
X = \sum_{i=1}^{l}\sum_{j=i+1}^{l}\sum_{k=j+1}^{l}X_{ijk}.
\end{align*}

Taking expectations of both sides and applying linearity of expectation, we 
obtain
\begin{align*}
E[X] & = E \Bigg[\sum_{i=1}^{l}\sum_{j=i+1}^{l}\sum_{k=j+1}^{l}X_{ijk} \Bigg] \\
& = \sum_{i=1}^{l}\sum_{j=i+1}^{l}\sum_{k=j+1}^{l}E[X_{ijk}] \\
& = \binom{l}{3}\frac{1}{n^2} \\
& = \frac{l(l-1)(l-2)}{6n^2}
\end{align*}
When $l(l-1)(l-2) \ge 6n^2$, the expected number of pairs of people with the 
same birthday is at least 1. Thus, if we have at least $\sqrt{6n^2}+2$ people 
invited, we can expect at least three to have the same birthday. The number is 
about 897.

\subsection*{5.4-5}
(1) $$Pr = \frac{n!}{n^k(n-k)!}$$
(2) n is the days all year have, k is how many people are invited to the party, 
and k-permutation denotes all people have distinct birthdays, no one has the 
same birthday as the other's.

\subsection*{5.4-7}
In the proof in subsection* 5.4.3, treat $\lfloor(\lg n)/2\rfloor$ as a function 
$f(n)$, replace the function with $\lfloor\lg n - 2\lg\lg n\rfloor$.
Then the probability required is at most:
\begin{align*}
  (1-2^{-f(n)})^{\lfloor \frac{n}{f(n)}\rfloor} & = (1-2^{2\lg\lg n - \lg n})^{\lfloor\frac{n}{f(n)}\rfloor}\\
  & = \bigg(1 - \frac{(\lg n)^2}{n}\bigg)^{\lfloor\frac{n}{f(n)}\rfloor} \\
  & = \bigg(1 - \frac{(\lg n)^2}{n}\bigg)^{\lfloor\frac{n}{\lfloor \lg n - 2\lg \lg n \rfloor }\rfloor} \\
  & \le \bigg(1 - \frac{(\lg n)^2}{n}\bigg)^{\frac{n}{\lfloor \lg n - 2\lg \lg n \rfloor }-1} \\
  & \le \bigg(1 - \frac{(\lg n)^2}{n}\bigg)^{\frac{n}{\lg n - 2\lg \lg n}-1} \\
  & \le e^{-\frac{(\lg n)^2}{n} (\frac{n}{\lg n - 2\lg \lg n} - 1)} \qquad, (1+x) \le e^x\\
& = O(e^{-\lg n}) \\
& = O(\frac{1}{n})
\end{align*}

We can prove that for sufficient large $n$, $\frac{(\lg n)^2}{n} (\frac{n}{\lg n - 2\lg \lg n} - 1) \ge \lg n$. Divide 
$\lg n$, we have:

\begin{align*}
\frac{(\lg n)^2}{n} (\frac{n}{\lg n - 2\lg \lg n} - 1) \frac{1}{\lg n} & \ge 1 \\
\frac{(\lg n)}{n} (\frac{n}{\lg n - 2\lg \lg n} - 1) & \ge 1\\
\frac{n\lg n - (\lg n)^2 + 2 \lg n\lg \lg n}{n\lg n - 2n\lg\lg n} \ge 1
\end{align*}

To prove the above inequality, we use the denominator to minus the numerator. We have:
\begin{align}
2n\lg\lg n - (\lg n)^2 + 2\lg n \lg\lg n & \ge n - (\lg n)^2 \\
 & \ge 0
\end{align}

Any positive polynomial function grows faster than any polylogrithmic function 
and we can find a sufficient large constant $n_0$ to let (2.2) holds, (2.2) 
holds.

\subsection*{5-2 Searching an unsorted array}
\textbf{a:} The code:
\begin{codebox}
\Procname{$\proc{Random-Search}(A, x)$}
\li Let $F[1..n]$ be a a new array
\li \For $i \gets 1$ \To $n$
\li \Do $A[i] \gets 0$
\End
\li $count \gets 0$
\li \While $count < n$
\Do \li $k = \proc{Random}(1, n)$
\li \If $x \isequal A[i]$
\li \Then \Return $k$
\End
\li \If $F[k] \isequal 0$
\li \Then $count \gets count + 1$
\li $F[k] \gets 1$
\End
\End
\end{codebox}

\textbf{b:} The number of generated indexes obeys geometric distribution. $p$ is 
$1/n$. So the expected number is $1/p=n$.

\textbf{c:} Solved in a way similar to b. The expected number is $n/k$.

\textbf{d:} This problem is the same "How many balls must we toss before every 
bin contains at least one ball?". So the answer is $n(\ln n + O(1)) 
$.

\textbf{e:} The average-case running time:
\begin{align*}
E[X] & = \sum_{i = 1}^{n} i Pr\{\text{running time is $i$}\}\\
& = \sum_{i = 1}^{n} i (1/n) \\
& = (n+1)/2
\end{align*}
The worst-case running time is $n$.

\textbf{f:}
Define $E_i$ as the event that the $i$ element is a $x$ which appears before all 
the other $x$s.
The probability of the running time is ($1 < i \le n -k +1$):
\begin{align*}
Pr\{E_i\} & = Pr\{\text{$1$ element is not $x$}\} \dots Pr\{\text{$i-1$ element 
is not $x$}\} Pr\{\text{$i$ element is $x$}\} \\
& = \underbrace{\frac{n-k}{n} \dots \frac{n-k-i+2}{n-i+2} 
\frac{k}{n-i+1}}_\text{$i$ factors} \\
& = \frac{(n-i)!}{n!} \frac{(n-k)!}{(n-i-k+1)!} k \\
& = \frac{(n-i)!(n-k)!k}{n!(n-i-k+1)!} \\
& = \frac{k(n-k)!}{n!}\frac{(n-i)!}{(n-i-k+1)!} \\
& = \frac{k!(n-k)!}{n!}\frac{(n-i)!}{(n-i-k+1)!(k-1)!} \\
& = \frac{\frac{(n-i)!}{(n-i-k+1)!(k-1)!}}{\frac{n!}{k!(n-k)!}} \\
& = \frac{\binom{n-i}{k-1}}{\binom{n}{k}}
\end{align*}
Assign index to $k$ $x$-elements. There are $\binom{n}{k}$ combinations. If $i$ 
element is the first $x$-element, $n-i$ positions will be assinged to $k-1$ 
$x$-elements. So the probability that $i$ element is the first $x$-element is 
$\frac{\binom{n-i}{k-1}}{\binom{n}{k}}$.


The average-case running time:
\begin{align*}
E[X] & =\sum_{i=1}^{n-k+1}\frac{\binom{n-i}{k-1}}{\binom{n}{k}}i \\
& = \frac{n+1}{k+1}
\end{align*}
\href{http://stackoverflow.com/questions/5125525/average-case-running-time-of-linear-search-algorithm}{answer 
3} is a great explanation.

The worst-case running time is $n-k+1$.

\textbf{g:} Both the average-case running time and the worst-case running time are $n$.

\textbf{h:} Besides the elements comparisons, \proc{Scramble-Search} incurs the 
cost of permuting the input array which is $n$. For $k=0$, both the worst case 
and the average-case are $n+n=2n$. For $k=1$, the worst case happens when $x$ is 
shuffled to the end of the array. The cost is $n+n=2n$. The average-case is 
$n+\frac{n+1}{2} = \frac{3n+1}{2}$. For generalized case, the worst case happens 
when all $x$s are shuffled to the end of the array. The cost is 
$n+(n-k+1)=2n-k+1$.  The average case is $n+\frac{n+1}{k+1}$.

\textbf{i:} I choose \proc{Deterministic-Search} since it is simple and 
efficient for general case unless I have specific requirements. The other 2 
algorithms all incurs cost related to random number generations.

\part{Sorting and Order Statistics}

\chapter{Heapsort}
Prove the following equation for $h$ = $0$, ..., $\lfloor\lg n\rfloor$
\begin{equation}
  \lceil\frac{n}{2^{h+1}}\rceil \le \frac{n}{2^h}
\end{equation}

\begin{proof}
  For $h = \lfloor\lg n\rfloor$
  \begin{align*}
    \lceil\frac{n}{2^{h+1}}\rceil & = \lceil\frac{2^{\lg n}}{2^{\lfloor\lg 
n\rfloor+1}}\rceil \\
& = \lceil 2^{\lg n - (\lfloor \lg n \rfloor + 1)}\rceil
  \end{align*}
  Since $\lg n < \lfloor \lg n \rfloor + 1$, we have $\lg n - (\lfloor \lg n 
  \rfloor + 1) < 0$. Then we have $0 < \lceil 2^{\lg n - (\lfloor \lg n \rfloor + 
  1)}\rceil < 1$. So $\lceil\frac{n}{2^{h+1}}\rceil = 1$.

  \begin{align*}
    \frac{n}{2^h} & = \frac{2^{\lg n}}{2^{\lfloor\lg n\rfloor}} \\
                  & = 2^{\lg n - \lfloor \lg n \rfloor} \\
                  & \ge 1\text{  }(\lg n - \lfloor \lg n \rfloor \ge 0)
  \end{align*}
  So we have $\lceil\frac{n}{2^{h+1}}\rceil \le \frac{n}{2^h}$ for h = $\lfloor\lg 
  n \rfloor$.

  For $h$ = $0$, ..., $\lfloor\lg n \rfloor - 1$
  \begin{align*}
    \lceil\frac{n}{2^{h+1}}\rceil & \ge \lceil\frac{n}{2^{\lfloor\lg n\rfloor - 
1+1}}\rceil \\
& = \lceil\frac{2^{\lg n}}{2^{\lfloor\lg n\rfloor}}\rceil \\
& \ge 1
  \end{align*}
  \begin{align*}
    \lceil\frac{n}{2^{h+1}}\rceil & < \frac{n}{2^{h+1}} + 1 \\
                                  & \le \frac{n}{2^{h+1}} + \frac{n}{2^{h+1}} \\
                                  & = \frac{n}{2^h}
  \end{align*}
  So we have $\lceil\frac{n}{2^{h+1}}\rceil \le \frac{n}{2^h}$ for $h$ = $0$, 
  ..., $\lfloor\lg - 1$
\end{proof}

\subsection*{6.1-7}
For this proof, the following complete binary tree property is used:
\begin{itemize}
  \item If $2i > n$, the node $i$ does not have a left child.  Otherwise, its 
    left child is node $2i$.
  \item If $2i+1 > n$, the node $i$ does not have a right child.  Otherwise, its 
    right child is $2i+1$.
\end{itemize}.  These are properties are from page 125 of
\textit{Dropbox/book/algorithm/数据结构.pdf}.

If $n$ is even, $2i+1 > 2i = 2(\lfloor n/2 \rfloor +1) = n + 2$. So node 
$\lfloor n/2 \rfloor + 1$ does not have children. So it is obvious that leaves 
are $\lfloor n/2 \rfloor + 1$, $\lfloor n/2 \rfloor + 2$, ..., $n$.

if $n$ is odd, $2i+1 > 2i = 2(\lfloor n/2 \rfloor +1) = n -1 + 2 = n + 1$. This 
case is similar to the case that $n$ is even. We have the same conclustion.

So with the array representation of storing an $n$-element heap, the leaves are 
the nodes indexed by $\lfloor n/2 \rfloor + 1$, $\lfloor n/2 \rfloor + 2$, ..., 
$n$.

\subsection*{6.4-5}
"On the Best Case of Heapsort" has the answer. It is difficult.

\subsection*{6.5-4}
\proc{Max-Heap-Insert} invokes \proc{Heap-Increase-Key}. Setting the inserted 
node to $-\infty$ is to make sure that $key$ is not smaller that 
$A[A.heap-size]$.

\subsection*{6.5-5}
\textbf{Initialization:} When \proc{Heap-Increase-Key} is called, A is a heap.  
Then, the value of $A[i]$ is increased to $key$. Because $A[i]$ is increased, 
this update will not influence the max-heap properties of $A[i]$'s children.  
This update will only influence the max-heap properties of its parent. And 
$A[Parent(i)]$ is no less that $A[i]$'s children.

\textbf{Maintenance:} Before line 5, the only violation is $A[i]$ is larger that 
$A[Parent(i)]$. After the exchange, $i$ node maintains the max-heap property 
since $A[Parent(i)]$ is no less than $i$'s children. And after the exchange, 
$A[Parent(i)]$ has the max-heap property. But since $A[Parent(i)]$ is increased, 
$A[i]$'s parent's parent may not has the max-heap property. Set $i$ to 
$Parent(i)$. We have that $i$'s parent is not less than $i$' children.

\textbf{Termination:} There are two situations for the termination. When $i = 
1$, the subarray $A[1\dots A.heap-size]
$ satisfies the max-heap property except that $A[1]$ may be larger than $A[0]$.  
Since $A[0]$ doest not exist, $A[1\dots A.heap-size]$ is a heap. When 
$A[\proc{Parent}(i)]$ is no less than $A[i]$, the only violation does not exist. 
So A is a heap.

\subsection*{6.5-6}
\begin{codebox}
\Procname{$\proc{Heap-Increase-Key}(A, key)$}
\li \If $key < A[i]$
\li \Then \Error "new key is smaller than current key"
\End
\li \While $i > 1 $ and $A[\proc{Parent}(i)] < key$
\li \Do $A[i] \gets A[\proc{Parent}(i)]$
\li $i \gets \proc{Parent}(i)$
\End
\li $A[i] \gets key$
\end{codebox}

\subsection*{6.5-7}
\textbf{Queue:} Add a new field $current$ to a max heap. Set $current$ to 
integer's maximum value. Queue's $append$ method invokes 
$\proc{Max-Heap-Insert}(A, current)$ and decreases $current$ with 1. Queue's 
$remove$ method invokes $\proc{Heap-Extract-Max}$.

\textbf{Queue:} Add a new field $current$ to a max heap. Set $current$ to 
integer's minimum value. Queue's $append$ method invokes 
$\proc{Max-Heap-Insert}(A, current)$ and increases $current$ with 1. Queue's 
$remove$ method invokes $\proc{Heap-Extract-Max}$.

\textbf{6.5-8}
\begin{codebox}
\Procname{$\proc{Heap-Delete}(A, i)$}
\li $deleted \gets A[i]$
\li \If $i \isequal A.heap\mbox{-}size$
\li \Then $A.heap-size \gets A.heap\mbox{-}size - 1$
\li \Else $A[i] \gets A[A.heap\mbox{-}size]$
\li $A.heap-size \gets A.heap\mbox{-}size - 1$
\li $\proc{Max-Heapify}(A,i)$
\End
\li \Return $deleted$
\end{codebox}

\textbf{6.5-9}
\begin{codebox}
\Procname{$\proc{K-Way-Merge}(A, k)$}
\li $B[1\dots k]$, $C[1\dots k]$ and $D[1\dots \sum_{i=1}^{k}A[i].length]$ be 
new arrays
\li \For $i \gets 1$ \To $k$
\li \Do $B[i].no \gets i$
\li$B[i].key \gets A[i][1]$
\End

\li $\proc{Build-Min-Heap}(B)$

\li \For $i \gets 1$ \To $k$
\li \Do $C[i] \gets 2$
\End

\li \For $i = 1$ \To $B.length$
\li \Do $min \gets B[1]$
\li $D[i] \gets min.key$
\li $arrno \gets min.no$
\li $C[arrno] \gets C[arrno] + 1$
\li $B[1].no = arrno$
\li $B[1].key = A[arrno][C[arrno]]$
\li $\proc{Min-Heapify}(B)$
\End

\li \Return $D$
\end{codebox}

\subsection*{6.2}
\textbf{a:}
\begin{codebox}
\Procname{$\proc{D-Parent}(i)$}
\li \Return $\lfloor (i-2)/d + 1\rfloor$
\end{codebox}
\begin{codebox}
\Procname{$\proc{D-Ary-Child}(i,j)$}
\li \Return $d(i-1)+j+1$
\end{codebox}

\textbf{b:} The total number of nodes in a tree of height $h$ is between 
$\frac{d^{h+1}-1}{d-1}$ and $\frac{d^h-1}{d-1}+1$. Multiply with $d-1$ we have 
$d^{h+1}-1$ and $d^h+d-2$. Since $d \ge 2$, we have $h=\lfloor \lg_d n(d-1) 
\rfloor = \Uptheta(\lg_d n)$.

\textbf{c:} The procedure $\proc{Heap-Extract-Max}$ for binary heaps also works 
for $d$-ary heaps. The only needed update is $\proc{Max-Heapify}$.
\begin{codebox}
\Procname{$\proc{D-Ary-Max-Heapify}(A,i)$}
\li $largest \gets i$
\li \For $j \gets 1$ \To $d$
\li \Do $index = \proc{D-Ary-Child}(i,j)$
\li \If $index \le A.heap\mbox{-}size$ and $A[index] > A[largest]$
\li \Then $largest \gets index$
\End
\End
\li \If $largest \ne i$
\li \Then exchange $A[i]$ with $A[largest]$
\li $\proc{D-Ary-Max-Heapify}(A, largest)$
\End
\end{codebox}
$T(n) = \Uptheta(d\lg_d n)$

\subsection*{6.3}
\textbf{a:}
\begin{equation*}
\left(
\begin{array}{cccc}
2 & 3 & 4 & 5 \\
8 & 9 & 12 & 14 \\
16 & \infty & \infty & \infty \\
\infty & \infty & \infty & \infty \\
\end{array} \right)
\end{equation*}

\begin{tabular}{|c|c|c|c|}
\hline
2 & 3 & 4 & 5 \\ \hline
8 & 9 & 12 & 14 \\ \hline
16 & $\infty$ & $\infty$ & $\infty$ \\ \hline
$\infty$ & $\infty$ & $\infty$ & $\infty$ \\ \hline
\end{tabular}

\textbf{b:} First, argue the case that $Y[1,1] = \infty$.From the definition of 
Young tableau, we have $A[1,j] \ge A[1,1]$ and $A[i,j] \ge A[1,j]$ ($1\le i \le 
m, 1 \le j \le n$). So we have $A[i,j] \ge A[1,1]$.  Since $A[1][1]$ is 
$\infty$, $A[i][j]$ is $\infty$. So $Y$ is empty. Second, argue the case that 
$Y[m,n] < \infty$. From the definition of Young tableau, we have $A[m,j] \le
A[m,n]$ and $A[i,j] \le A[m,j]$. So we have $A[i,j] \le A[m,n]$. Since $A[m,n] < 
\infty$, we have $A[i,j] < \infty$. So $Y$is full.

\textbf{c:}

\begin{codebox}
\Procname{$\proc{Extract-Min}(A,m,n)$}
\li $min \gets A[1,1]$
\li $A[1,1] \gets -\infty$
\li $\proc{Extract-Min}(A,1,1,m,n)$
\li \Return $min$
\end{codebox}

\begin{codebox}
\Procname{$\proc{Extract-Min}(A,i,j,m,n)$}
\li $down \gets i+1$
\li $right \gets j+1$
\li \If \bf{not} ($down \le m$ \bf{and} $A[down,j] < A[i,j]$)
\li \Then $down \gets i$
\End
\li \If \bf{not} ($right \le n$ \bf{and} $A[i,right] < A[down,j]$)
\li \Then $right \gets j$
\li \Else $down \gets i$
\End

\li \If $down \ne i$ \bf{or} $right \ne j$
\li \Then exchange $A[i,j]$ with $A[down,right]$
\li $\proc{Extract-Min}(A,down,right,m,n)$
\End
\end{codebox}

\begin{align*}
T(p) & = T(p-1) + \Uptheta(1) \\
T(p) & = T(p-1) + b \\
\end{align*}
Assume $T(p) \le cp$. The base case $T(1) = d$. We can choose make $c \ge b$ and 
$c \ge d$.
\begin{align*}
T(p) & = T(p-1) + c \\
& \le c(p-1) + c \\
& =cp
\end{align*}

\textbf{d:}
\begin{codebox}
\Procname{$\proc{Insert}(A,m,n,key)$}
\li \If $A[m,n] \ne \infty$
\li \Then \Error "Young tableau is full"
\End
\li $A[m,n] \gets key$
\li $\proc{Decrease}(A,m,n)$
\end{codebox}

\begin{codebox}
\Procname{$\proc{Decrease}(A,i,j)$}
\li $up \gets i-1$
\li $left \gets j-1$
\li $largest_i \gets i$
\li $largest_j \gets j$
\li \If $up \ge 1$ and $A[up, j] > A[i,j]$
\li \Then $largest_i = up$
\li $largest_j = j$
\End
\li \If $left \ge 1$ and $A[i, left] > A[largest_i, largest_j]$
\li \Then $largest_i = i$
\li $largest_j = left$
\End
\li \If $largest_i \ne i$ or $largest_j \ne j$
\li \Then exchange $A[i,j]$ with $A[largest_i, largest_j]$
\li $\proc{Decrease}(A, largest_i, largest_j)$
\End
\end{codebox}
\textbf{Loop invariant:} $A[i\dots m, i\dots n]$ is a young tableau. Define 
$largest_i$ and $largest_j$ as the row and column indices of the largest element 
among $A[i,j], A[i-1,j]$ and $A[i,j-1]$. $A[largest_i, largest_j] \le A[i+1,j]$ 
and $A[largest_i, largest_j] \le A[i,j+1]$.

\textbf{Initialization:} $i=m, j=n$, $A[m\dots m, n\dots n]$ contains only a 
element $A[m,n]$. So $A[m\dots m,n\dots n]$ is a young tableau. And $A[i+1,j]$ 
and $A[i,j+1]$ do not exist. So the loop invariant holds.

\textbf{Maintenance:} $A[i,j]$ is replaced with $A[largest_i, largest_j]$. Since 
$A[largest_i, largest_j] \le A[i+1,j]$ and $A[largest_i, largest_j] \le 
A[i,j+1]$. So $A[i\dots m, j\dots n]$ is still a young tableau. 

Now lets explain the case when $largest_i = i-1$ and $largest_j = j$.  
$A[i-1\dots m, j\dots n]$ is produced by prefixing $A[i-1,j\dots n]$ row to the 
young tableau $A[i\dots m, j\dots n]$.  We have $A[i-1,j] \le A[i,p]$ ($j < p 
\le n$) since $A[i-1,j] \le A[i-1,p]$ before the element exchange and $A[i-1,j]$ 
is decreased by the element exchange.  Since $A[i-1,j+1\dots n]$ is young 
tableau row, $A[i-1,j\dots n]$ is a young tableau row.  $A[i-1,j]$ satisfies the 
yount tableau row property since $A[i-1,j] \le A[i,j]$.  $A[i-1,p]$ satisfies 
the yount tableau column property since $A[i+1\dots m, j+1 \dots n]$ is produced 
by increasing values of the young tableau $A[1\dots m, j+1\dots n]$. So 
$A[i-1\dots m,j\dots n]$ is a young tableau. 

No to prove that $A[largest_{i-1}, largest_j] \le A[i-1,j+1]$ and 
$A[largest_{i-1},largest_j] \le A[i,j]$. First to prove 
$A[largest_{i-1},largest_j] \le A[i-1,j+1]$.  Before the element exchange, we 
have $A[i-1,j-1] \le A[i-1,j] \le A[i-1,j+1]$ and $A[i-2,j] \le A[i-1,j] \le 
A[i-1,j+1]$. After the element exchange, we still have $A[i-1,j-1] \le 
A[i-1,j+1]$ and $A[i-2,j] \le A[i-1,j+1]$. And $A[i-1,j] \le A[i-1,j+1]$ since 
$A[i-i,j]$ is decreased by the element exchange. So we have $A[largest_{i-1}, 
largest_j] \le A[i-1,j+1]$.  Second to prove $A[largest_{i-1},largest_j] \le 
A[i,j]$. Before the element exchange, we have $A[i-2, j] \le A[i-1,j]$ and 
$A[i-1,j-1] \le A[i-1,j]$. And $A[i,j] < A[i-1,j]$ since $largest_i = i-1$ and 
$largest_j = j$.After the element exchange, we have $A[i-2,j] \le A[i,j]$ and 
$A[i-1,j-1] \le A[i,j]$. And we have $[i-1,j] < A[i,j]$.

The case for $largest_i = i$ and $largest_j = j-1$ is similar. 

So the loop invariant for $largest_i, largest_j$ holds.

\textbf{Termination:} When $largest_i = i$ and $largest_j = j$, $A[1\dots m, 
1\dots n]$ is a young tableau by the definition of young tableau.

\textbf{e:}
\begin{codebox}
\Procname{$\proc{Sort}(B)$}
\li \For $i \gets 1$ \To $n_2$
\li \Do $\proc{Insert}(A,n,n,B[i])$
\End
\li \For $i \gets 1$ \To $n_2$
\li \Do $B[i] \gets \proc{Extract-Min}(A,n,n)$
\End
\end{codebox}

\textbf{f:}
\begin{codebox}
\Procname{$\proc{Contain}(A,i,j,m,n,key)$}
\li \If $key \isequal A[i,j]$
\li \Then \Return True
\li \ElseIf $key < A[i,j]$
\li \Then \Return False
\End
\li $k \gets i$
\li \While $k+1 \le m$ and $key \le A[k+1,j]$
\li \Do $k \gets k+1$
\End
\li $l \gets j$
\li \While $l+1 \le n$ and $key \le A[i,l+1]$
\li \Do $l \gets l+1$
\End
\li \If $k > i$ and $l > j$
\li \Then $\proc{Contain}(A,i+1,j+1,k,l,key)$
\li \Else \Return False
\End
\end{codebox}
Let $p = m+n$. We have $T(p) = T(p-q) + \Uptheta(q)$. $T(p) \le T(p-q) + bq$. We 
will prove $T(p) \le cp$ ($c \ge b$).
\begin{align*}
T(p) & \le T(p-q) + bq \\
& \le c(p-q) + bq \\
& \le c(p-q) + cq \\
& = cp
\end{align*}
And we can increase $c$ to satisfy the base base. So $T(p) \le cp$. $T(p) = O(p) 
= O(m+n)$.


\chapter{Quicksort}
\subsection*{7.1-2} $q$ is $r$ when all element in the array $A[p\dots r]$ have the 
same value.
\begin{codebox}
\Procname{$\proc{Partition}$(A,p,r)}
\li $x \gets A[r]$
\li $i \gets p-1$
\li $equals \gets 0$
\li \For $j \gets p$ \To $r-1$
\li \Do \If $A[j] \isequal x$
\li \Then $equals \gets equals+1$
\End
\li \If $A[j] \le x$
\li \Then $i \gets i+1$
\li exchange $A[i]$ with $A[j]$
\End
\End
\li exchange $A[i+1]$ with $A[r]$
\li \If $equals \isequal r-1$
\li \Then \Return $\lfloor (p+r)/2 \rfloor$
\li \Else \Return $i+1$
\End
\end{codebox}

\subsection*{7.2-1} From $T(n) = T(n-1) + \Uptheta(n)$, we have $T(n) = T(n-1) + 
f(n)$. From the defintion of $\Uptheta$, we have $c_1 n \le f(n) \le c_2 n$ when 
$n \ge n_0$ ($c_1$, $c_2$ and $n_0$ are positive constants). And we have $d_1 n 
\le c_1 n \le f(n) \le c_2 n \le d_2 n$ ($0 < d_1 \le c_1$ and $d_2 \ge c_2$). 

First, we prove $T(n) = O(n_2)$. We assume that $T(n) \le e n_2$.
\begin{align}
T(n) & \le e(n-1)_2 + d_2 n \\
& = en_2 + (d_2-2e)n + d \\
& \le en_2
\end{align}
4.3 holds when $e \le d_2/2$. The base cases are $T(1) = 1$. We can increase 
  $d_2$ to satisfy it.

The proof for $T(n) = \Omega(n_2)$ is similar.

\subsection*{7.2-4} If the elements of the input array is in ascending order, then 
$\proc{Partition}$ always return $r$. So $T(n) = T(n-1) + \Uptheta(n)$. Solve 
it, we have $T(n) = \Uptheta(n_2)$. For $\proc{Insert-Sort}$, if the input array 
is in in ascending order, $T(n) = \Uptheta(n)$. So the latter is better for this 
use case.

\subsection*{7.2-6} The balance of the split depends on the $r$ element's relative 
ordering of the input array. When $A[r]$ is among the input array's $\alpha n$ 
smallest elements or $\alpha n$ largest elements, $\proc{Partition}$ produces a 
split which is less than $1-\alpha$ to $\alpha$.  We defines this case as event 
$E_1$.  The probability of $E_1$ is $2\alpha$. The event that $\proc{Partition}$ 
produces a split more balanced than $1-\alpha$ to $\alpha$ is $\bar{E_1}$. So 
the probability of $\bar{E_1}$ is approximately $1-2\alpha$.

\subsection*{7.3-2} Use $T(n)$ to denote the count of calls made to $\proc{Random}$.  
For the worst case, $T(n) = T(n-1) + 1$. Solve it, we have $T(n) = \Uptheta(n)$.  
For the best case, $T(n) = 2T(n/2) + 1$. This equation satisfies the case 1 of 
the Master Theorem. So $T(n) = \Uptheta(n)$.

\subsection*{7.4-1} Assume that $T(n) \ge cn_2$.
\begin{align*}
T(n) & = \max_{0\le q \le n-1} (T(q)+T(n-q-1)) + \Uptheta(n) \\
& \ge \max_{0\le g \le n-1} (cq^2 + c(n-q-1)^2) + \Uptheta(n) \\
& \ge c \cdot \max_{0\le g \le n-1} (q^2 + (n-q-1)^2) + \Uptheta(n)
\end{align*}
Since $\max_{0\le q \le n-1}(q^2+(n-q-1)^2) \le (n-1)^2 = n_2-2n+1$, we have:
\begin{align*}
T(n) & \ge cn^2-c(2n-1) + \Uptheta(n) \\
& \ge cn_2,
\end{align*}
since we can pick the constant $c$ small enough so that $\Uptheta(n)$ term 
dominates the $c(2n-1)$ term. Thus, $T(n) = \Omega(n^2)$.

\subsection*{7.4-4} From the discussion in the book, we know that $T(n) \ge X$.
\begin{align}
E[X] & = \sum_{i=1}^{n-1}\sum_{j=i+1}^{n}\frac{2}{j-i+1} \\
& = \sum_{i=1}^{n-1}\sum_{k=1}^{n-i}\frac{2}{k+1} \\
& = \sum_{i=1}^{n-1}\sum_{k=2}^{n-i+1}\frac{2}{k} \\
& \ge \sum_{i=1}^{n-1}\sum_{k=2}^{n}\frac{1}{k} \\
& = \sum_{i=1}^{n-1}\Big(\sum_{k=1}^{n}\frac{1}{k} - 1\Big) \\
& = \sum_{i=1}^{n-1}\sum_{k=1}^{n}\frac{1}{k} - (n-1) \\
& \ge \sum_{i=1}^{n-1}\ln(n+1) - (n-1) \\
& = n\ln(n+1) - \ln(n+1) - (n-1) \\
& = \Omega(n\ln n) \\
& = \Omega(n\lg n)
\end{align}

From $T(n) \ge X$ and $E[X] = \Omega(n\lg n)$, we have $T(n) = \Omega(n \lg n)$.  
(4.7) holds, since $\sum_{i=1}^{n-1}\sum_{k=2}^{n-i+1}\frac{1}{k}$ is the sum of 
all the elements in the following matrix multiplied by $2$:

\begin{equation*}
\left(
\begin{array}{ccccc}
\frac{1}{2} & \frac{1}{3} & \dots \frac{1}{n-1} & \frac{1}{n} \\
\frac{1}{2} & \frac{1}{3} & \dots \frac{1}{n-1} &  \\
\dots \\
\frac{1}{2} & \frac{1}{3} \\
\frac{1}{2}
\end{array} \right)
\end{equation*}

\begin{align*}
\sum_{i=1}^{n-1}\sum_{k=2}^{n-i+1}\frac{2}{k} & = 
\sum_{i=1}^{n-1}\Big(\sum_{k=2}^{n-i+1}\frac{1}{k} + 
\sum_{k=2}^{n-i+1}\frac{1}{k}\Big) \\
& \ge \sum_{i=1}^{n-1}\Big(\sum_{k=2}^{n-i+1}\frac{1}{k} + 
\sum_{k=2}^{n-i}\frac{1}{k}\Big) \\
& = \sum_{i=1}^{n-1}\Big(\sum_{k=2}^{n-i+1}\frac{1}{k} + 
\sum_{k=2}^{i}\frac{1}{k}\Big) \\
& = \sum_{i=1}^{n-1}\Big(\sum_{k=2}^{n-i+1}\frac{1}{k} + 
\sum_{k=2}^{i}\frac{1}{k}\Big) \\
& \ge \sum_{i=1}^{n-1}\Big(\sum_{k=2}^{n}\frac{1}{k}\Big) \\
\end{align*}

\subsection*{7.4-5} There are two parts of this sorting algorithm. One part is the 
time to run quicksort. Among this part, there are at most $n/k$ calls to 
partition and the comparsions between elements. Only $z_{ij}$ ($j-i \ge k$) has 
a chance to have a comparsion. So the expected comparsions number is:
\begin{align*}
E[X] & = \sum_{i=1}^{n-k+1}\sum_{j=i+k-1}^{n}\frac{2}{j-i+1} \\
& = \sum_{i=1}^{n-k+1}\sum_{l=k-1}^{n-i}\frac{2}{l+1} \\
& \le \sum_{i=1}^{n-k+1}\sum_{l=k-1}^{n-i}\frac{2}{l+1} \\
& \le \sum_{i=1}^{n-k+1}\sum_{l=k}^{n}\frac{2}{l} \\
& \le 2n\sum_{l=k}^{n}\frac{1}{l} \\
& = 2n \cdot O(\lg(\frac{n}{k})) \\
& = O(n\lg(\frac{n}{k}))
\end{align*}

Here is a proof for $\sum_{l=k}^{n} = O(\lg(\frac{n}{k}))$.
\begin{align*}
\sum_{l=k}^{n}\frac{1}{l} & \le \int_{k-1}^{n}\frac{\mathrm{d}x}{x} \\
& = \ln(n) - \ln(k-1) \\
& = \ln(\frac{n}{k-1}) \\
& = O(\ln(\frac{n}{k})) \\
& = O(\lg(\frac{n}{k}))
\end{align*}

The other part is to run insertion-sort. The cost of runing insertion-sort over 
q $k$ element arrray is $\Uptheta(k^2)$. There are $n/k$ such arrays. So the 
total cost of running insertion-sort is $\Uptheta(k^2)\cdot n/k = \Uptheta(nk)$.

So the total cost is $O(nk+n\lg(n/k))$. The other discussion is similar to the 
solutoin to Problem 2-1.

\subsection*{7.4-6} Define $E$ as the event for getting at best 
$\alpha-to-(1-\alpha)$ split. Define $x_1, x_2 and x_3$ as the three elements 
randomly chosen. And assume that $x_1 \le x_2 \le x_3$. There are 2 cases to get 
a $\alpha-to-(1-\alpha)$ split. One is when $x_1 \le \alpha$ and $x_2 \le 
\alpha$. The other is when $x_2 \le (1-\alpha)$ and $x_3 \le (1-\alpha)$. Due to 
their symmetriy, their probabilities are the same. Now let's compute the 
probability for the former. Choose 3 elements with a probability to choose a 
element which is among the $A$'s $\alpha n$ smallest elements. The probability 
is $\alpha$. This is a Bernoulli trial. So the probability for the for case is 
$\binom{3}{2}\alpha_2(1-\alpha) = 3\alpha_2(1-\alpha)$. So $Pr\{E\} = 
6\alpha_2(1-\alpha)$. $\bar{E}$ is the event for getting at worst 
$\alpha-to-(1-\alpha)$ split. $Pr\{\bar{E}\} = 1 - Pr{E} = 1 - 
6\alpha_2(1-\alpha)$.

\subsection*{7.1}
\textbf{b:} In the first iteration, $i$ and $j$ do not access an element of $A$ 
outside of $A[p\dots r]$. After the first iteration, $i=p$ and $p\le j \le r$.  
Let $k = j$ after the first iteration. If the subroutine does not return, we 
have $A[p] \le x$ and $A[k] = x$. In the following iterations, $j$ will never 
access an element outside of $A[p\dots r]$ since $A[p] \le x$. And $i$ will 
never access an element outside of $A[p\dots r]$ since $A[k] =x $.

\textbf{c:} In a:, we already have $p\le j \le r$. We only need to prove that $j 
< r$. If the subroutine terminates after the first subroutine. We have $i\ge j$ 
and $i = p, p < r$. So we have $j < r$. If the subroutine doest not terminate in 
the first iteration, $j$ will be decreased at least one time. So we still have 
$j < r$.

\textbf{d:} We will proove the following \textbf{loop invariant}: \\
At the start of every iteration, $A[p\dots i] \le x$, $A[j\dots r] \ge x$.\\
\textbf{Initialization:} Before the first iteration, $i = p-1$ and $j=r+1$.  
$A[p\dots p-1]$ and $A[r+1\dots r]$ are empty, the loop invariant holds.
\textbf{Maintenance:} During a iteration which does terminate the subroutine, 
$j$ is decreased to $j'$ until $A[j'] \ge x$. $i$ is increased to $i'$ until 
$A[i'] \ge x$. Then $A[i']$ is exchanged with $A[j']$. So $A[p\dots i'] \le x$ 
and $A[j'\dots r] \ge x$. \\
\textbf{Termination:} When the subroutine is terminated, $i \ge j$, $A[p\dots 
i-1] \le x$ and $A[j+1\dots r] \ge x$. When $i=j$, $A[i] = A[j] = x$, So 
$A[p..j]$ is less than or equal to every element of $A[j+1\dots r]$. When $i = 
j+1$, $A[j] \le x$, $d$'s conclusion also holds. There no there relations 
between $i$ and $j$. So $d$ is proved.
 
\textbf{e:}
\begin{codebox}
\Procname{$\proc{Quicksort}(A,p,r)$}
\li \If $p < r$
\li \Then $q \gets \proc{Hoare-Partition}(A,p,r)$
\li $\proc{Quicksort}(A,p,q)$
\li $\proc{Quicksort}(A,q+1,r)$
\End
\end{codebox}

\subsection*{7.2} \textbf{a:} Since all element values are equal, 
$\proc{Randomized-Partition}$ always return the same element. The behaviour of 
the randomized quicksort is the same as the ordinary quicksort. So the running 
time is $\Uptheta(n^2)$.

\textbf{b:}
\begin{codebox}
\Procname{$\proc{Partition}'(A,p,r)$}
\li $x \gets A[r]$
\li $i \gets p-1$
\li $k \gets p-1$
\li \For $j=p$ \To $r-1$
\li \Do \If $A[j] < x$
\li \Then $i \gets i+1$
\li $k \gets k+1$
\li $temp \gets A[i]$
\li $A[i] \gets A[j]$
\li $A[j] \gets A[k]$
\li $A[k] \gets temp$
\li \ElseIf $A[j] \isequal x$
\li \Then $k \gets k+1$
\li exchange $A[k]$ with $A[j]$
\End
\End
\li exchange $A[k+1]$ with $A[r]$
\li \Return $(i+1,k+1)$
\end{codebox}
We can do a loop invariant analysis for this subroutine. It is easy. And it is 
easy to see that this subroutine takes $\Uptheta(r-p)$ time.

\textbf{c:}
\begin{codebox}
\Procname{$\proc{Randomized-Partition}'(A,p,r)$}
\li $i \gets \proc{Random}(p,r)$
\li exchange $A[r]$ with $A[i]$
\li \Return $\proc{Partition}'(A,p,r)$
\end{codebox}

\begin{codebox}
\Procname{$\proc{Quicksort}'(A,p,r)$}
\li \If $p < r$
\li \Then $(q,t) \gets \proc{Randomized-Partition}'(A,p,r)$
\li $\proc{Quicksort}'(A,p,q-1)$
\li $\proc{Quicksort}'(A,t+1,r)$
\End
\end{codebox}

\textbf{d:} We need to rename the elements of the array $A$ as 
$z_1,z_2,\dots\dots,z_n$ ($z_a \le z_b, a < b$). This is the only adjustment 
needed.

\subsection*{7.3} \textbf{a:} E[$X_i$] = $1/n$. \\
\textbf{b:} First we argue that $T(n) = T(q-1) + T(n-q) + \Uptheta(n)$. $T(n)$ 
has two costs. The first cost is to partition the input array. This cost is 
$\Uptheta(n)$. The other part is to sort the partitioned two parts. The size of 
one part is $q-1$. The size of the other part is $n-q$. So the cost to sort 
these two parts are $T(q-1) + T(n-q)$. So the total cost 
$T(q-1)+T(n-q)+\Uptheta(n)$. Since the probability of choosing a element as 
pivort is $1/n$. We have:
\begin{align*}
E[T(n)] &= E\Bigg[\sum_{q=1}^{n}\frac{1}{n}(T(q-1)+T(n-q)+\Uptheta(n)) \Bigg] \\
&= E\Bigg[\sum_{q=1}^{n}X_q(T(q-1)+T(n-q)+\Uptheta(n)) \Bigg]
\end{align*}

\textbf{c:}
\begin{align*}
E[T(n)] &= E\Bigg[\sum_{q=1}^{n}X_q(T(q-1)+T(n-q)+\Uptheta(n)) \Bigg] \\
&= E\Bigg[\sum_{q=1}^{n}\frac{1}{n}(T(q-1)+T(n-q)+\Uptheta(n)) \Bigg] \\
&= \frac{1}{n}E\Bigg[\sum_{q=1}^{n}(T(q-1)+T(n-q)) \Bigg] + \Uptheta(n) \\
&= \frac{1}{n}E\Bigg[\sum_{q=1}^{n}T(q-1)+\sum_{q=1}^{n}T(n-q) \Bigg] + \Uptheta(n) \\
&= \frac{2}{n}E\Bigg[\sum_{q=1}^{n}T(q-1) \Bigg] + \Uptheta(n) \\
&= \frac{2}{n}E\Bigg[\sum_{q=2}^{n-1}T(q)+ T(0) + T(1)\Bigg] + \Uptheta(n) \\
&= \frac{2}{n}E\Bigg[\sum_{q=2}^{n-1}T(q)\Bigg] + T(0) + T(1) + \Uptheta(n) \\
&= \frac{2}{n}E\Bigg[\sum_{q=2}^{n-1}T(q)\Bigg] + \Uptheta(n) \\
&= \frac{2}{n}\sum_{q=2}^{n-1}E[T(q)] + \Uptheta(n)
\end{align*}

\textbf{d:}
\begin{align*}
\sum_{k=2}^{n-1}k\lg k & = \sum_{k=2}^{\lceil\frac{n}{2}\rceil -1}k\lg k + 
\sum_{k=\lceil\frac{n}{2}\rceil}^{n-1}k\lg k \\
& \le \lg(\lceil\frac{n}{2}\rceil-1)\sum_{k=2}^{\lceil\frac{n}{2}\rceil-1} k + 
\lg(n-1)\sum_{k=\lceil\frac{n}{2}\rceil}^{n-1}k \\
& = \lg(\lceil\frac{n}{2}\rceil-1) 
\frac{1}{2}(\lceil\frac{n}{2}\rceil-1+2)(\lceil\frac{n}{2}\rceil-2) + 
\lg(n-1)\frac{1}{2}(n-1+\lceil\frac{n}{2}\rceil)(n-\lceil\frac{n}{2}\rceil) \\
& \le \lg(\frac{n}{2})
\frac{1}{2}(\lceil\frac{n}{2}\rceil+1)(\lceil\frac{n}{2}\rceil-2) + 
\lg(n)\frac{1}{2}(n-1+\lceil\frac{n}{2}\rceil)(n-\lceil\frac{n}{2}\rceil) \\
& = \frac{1}{2}(\lg n-1)(\lceil\frac{n}{2}\rceil^2-\lceil\frac{n}{2}\rceil - 
2) + \frac{1}{2}\lg(n)(n^2-\lceil\frac{n}{2}\rceil^2-n+\lceil\frac{n}{2}\rceil) 
   \\
& \le \frac{1}{2}(\lg n-1)(\lceil\frac{n}{2}\rceil^2) + \frac{1}{2}\lg(n)(n^2-\lceil\frac{n}{2}\rceil^2) \\
& \le \frac{1}{2}(\lg n-1)(\lceil\frac{n}{2}\rceil^2) + 
\frac{1}{2}\lg(n)(n^2-\lceil\frac{n}{2}\rceil^2) \\
& = -\frac{1}{2}\lceil\frac{n}{2}\rceil^2 + \frac{1}{2}n^2\lg n \\
& \le \frac{1}{2}n^2\lg n - \frac{1}{8}n^2
\end{align*}

We have another way to prove the inequality. Since $k\lg k$ is a monotonically 
increasing function, we have from (A.11):
\begin{align*}
\sum_{k=2}^{n-1}k\lg k & \le \int_{2}^{n}x\lg x \mathrm{d}x \\
& = \Big[\frac{1}{2}x^2\lg x - \frac{x^2}{4\ln 2}\Big]_2^{n} \\
& = \frac{1}{2}n^2\lg n - \frac{1}{4\ln2}n^2-(2-\frac{1}{\ln2}) \\
& \le \frac{1}{2}n^2\lg n - \frac{1}{4\ln2}n^2 \\
& \le \frac{1}{2}n^2\lg n - \frac{1}{8}n^2 \\
\end{align*}

\textbf{e:} First to prove that $E[T(n)] = O(n\lg n)$. Assume $E[T(n)] \le an\lg 
n$.
\begin{align*}
\text{E}[T(n)] & = \frac{2}{n}\sum_{q=2}^{n-1}E[T(q)] + \Uptheta(n) \\
& \le \frac{2a}{n}\sum_{q=2}^{n-1}q\lg q + \Uptheta(n) \\
& \le \frac{2a}{n}(\frac{1}{2}n^2\lg n - \frac{1}{8}n^2) + \Uptheta(n) \\
& = an\lg n - \frac{1}{4}an + \Uptheta(n) \\
& \le an\lg n
\end{align*}
since we can pick a large enough. So E$[T(n)] = O(n\lg n)$.From Exercise 7.4-2, 
we know quicksort's best-case running time is $\Omega(n\lg n)$. So E$[T(n)] = 
\Uptheta(n \lg n)$.

\subsection*{7.4}
\textbf{a:}
$$p_i = \frac{(i-1)(n-i)}{\binom{n}{3}}, i = 2,3,\dots, n-1$$

\textbf{b:} New probability 
$\frac{(\lfloor(n+1)/2\rfloor-1)(n-\lfloor(n+1)/2\rfloor)}{\binom{n}{3}}$. Old 
probability $1/n$. So the increasement is 
$\frac{(\lfloor(n+1)/2\rfloor-1)(n-\lfloor(n+1)/2\rfloor)}{\binom{n}{3}}-\frac{1}{n}$.

\begin{align*}
\frac{P_{new}}{P_{old}} & = 
\frac{\frac{(\lfloor(n+1)/2\rfloor-1)(n-\lfloor(n+1)/2\rfloor)}{\binom{n}{3}}}{\frac{1}{n}}\\
& = 6\frac{(\lfloor(n+1)/2\rfloor-1)(n-\lfloor(n+1)/2\rfloor)}{(n-1)(n-2)} \\
& \ge 6\frac{(\frac{n+1}{2}-1)(n-\frac{n+1-1}{2})}{(n-1)(n-2)} \\
& = \frac{3}{2}(\frac{n}{n-2})
\end{align*}
When $n \to \infty$, the ratio is $3/2$.

\textbf{c:}
For the ordinary implementation, the probability is $\frac{\frac{n}{3}+1}{n}$.  
Its approximation is $1/3$. Let $f(i) = (i-1)(n-i) = -i^2+(n+1)i-n$. $f'(i) = 
n+1-2i$. When $i=(n+1)/2$, $f'(i)$ is zero. So $f(i)$ is monotonically 
increasing in $[-\infty, \frac{n+1}{2}]$. And $f(i)$ is monotonically decreased 
in $[\frac{n+1}{2},\infty]$.For the median-of-3 implementation.  
\begin{align*}
p & = 1 - \sum_{i=2}^{\frac{n}{3}}\frac{f(i)}{\binom{n}{3}} - 
\sum_{i=\frac{2n}{3} + 1}^{n-1}\frac{f(i)}{\binom{n}{3}} \\
& \ge 1 - \frac{1}{\binom{n}{3}}(\int_2^{\frac{n}{3}+1}f(x)\mathrm{d}x + 
\int_{\frac{2n}{3}}^{n-1}f(x)\mathrm{d}x) \\
& = 1 - 
\frac{6}{n(n-1)(n-2)}(\Big[-\frac{i^3}{3}+\frac{n+1}{2}i^2-ni\Big]_2^{\frac{n}{3}+1}+
\Big[-\frac{i^3}{3}+\frac{n+1}{2}i^2-ni\Big]_{\frac{2n}{3}}^{n-1}) \\
& \approx 
1 - 
  \frac{6}{n(n-1)(n-2)}(\Big[-\frac{i^3}{3}+\frac{n+1}{2}i^2-ni\Big]_2^{\frac{n}{3}}+
\Big[-\frac{i^3}{3}+\frac{n+1}{2}i^2-ni\Big]_{\frac{2n}{3}}^{n}) \\
& \approx 1 - \frac{6}{n^3} \frac{7}{81}n^3, (only\: consider\: n^3\: terms) \\
& = \frac{13}{27}
\end{align*}
So the increasement is approximately $\frac{13}{27}-\frac{1}{3}=\frac{4}{27}$.

\textbf{d:} Even though we always pick the median of the input array as the pivot
during the recursion call, the best running time is still $\Omega(n \lg n)$. So 
the median-of-3 version of quicksort cannot affect the asymptotic running time, 
it only affect the constant factor.

\subsection*{7.6}
\textbf{a:}
\begin{codebox}
\Procname{$\proc{Fuzzy-Sort}(A,p,r)$}
\li \If $p < r$
\li \Then $(q,t) \gets \proc{Randomized-Partition}(A,p,r)$
\li $\proc{Fuzzy-sort}(A,p,q-1)$
\li $\proc{Fuzzy-sort}(A,q+1,r)$
\end{codebox}

\begin{codebox}
\Procname{$\proc{Randomized-Partition}(A,p,r)$}
\li $i \gets \proc{Random}(p,r)$
\li exchange $A[r]$ with $A[i]$
\li \Return $\proc{Partition}(A,p,r)$
\end{codebox}

\begin{codebox}
\Procname{$\proc{Compare}(x,y)$}
\li \If $x.b < y.a$
\li \Then \Return $-1$
\li \ElseIf $x.a > y.b$
\li \Then \Return $1$
\li \Else \Return $0$
\end{codebox}


\begin{codebox}
\Procname{$\proc{Partition}(A,p,r)$}
\li $x \gets A[r]$
\li $i \gets p-1$
\li $k \gets p-1$
\li \For $j=p$ \To $r-1$
\li \Do $comp = \proc{Compare}(A[j], x)$
\li \If $comp < 0$
\li \Then $i \gets i+1$
\li $k \gets k+1$
\li $temp \gets A[i]$
\li $A[i] \gets A[j]$
\li $A[j] \gets A[k]$
\li $A[k] \gets temp$
\li \ElseIf $comp \isequal 0$
\li \Then $k \gets k+1$
\li exchange $A[k]$ with $A[j]$
\End
\End
\li exchange $A[k+1]$ with $A[r]$
\li \Return $(i+1,k+1)$
\end{codebox}

\textbf{b:} The running time of subroutine $\proc{Partition}$ is clearly 
$\Theta(n)$.  Hence, the total running time of $\proc{Fuzzy-Sort}$ is $\Theta(n 
\lg n)$.  As more and more intervals overlap, the left group and right group 
will have fewer and fewer intervals. So the running time will speed up. If all 
intervals overlap, then the running time is $\Theta(n)$.

\chapter{Sorting in Linear Time}
\subsection*{8.1-1} $n-1$.
\subsection*{8.1-2} $\lg(n!) = \sum_{i=1}^{n}\lg n$. Since $\sum_{i=1}^{n}\lg i \le 
n\lg n$. So we have $\lg(n!) = O(n\lg n)$.
\begin{align}
\sum_{i=1}^{n}\lg i & \ge \sum_{i=\lceil\frac{n}{2}\rceil}^{n}\lg i \\
& \ge \sum_{i=\lceil\frac{n}{2}\rceil}^{n}\lg(\lceil\frac{n}{2}\rceil) \\
& = (n-\lceil\frac{n}{2}\rceil+1)\lg(\lceil\frac{n}{2}\rceil) \\
& = (\lfloor\frac{n}{2}\rfloor+1)\lg(\lceil\frac{n}{2}\rceil) \\
& \ge \frac{n}{2}\lg(\frac{n}{2}) \\
& \ge \frac{1}{2}n\lg\sqrt n \\
& = \frac{1}{4}n\lg n
\end{align}
(5.6) holds when $n \ge 4$. So we have $\lg(n!) = \Omega(n\lg n)$. Now $\lg(n!) 
= \Uptheta(n\lg n)$.

\subsection*{8.1-4}
For a $k$ element subsequence, the comparison number $h_i \ge \lg(k!)$ ($1 \le i 
\le n/k$).  So for $n/k$ such subsequences, the comparison number 
$\sum_{i=1}^{n/k}h_i \ge \frac{n}{k}lg(k!) \ge \frac{n}{k}(k\lg k - k\lg e) = 
n\lg k - n\lg e = \Omega(n\lg k)$.

\subsection*{8.2-4}
\begin{codebox}
\Procname{$\proc{Preprocess}(A, C, k)$}
\li let $C[-1\dots k]$ be a new array
\li \For $i \gets -1$ \To $k$
\li \Do $C[i] \gets 0$
\End
\li \For $j \gets 1$ \To $A.length$
\li \Do $C[A[j]] \gets C[A[j]]+1$
\End
\li \Comment $C[i]$ now contains the number of elements equal to $i$.
\li \For $i \gets 1$ \To $k$
\li \Do $C[i] \gets C[i] + C[i-1]$
\End
\li \Comment $C[i]$ now contains the number of elements less than or equal to 
$i$.
\end{codebox}

\begin{codebox}
\Procname{$\proc{Query}(C, a, b)$}
\li $a \gets \proc{Max}(a, 0)$
\li $b \gets \proc{Min}(b, k)$
\li \Return $C[b] - C[a-1]$
\end{codebox}

\subsection*{8.3-5}
In the first card-sorting algorithm in this subsection*, exactly how many sorting passes are needed to sort d-digit decimal numbers in the worst cas? How many piles of cards would an operator need to keep track of in the worst case?
solution:
There are $$1 + 10 + 10^2 + \cdots +10^{d-1} = \frac{10^d - 1}{10 - 1} = \frac{10^d - 1}{9}$$ passes needed to sort all numbers.
Each time we recursively sort numbers in a bin, we must keep track of other 9 
bins. And we have to recursively sort the numbers for each digit. Hence we have 
to keep track of $9(d-1)$ piles of cards in the worst case.

\subsection*{8.4-3} This is a 2 Bernoulli trails with $p=1/2$.
\begin{align*}
E[X]_2 & = (np)^2 \\
& = (2 \cdot \frac{1}{2})^2 \\
&= 1^2 \\
& = 1
\end{align*}
Define $X_i$ as the event that head is shown up in the $i$th filp.
\begin{align*}
E[X_i] & = 1 \cdot \frac{1}{2} + 0 \cdot \frac{1}{2} \\
& = \frac{1}{2}
\end{align*}
\begin{align*}
E[X_i^2] & = 1^2 \cdot \frac{1}{2} + 0^2 \cdot \frac{1}{2} \\
& = \frac{1}{2}
\end{align*}
\begin{align*}
E[X^2] & = E[(X_1+X_2)^2] \\
& = E[X_1^2+X_2^2+2X_1 X_2] \\
& = E[X_1^2] + E[X_2^2] + 2 E[X_1] E[X_2] \\
& = \frac{1}{2} + \frac{1}{2} + 2 \cdot \frac{1}{2} \cdot \frac{1}{2} \\
& = \frac{3}{2}
\end{align*}

\subsection*{8.4-4}
Note that the expected number of points residing in an area is proportional to 
the size of the area. To satisfy the assumption of BUCKET-SORT, we have to make 
the interval $(0, 1]$ split in a proper way. Consider a distance $d$ between $a$ 
and $b$, where $0 \leq a < d < b \leq 1$. The area is $\pi(b^2 - a^2)$. Dividing 
$(0, 1]$ into n equal-sized intervals does not help a lot. It does not mean we 
divide a unit circle into n areas evenly. To achieve our goal, we divide $(0, 
1]$ into $<0, \sqrt{1/n}, \sqrt{2/n}, \ldots, \sqrt{n-1/n}, 1>$ intervals 
instead. In the context of such dividing, we can see that the $\pi(1/n) = 
\pi((\sqrt{2/n})^2 - (\sqrt{1/n})^2) = \cdots = \pi(1 - (\sqrt{n-1/n})^2)$. 
Because of the equal size of area, we'll expect that each bucket contain roughly 
the same points.

\subsection*{8.4-5}
\begin{codebox}
\Procname{$\proc{Bucket-Sort}(A)$}
\li let $B[0\dots n-1]$ a new array
\li $n \gets A.length$
\li \For $i \gets 0$ \To $n-1$
\li \Do make $B[i]$ a empty list
\End
\li \For $i\gets 1$ \To $n$
\li \Do Insert $A[i]$ into list $B[\lfloor n \proc{P}(A[i] )\rfloor]$
\End
\li \For $i \gets 0$ \To $n-1$
\li \Do sort list $B[i]$ with insert sort
\End
\li concatenate the lists $B[0], B[1], \dots, B[n-1]$ together in order
\end{codebox}

$P(x)$ is defined in a manner of cumulative distribution function. By 
probability integral transform, we know that $P(X_i)$ has a uniform 
distribution, where $i = 1, 2, \ldots, n$. As a result, we can sort these 
transformed numbers by BUCKET-SORT in linear time.

\subsection*{8.2}
\textbf{a:} counting sort \\
\textbf{b:} Use a partition \\
\textbf{c:} insert sort \\
\textbf{d:} Counting sort. Choose $r=1$. The running time 
$$\Uptheta((b/r)(n+2^r) = \Uptheta(b(n+2)) = \Uptheta(bn)$$
\textbf{e:}

\begin{codebox}
\Procname{$\proc{Counting-Sort}(A, k)$}
\li let $C[0\dots k]$ be a new array
\li \For $i \gets 0$ \To $k$
\li \Do $C[i] \gets 0$
\End
\li \For $j \gets 1$ \To $A.length$
\li \Do $C[A[j]] \gets C[A[j]] + 1$
\End
\li $index \gets 1$
\li \For $i \gets 0$ \To $k$
\li \Do \For $j \gets 1$ \To $C[i]$
\li \Do $A[index] = i$
\li $index \gets index + 1$
\End
\End
\End
\end{codebox}
This algorithm is not stable.

\subsection*{8.3}
\textbf{a:} First to sort the integers using the $i$th digit (Count sort can be 
used). Then put all the integers which only has $i$ digits before other 
integers. Invoke this algorithm recursively on the latter group with $i$ 
increased to $i+1$. The initial value of $i$ is $1$ (the least significant 
digit).  Terminate this algorithm when the latter group is empty. Let $m_i$ be 
the number of integers which have $i$ digits and $l$ the maximum length of all 
the integers.  $T=\sum_{i=1}^{l}O(m_i+k) = O(\sum_{i=1}^l (m_i+k)) O(n+lk)$.  
Since $l \le n$. So $T=O(n)$.

\textbf{b:} When a string $x$ and a string $y$ are compared, if the first letter 
of a string $x$ is lexicographically greater than the first letter of a string 
$y$, then $x$ is lexicographically greater than $y$. We need a algorithm which 
sorts the strings on the $i$th letter, and gather strings with the same $i$th 
letter into groups (Counting sort can be used). The strings which don't have the 
$i$th letter are taken out and put first. Then, the sorting is recursive within 
each group with $i$ increased to $i+1$. We invokes this algorithm with $i = 1$.  
Finally, all strings will be sorted.
We analyze the running time of the sorting process. The $m_i$ denotes the number 
of strings of i characters. $l$ is the maximum length of all the input strings.  
And we have $\sum_{i = 1}^l i \cdot m_i = n$. If there are $t_i$ strings with i 
characters and the total characters over all strings is $n$, then the total cost 
to call counting sort is $\sum_{i=1}^l O(t_i) = O(\sum_{i=1}^l t_i) = O(n)$.  
Thus, the overall running time is $O(n)$.  

\subsection*{8.5}
\textbf{a:} Ordinaray sorting. \\
\textbf{b:} \\
\begin{tabular}{|c|c|c|c|c|c|c|c|c|c|}
\hline
6 & 1 & 7 & 2 & 8 & 3 & 9 & 4 & 10 & 5 \\ \hline
\end{tabular} \\
\textbf{c:} First to prove the $if$ part. For $A[i] \le A[i+k]$, for all 
$i=1,2,\dots, n-k$. Add $A[i+1] + \dots + A[i+k-1]$ to the two sides of the 
equation. We have $A[i] + A[i+1] + \dots + A[i+k-1] \le A[i+1] + \dots + 
A[i+k-1] + A[i+k]$. Divide the tow sides of the equation with $k$. We have 
$\frac{\sum_{j=i}^{i+k-1}A[j]}{k} \le \frac{\sum_{j=i+1}^{i+k}A[j]}{k}$.

Second to prove the $only-if$ part. Mutiply $\frac{\sum_{j=i}^{i+k-1}A[j]}{k} 
\le \frac{\sum_{j=i+1}^{i+k}A[j]}{k}$ with $k$ and subtract 
$A[i+1]+\dots+A[i+k-1]$. We have $A[i] \le A[i+k]$.

\textbf{d:} We can divide the input array array to $k$ groups. $i$ ($0 \le i \le 
k-1$) groups contains elements $A[i], A[i+k], \dots$. Heapsort every groups.  
From c:, we know that the input arrary is k-sorted. The cost to sort 1 group is 
$O((n/k)\lg (n/k))$. So the total cost is $k \cdot O((n/k)\lg(n/k)) = 
O(n\lg(n/k))$.

\textbf{e:} Without loss of generality, we assume that $n = km$. That is, an 
n-element array can be divide into m group of length k. The first element of 
each group can form a sorted list. The second element of each group form a 
sorted list as well and so on. By Exercise 6.5-9, we can merge these $(n/k)$ 
sorted list into a sorted list in $O(n\lg(n/k))$ time.

\textbf{f:} The problem of k-sorting an input array is equivalent to sort $k$ 
groups of the input array. Sorting a group requires $\Omega((n/k)\lg(n/k))$. So 
the total cost is $k \Omega((n/k)\lg(n/k)) = \Omega(n\ln(n/k))$.

\subsection*{8-6}
\textbf{a:} $\binom{2n}{n}$.

\textbf{b:} Let the height of the decision tree be h. The decision tree contains 
at least $C^{2n}_n$ leaves.  By Stirling's formula, we know that $$n! = \sqrt{2\pi 
n}(\frac{n}{e})^ne^{\alpha_n}$$, where $$\frac{1}{12n+1} < \alpha_n < 
\frac{1}{12n}$$. Then,
\begin{align*}
2^h \geq C^{2n}_n &= \frac{(2n)!}{n!n!} \\
& \geq \frac{\sqrt{4\pi n}(\frac{2n}{e})^{2n}e^{\frac{1}{24n+1}}}{[\sqrt{2\pi 
n}(\frac{n}{e})^ne^{\frac{1}{12n}}]^2} \\
& \geq \frac{\sqrt{4\pi n}(\frac{2n}{e})^{2n}e^{\frac{1}{25n}}}{[\sqrt{2\pi 
n}(\frac{n}{e})^ne^{\frac{1}{12n}}]^2} \\
& \geq \frac{(\frac{2n}{e})^{2n}}{2\pi n(\frac{n}{e})^{2n}e^{\frac{19}{150n}}}  
\\
& \geq \frac{2^{2n}}{2\pi n\cdot e^{\frac{19}{150n}}}
\end{align*}

$$\Rightarrow h \geq 2n - \lg 2\pi n - \frac{19}{150n}\lg e = 2n - o(n)$$, where
$$\lim_{n \rightarrow \infty}\frac{\lg 2\pi n + \frac{19}{150n}\lg e}{n} = 0 
\Rightarrow \lg 2\pi n + \frac{19}{150n}\lg e \in o(n)$$.

c. Let the two sorted array be A and B, and $a_1, a_2, \ldots, a_n$ and $b_1, 
b_2, \ldots, b_n$ are elements of A and B respectively.Suppose that $a_i$ and 
$b_j$ are adjacent in final sorted list.
$\cdots, a_{i-1}, a_i, a_{i+1}, \cdots$
$\cdots, b_{j-1}, b_j, b_{j+1}, \cdots$
If $b_j$ compares to $a_k$ where $k = 1, \ldots, i-1$, the result must be $a_k 
\leq b_j$ but we are not able to determine the whether $b_j$ is greater than 
$a_i$ or not. Similarly, if $b_j$ compares to $a_K$ where $K = i+1, \ldots, n$, 
the result must be $a_K \ge b_j$ we cannot know that if $b_j$ is greater than 
$a_i$ or not, either. Therefore, the only way we figure out the order of $b_j$ 
and $a_i$ in the final sorted list is comparing $b_j$ to $a_i$ directly.

d. We consider a worst-case to determine the lower bound. Let $a_1, a_2, \ldots, 
a_{2n}$ be a sorted list, and $A = \langle a_1, a_3, \ldots, a_{2n-1}\rangle$, 
$B = \langle a_2, a_4, \ldots, a_{2n}\rangle$. By problem c, we know that $a_i$ 
must compare to $a_{i+1}$ in the merging process. So, there must be at least 
$2n-1$ comparisons.

\subsection*{8-7}
\textbf{a:} If $A[p] \isequal A[q]$, $A[p]$ is in its right position which 
contradicts with that $A[p]$ is in a wrong position. So we have $A[p] \neq 
A[q]$.  If there are $l$ elements equal to $A[q]$, one of them must be in a 
wrong position. Assmue it is $A[q'] = A[q]$. Since $A[p]$ is the samllest of the 
values which are put into wrong positions, $A[q'] >= A[p]$.
$$A[p]\neq A[q], A[q] \ge A[p] \Rightarrow A[q] > A[p]$$

So we have $A[q] > A[p]$.  $B[p] = 0$ and $B[q] = 1$ follows. 

Let $k$ be the first position where a wrong element is put. And since $A[p]$ is 
the smallest of values which are put into wrong positions, $A[p]$ should have 
gone into $k$. Since $A[k]$ is in a wrong position, $A[k] \neq A[p]$. And $A[p]$ 
is smallest, we have $A[k] > A[p]$. So we have $B[k] = 1$ and $k < p$.

\textbf{b:} Whenever the algorithm compares a pair of 1's or 0's in $B$, it not 
important whether it exchanges the values or not, so we may simply assume that 
it does the same as on $A$. On the other hand, whenever the algorithm compares 
some values $B[i]=0$ and $B[j]=1$, this means that $A[i] \le A[p] < A[j]$.  
Therefore, in this case the oblivious compare-exchange algorithm will do the 
same on $A$ and $B$. So after the sorting, $B$'s definition still holds. And 
from the discussion in a:, we know that X fails to correctly sort $B$.

\textbf{c:} Difference of sorting methods used in odd steps does not have any 
effect of the columnsort. So we can assume that oblivious compare-exchange 
insert-sort is used. For the event steps, they only involves compare-exchange 
operations which are based on elements' indices. So columnsort can be treated as 
an oblivious compare-exchange algorithm.

\textbf{d:} After step 1, every column is sorted. Every column is a string 
$\underbrace{0\dots0}_\text{$l$ 0s}\underbrace{1\dots1}_\text{$m$ 1s}$. After 
step 2, every column is transposed into at least $2s$ rows. Since the column is 
sorted before the transposition, there is at most 1 dirty row among these rows.  
Since there are $s$ columns which are transposed, the total number of dirty rows 
is at most $s$. After step 3, all the clean 0-rows are at the top and all the 
clean 1-rows are at the bottom. And there are at most $s$ dirty rows between 
them.

\textbf{e:} After step 4's inverse permutation, all clean 0-rows at the top are 
combined into 0-columns on the left side since $s$ is a divisor of $r$. All 
clean 1-rows at the bottom are combined into 1-columns on the right side. The at 
most $s$ dirty rows are transposed into a dirty area which contains at most 
$s^2$ elements.

\textbf{f:} If there is no dirty column, a fully-sorted output is already 
produced.  If there is only $1$ dirty column, step 5 will produce a fully-sorted 
output.  The following steps does not have any effects on the final output since 
the sorting on sorted columns does nothing and a permutation and an inverse 
permutation means no array element movements.

The other case is that there are $2$ dirty columns.  And dirty column $j$'s top 
half part is all 0s. Dirty column $j+1$'s bottom half part is all 1s. After step 
6, column $j+1$ are composed of the bottom part of column $j$ and the top part 
of column $j+1$. And all the other areas are clean. After step 7, the top part 
of column $j+1$ are clean-0s. The bottom part is $\underbrace{0\dots0}_\text{$a$ 
0s}\underbrace{1\dots1}_\text{$b$ 1s}$. After step 8, column $j$ are 0-column.  
And column $j+1$ are $\underbrace{0\dots0}_\text{$a$ 
0s}\underbrace{1\dots1}_\text{$b+s^2$
1s}$. All the columns on the left side are 0-columns and all the columns on the 
right side are 1-columns. So the output is fully-sorted.

\textbf{g:} Since $s$ does not divide $r$, every column is transposed into some 
0-rows, 1 dirty row, some 1-rows and one sequence of 1 whose lenght is less than 
$s$. The transposed result is that columns except the first one are shifted by 
some number less than $s$. So for the $s$ dirty rows mentioned above, there are 
still at most $s$ dirty rows. And there are $s-1$ rows which are contribute by 
the ending part of $i$ column and $i+1$ column. These rows may be dirty. So 
there are at most $2s-1$ dirty rows. At the top, there are 0-rows. At the 
bottom, there are 1-rows. After step 4, there will a at most $s(2s-1)$ dirty 
area. So if $r \ge 2\cdot s(2s-1)$, columnsort still works.

\textbf{h:} If $s$ does not divide $r$, we can append $r \mod s$ $\infty$'s to 
every column. The new column lengh is $r'=r+(r\mod s)$. Since these appended 
$\infty$s will be in the last $s (r \mod s)$ positions of the fully-sorted 
output, they are just be discarded.  The left elements are fully-sorted. \\
Another solution is to append $s(r \mod s)$ to the end of the input array. The 
discussion is similar.


\chapter{Medians and Order Statistics}

\subsection*{9.1-2}
The following text is 
\href{http://rightwayman.blogspot.com/2010/12/algo-exercise-91.html}{rightwayman's 
solution}. \\
Initially, there are n potential items of maximum candidates and minimum 
candidates. If $a_i$ and $a_j$ are simultaneous in the candidate set of maximum 
and minimum respectively, after $a_i$ compares to $a_j$ the cardinality of 
candidate set of maximum and minimum will decrease by 1. If $a_i$ and $a_j$ are 
in the candidate set of maximum or minimum, after $a_i$ compares to $a_j$ the 
cardinality of the set will decrease by 1.
(1) when n is even:
After $n/2$ comparisons, the candidate set of maximum and minimum will remain 
$n/2$ candidates. Next, it requires $n/2 - 1$ comparisons to determine the 
maximum and another $n/2 - 1$ comparisons to determine the minimum. Total 
comparisons needed is $3n/2 - 2 = \lceil 3n/2 \rceil -2$.
(2) when n is odd:
After $(n-1)/2$ comparisons, the candidate set of maximum and minimum will 
remain $(n+1)/2$ candidates. Next, it requires at most $(n+1)/2 - 1$ comparisons 
to determine the maximum and another at most $(n+1)/2 - 1$ comparisons to 
determine the minimum. Total comparisons needed is $3n/2 - 3/2 = \lceil 3n/2 
\rceil -2$.

\subsection*{9.2-1} There are two possible cases of recursive calls to 0-length 
array. One case is when $q = p$. In this case, $k = 1$. If $i = 1$, line-6 is 
executed.  If $i > 1$, line-9 is executed. Since $i < k$ is imposssible, line-8 
0-length recursive call will not be executed. The discussion for the case that 
$q = r$ is similar.

\subsection*{9.2-2} Occurrence of $X_k$ makes the time it takes to do the recursive 
call neither more nor less.

\subsection*{9.2-3}
\begin{codebox}
\Procname{$\proc{Randomized-Select}(A,p,r,i)$}
\li \While $p < r$
\li \Do $q \gets Randomized-Partition(A,p,r)$
\li $k \gets q-p+1$
\li \If $i \isequal k$
\li \Then \Return $A[q]$
\li \ElseIf $i < k$
\li \Then $r \gets q-1$
\li \Else $p \gets q+1$
\li $i \gets i-k$
\End
\End
\li \Return $A[p]$
\end{codebox}

\subsection*{9.2-4} 9, 8, 7, 6, 5, 4, 3, 2, 1, 0

\subsection*{9.3-2}
$$n \ge 140 \Rightarrow \frac{3n}{10} -6 \ge \frac{n}{4}+1 \Rightarrow 
\frac{3n}{10}-6 \ge \lceil\frac{n}{4}\rceil$$
Since there are at least $\frac{3n}{10}-6$ elements which are greater that the 
median-of-medians $x$ and at least $\frac{3n}{10}-6$ elements which are less 
than $x$. So we have the solution.

\subsection*{9.3-3}
Before PARTITION is performed, we call $\proc{SELECT}$ to pick up the median of 
the input array. The median is used for a pivot to split the array into two 
subarrays.  Because of the median, it guarantees the left hand side and right 
hand side must be of the same length. Let $T(n)$ be the running time needed by 
the modified quicksort to perform on input array of n elements. We can derive 
the recurrence as follows: $T(n) \leq 2T(n/2) + \Theta(n)$. By master theorem, 
$T(n) = \Theta(n\lg n)$.

\begin{codebox}
\Procname{$\proc{Modified-Partition}(A, p, r)$}
\li $n \gets n-p+1$
\li $k \gets \proc{Select}(A,p,r,\lfloor(n+1)/2\rfloor)$
\li exchange $A[r]$ with $A[k]$
\li \Return $\proc{Partition}(A,p,r)$
\end{codebox}

\subsection*{9.3-4 $\star$}
Suppose that the algorithm uses $m$ comparisons to find the ith smallest element 
$x$ in a set of $n$ elements. If we trace these $m$ comparisons in the log, we 
can easily find the $i-1$ smaller elements by transitivity. If $x > a$ and $a > 
b$, then $x > b$ can be deduced without actually performing a comparison between 
$x$ and $b$.  Similarly, the $n-i$ larger elements can be found, too. Is it 
possible there exists a number, say $p$, we cannot decide whether $x$ is greater 
than $p$ or not by the comparison log? That's impossible! Otherwise, the 
algorithm does not work correctly.

\subsection*{9.3-6}
The k quantiles of an n-element array A are
$A[\lceil 1\cdot n/k \rceil], A[\lceil 2\cdot n/k \rceil],\cdots,A[\lceil 
(k-1)\cdot n/k \rceil]$.
The following algorithm finds kth quantiles of an array A, and these (k-1) 
elements are put  just like the positions they reside if entire range is sorted.

\begin{codebox}
\Procname{$\proc{Quantile-Base}(A,p,r,Q,k)$}
\li \If $k \le 0$
\li \Then \Return
\li \ElseIf $k \isequal 1$
\li \Then $x \gets \proc{Select}(A,p,r,Q[1])$
\li $\proc{Partition}(A,p,r,x)$
\li \Else $i \gets \lfloor(k+1)/2\rfloor$
\li $x \gets \proc{Select}(A,p,r,Q[i])$
\li $q \gets \proc{Partition}(A,p,r,x)$
\li $\proc{Quantile-Base}(A,p,q-1,Q[1\dots (i-1)], i-1)$
\li \For $j \gets (i+1)$ \To $Q.length$
\li \Do $Q[j] = Q[j] - Q[i]$
\End
\li $\proc{Quantile-Base}(A, q+1, r, Q[(i+1)\dots Q.length], k-i)$
\End
\end{codebox}

\begin{codebox}
\Procname{$\proc{Quantile}(A,p,r,k)$}
\li \If $k \le 1$
\li \Then \Return
\li \Else $n \gets r-p+1$
\li $\proc{Quantile-Base}(A,p,r,[\lceil 1\cdot n/k\rceil, \lceil 2\cdot 
n/k\rceil,\cdots, \lceil (k-1)\cdot n/k\rceil], k-1)$
\end{codebox}

$$
\left\{
\begin{array}{ll}
T(n, k) = T(\lceil\lfloor \frac{k+1}{2} \rfloor \frac{n}{k}\rceil-1, 
\lfloor\frac{k+1}{2}\rfloor-1) + T(n-\lceil\lfloor \frac{k+1}{2} \rfloor 
\frac{n}{k}\rceil, k - \lfloor\frac{k+1}{2}\rfloor) + O(n) & \mbox{if } k > 
1 \\ T(n, k) = O(n) & \mbox{otherwise}\\
\end{array}
\right.
$$
I will not prove that $T(n) = O(n\lg k)$. The proof seems quite messy.

\subsection*{9.3-7}
The median $x$ can be found by the procedure SELECT in $O(n)$ time. After 
finding the median $x$, we call the modified procedure SELECT' to find the kth 
element. In the modified procedure SELECT', the elements, for instance $a, b$ 
are compared by the distance to the median x($|a-x| - |b-x|$). The comparison 
takes $\Theta(1)$ time. So the modified SELECT' still runs in $O(n)$ time. After 
the kth element is found by SELECT', it is used as the pivot to partition the 
set $S$, and $S[1\dots k]$ is the k numbers that are closest to the median.

\subsection*{9.3-8}
Let $Z$ be the union of array $X$ and array $Y$ and in sorted order. For the 
$k$-th statistic of Z $a$, $a$ must be greater than exactly $(k-1)$ numbers. If 
$a$ is $X[x]$, we must have $Y[n-x] \leq X[p] \leq Y[n-x+1]$.  Why?  $X[p]$ is 
greater than exactly p-1 numbers in X. And $Y[n-p] \leq X[p] \leq Y[n-p+1]$, 
then X[p] is greater than exactly n-p numbers in Y.  Thus, X[p] is greater than 
exactly $k-1$ numbers in total and $X[x]$ must be the $k$-th statistic $a$.
\begin{codebox}
\Procname{$\proc{Median}(A,B,n)$}
\li $median \gets \proc{Select}(A,1,n,B,1,n,n)$
\li \Return $median$
\end{codebox}

\begin{codebox}
\Procname{$\proc{Select}(A,p,r,B,q,s,k)$}
\li $x \gets \lfloor(k+1)/2\rfloor$

\li \If $A[p+x-1] \isequal A[q+k-x-1]$
\li \Then \Return $A[p+x-1]$
\li \ElseIf $A[p+x-1] > A[q+k-x-1]$
\li \Then \If $A[p+x-1] \le A[q+k-x]$
\li \Then \Return $A[p+x-1]$
\li \Else $\proc{Select}(A, p, p+x-2, B, q+x-k, s,x)$
\End
\li \Else \If $A[q+k-x-1] \le A[p+x]$
\li \Then \Return $A[q+k-x-1]$
\li \Else $\proc{Select}(A, p+x, r, B, q, q+x-k-2,k-x)$
\End
\End
\end{codebox}

Some boundary case checks are omitted for simplicity. And by intuition, $T(n) = 
T(n/2) + \Uptheta(1)$. Solve it with master theorem. We have $T(n) = 
\Uptheta(\lg n)$.

\subsection*{9.3-9}
Let $(x_1, y_1), (x_2, y_2), \cdots, (x_n, y_n)$ be the coordinates of n wells. 
We want to find a number $y^*$ such that $$d(y^*) = \sum_{i = 1}^n |y_i - y^*|$$ 
achieves the minimum. Note that $$|y_i - y^*|$$ is the distance between $y_i$ 
and $y^*$. The median of $y_1, y_2, \ldots, y_n$ will always have less value 
than any other fixed number substituted into function $d(\cdot)$. So the optimal 
location can be determined by SELECT in $O(n)$ time.

\subsection*{9.2}
\textbf{a:} The lower median $x$ of the elements $x_1,x_2,\dots,x_n$ satisfying 
$|\{x_i: 1\le i \le n \text{ and } x_i < x\} \le \lceil n/2 \rceil -1 < n/2$ and 
$|\{x_i: 1\le i \le n \text{ and } x_i > x\}| \le \lfloor n/2 \rfloor$.

\begin{align*}
\sum_{x_i < x} w_i & = \sum_{x_i < x} \frac{1}{n} \\
& = \frac{1}{n}\cdot \sum_{x_i < x} 1 \\
& = \frac{1}{n}\cdot |\{x_i: 1 \le i \le n \text{ and } x_i < x\}| \\
& < \frac{1}{n} \cdot \frac{n}{2} \\
& = \frac{1}{2}
\end{align*}

\begin{align*}
\sum_{x_i > x} w_i & = \sum_{x_i > x} \frac{1}{n} \\
& = \frac{1}{n} \cdot \sum_{x_i > x} 1 \\
& = \frac{1}{n} \cdot |\{x_i: 1\le i \le n \text{ and } x_i > x\}| \\
& \le \frac{1}{n}\cdot\lfloor\frac{n}{2}\rfloor \\
& \le \frac{1}{n}\cdot \frac{n}{2} \\
& = \frac{1}{2}
\end{align*}

\subsection*{9.3}
\textbf{a:} We can give a intuitive proof of $U_i(n)+O(T(2i)\lg(n/i))$ by 
recursion-tree method. First we ignore floors and ceilings:
\begin{equation*}
U_i(n) = \left\{\begin{array}{lr}
T(n) & \text{if }2i \ge n, \\
n/2 + U_i(n/2) + T(2i) & \text{otherwise},
\end{array} \right.
\end{equation*}
The cost for every intermediate level is $\frac{n}{2^{h+1}}+T(2i)$. $h$ is level 
depth. The tree ends when $\frac{n}{2^h}=2i$. Solve it. We have $h=\lg(n/i)-1$.  
So the total cost:
\begin{align*}
U_i(n) & = \sum_{h=0}^{lg(n/i)-2}(\frac{n}{2^{h+1}}+T(2i))+T(2i) \\
& = \sum_{h=0}^{lg(n/i)-2}\frac{n}{2^{h+1}}+(\lg(n/i)-1)T(2i) \\
& = \frac{n}{2}\sum_{h=0}^{lg(n/i)-2}(\frac{1}{2})^h+(\lg(n/i)-1)T(2i) \\
& = 
\frac{n}{2}\cdot\frac{(\frac{1}{2})^{\lg(n/i)-1}-1}{\frac{1}{2}-1}+(\lg(n/i)-1)T(2i) 
\\
& = (1-(\frac{1}{2})^{\lg(n/i)-1})n + (\lg(n/i)-1)T(2i) \\
& = (1-\frac{2^{\lg i + 1}}{n})n + (\lg(n/i)-1)T(2i)
\end{align*}

Now we show by substitution that if $i < n/2$, then $U_i(n) = n + 
O(T(2i)\lg(n/i))$. In particular, we shall show that $U_i(n) \le n + 
cT(2i)\lg(n/i) - d(\lg\lg n)T(2i) = n + cT(2i)\lg n - cT(2i)\lg i - d(\lg\lg 
n)T(2i)$ for some positive constants $c$ and $d$, and $n\ge4$. We have
\begin{align*}
U_i(n) & = \lfloor n/2 \rfloor + U_i(\lceil n/2 \rceil) + T(2i) \\
& \le \lfloor n/2 \rfloor + \lceil n/2 \rceil + cT(2i)\lg\lceil n/2 \rceil - 
cT(2i)\lg i -d(\lg\lg\lceil n/2 \rceil)T(2i) + T(2i) \\
& = n + cT(2i)\lg\lceil n/2 \rceil -cT(2i)\lg i - d(\lg\lg\lceil n/2 
\rceil)T(2i) + T(2i) \\
& \le n + cT(2i)\lg(n/2+1) -cT(2i)\lg i - d(\lg\lg(n/2))T(2i) + T(2i) \\
& = n + cT(2i)\lg(n/2+1) - cT(2i)\lg i - d(\lg(\lg n - 1))T(2i) + T(2i) \\
& \le n + cT(2i)\lg n - cT(2i)\lg i - d(\lg\lg n)T(2i)
\end{align*}

If $cT(2i)\lg(n/2+1) -d(\lg(\lg n -1))T(2i) + T(2i) \le cT(2i)\lg n - d(\lg\lg 
n)T(2i)$. Simple algebraic manipulations gives the following sequence of 
equivalent conditions:

\begin{align*}
cT(2i)\lg(n/2+1) -d(\lg(\lg n -1))T(2i) + T(2i) & \le cT(2i)\lg n - d(\lg\lg 
n)T(2i) \\
c\lg(n/2+1) - d(\lg(\lg n - 1)) + 1 & \le c\lg n - d(\lg\lg n) \\
c(\lg(n/2+1)-\lg n) - d(\lg(\lg n -1) - \lg\lg n) + 1 & \le 0 \\
c\lg(\frac{1}{2}+\frac{1}{n}) - d\lg\frac{\lg n -1}{\lg n} + 1 & \le 0
\end{align*}

Observe that $1/2+1/n$ decreases as $n$ increases, but $(\lg n -1)/\lg n$ 
increases as $n$ increases. When $n = 4$, we have $1/2+1/n=3/4$ and $(\lg n - 
1)/\lg n = 1/2$. We just need to choose $d$ such that $c\lg(3/4) - d\lg(1/2) + 1 
  \le 0$. Solve it. We have $d \le c\lg(4/3) - 1$.

\textbf{c:} When $i$ is a constant, $T(2i) = O(1)$ and $\lg(n/i) = \lg n - \lg i 
= O(\lg n)$. Thus, when $i$ is a constant less than $n/2$, we have that
\begin{align*}
U_i(n) & = n + O(T(2i)\lg(n/i)) \\
& = n + O(O(1)\cdot O(\lg n)) \\
& = n + O(\lg n)
\end{align*}


\textbf{d:} Suppose that $i = n/k$ for $k \ge 2$. Then $i \le n/2$. If $k > 2$, 
then $i < n/2$, and we have
\begin{align*}
U_i(n) & = n + O(T(2i)\lg(n/i)) \\
& = n + O(T(2n/k)\lg(n/(n/k)) \\
& = n + O(T(2n/k)\lg k).
\end{align*}

If $k=2$, then $n = 2i$ and $\lg k = 1$. We have

\begin{align*}
U_i(n) & = T(n) \\
& = n + (T(n) - n) \\
& \le n + (T(2i) - n) \\
& = n + (T(2n/k) - n) \\
& = n + (T(2n/k)\lg k - n) \\
& = n + O(T(2n/k)\lg k).
\end{align*}

\subsection*{9.4}
\textbf{a:}

There are 3 cases.  When $k < i <j$, if any elements between $z_k$ and $z_j$ is 
chosen, $z_{ij}$ is broken. In this case, if $z_k$ is chosen, the algorithm 
terminates. If $z_l$ ($k+1 < l <i$) is chosen, the part containing $z_{ij}$ is 
discarded.  When $i \le k \le j$, if any elements between $z_i$ and $z_j$ is 
chosen, $z_{ij}$ is broken.  When $i < j < k$, if any elements between $z_i$ and 
$z_k$ is chosen, $z_{ij}$ is broken. The reason is similar to case 1. In all the 
cases, only if $z_i$ or $z_j$ is chosen, they will be compared.


\begin{equation*}
X_{ijk} = \left\{
\begin{array}{ll}
\frac{2}{j-k+1} & \mbox{if }k < i <j \\
\frac{2}{j-i+1}  & \mbox{if }i \le k \le j \\
\frac{2}{k-i+1} & \mbox{if } i < j < k
\end{array}
\right.
\end{equation*}

\textbf{b:}
Make a sum of $X_{ijk}$, we have:

\begin{align*}
E[X_k] & = \sum X_{ijk} \\
& = \sum_{j=k+2}^{n}\sum_{i=k+1}^{j-1}\frac{2}{j-k+1} + 
\Big(\sum_{i=1}^{k}\sum_{k-1}^{n}\frac{2}{j-i+1} + 
\sum_{j=k+1}^{n}\frac{2}{j-i+1}\Big)
+\sum_{i=1}^{k-2}\sum_{j=i+1}^{k-1}\frac{2}{k-i+1} \\
& = 2\sum_{j=k+2}^{n}\frac{j-k-1}{j-k+1} + 
\Big(\sum_{i=1}^{k}\sum_{k-1}^{n}\frac{2}{j-i+1} + 
\sum_{j=k+1}^{n}\frac{2}{j-i+1}\Big)
+ 
2\sum_{i=1}^{k-2}\frac{k-i-1}{k-i+1}\\
& = 2\sum_{j=k+1}^{n}\frac{j-k-1}{j-k+1} + 
\Big(\sum_{i=1}^{k}\sum_{k-1}^{n}\frac{2}{j-i+1} + 
\sum_{j=k+1}^{n}\frac{2}{j-i+1}\Big) + 2\sum_{i=1}^{k-2}\frac{k-i-1}{k-i+1}\\
& \le 2\sum_{j=k+1}^{n}\frac{j-k-1}{j-k+1} + 
\sum_{i=1}^{k}\sum_{k}^{n}\frac{2}{j-i+1} + 2\sum_{i=1}^{k-2}\frac{k-i-1}{k-i+1} 
\\
& = 2\Big(\sum_{i=1}^{k}\sum_{j=k}^{n}\frac{1}{j-i+1} + 
\sum_{j=k+1}^{n}\frac{j-k-1}{j-k+1} + \sum_{i=1}^{k-2}\frac{k-i-1}{k-i+1}\Big)
\end{align*}

\textbf{c:}

\begin{align*}
\sum_{j=k+1}^{n}\frac{j-k-1}{j-k+1} + \sum_{i=1}^{k-2}\frac{k-i-1}{k-i+1} & = 
\sum_{i=2}^{n-k+1}\frac{i-2}{i} + \sum_{i=3}^{k}\frac{i-2}{i} \\
& \le \sum_{i=2}^{n-k+1}1 + \sum_{i=3}^{k}1 \\
& = (n-k) + (k-2) \\
& = n-2 \\
& \le n
\end{align*}

\begin{align*}
\sum_{i=1}^{k}\sum_{j=k}^{n}\frac{1}{j-i+1} & = 
\sum_{i=1}^{k-1}\sum_{j=1}^{i}\frac{1}{i} + \sum_{i=k}^{n-k+1}\frac{k}{i} + 
\sum_{i=n-k+2}^{n}\sum_{j=1}^{n-i+1}\frac{1}{i} \\
& = \sum_{i=1}^{k-1}1 + \sum_{i=k}^{n-k+1}\frac{k}{i} + 
\sum_{i=n-k+2}^{n}\frac{n-i+1}{i} \\
& = (k-1) + \sum_{i=k}^{n-k+1}\frac{k}{i} + \sum_{i=n-k+2}^{n}\frac{n-i+1}{i} \\
& \le (k-1) + \sum_{i=k}^{n-k+1}1 + \sum_{i=n-k+2}^{n}\frac{n-i+1}{i} \\
& = (k-1) + (n-2k+2) + \sum_{i=n-k+2}^{n}\frac{n-i+1}{i} \\
& = n-k-1 + \sum_{i=n-k+2}^{n}\frac{n-i+1}{i} \\
& \le n-k-1 + \sum_{i=n-k+2}^{n}\frac{k-1}{n-k+2} \\
& \le n-k-1 + \sum_{i=n-k+2}^{n}1 \\
& = n-k-1 + (k-1) \\
& = n
\end{align*}

So we have
\begin{align*}
E[X_k] \le 2\Bigg(\sum_{i=1}^{k}\sum_{j=k}^{n}\frac{1}{j-i+1} + 
\sum_{j=k+1}^{n}\frac{j-k-1}{j-k+1} + \sum_{i=1}^{k-2}\frac{k-i-1}{k-i+1} \Bigg) 
& \le 2(n+n) \\
& = 4n
\end{align*}

The following matrix is an example for the first term when $n=7$ and $k=3$.
\begin{equation*}
\left(
\begin{array}{ccccccc}
 & &  \frac{1}{3} & \frac{1}{4} & \frac{1}{5} & \frac{1}{6} & \frac{1}{7} \\
 & \frac{1}{2} & \frac{1}{3} & \frac{1}{4} & \frac{1}{5} & \frac{1}{6} \\
1 & \frac{1}{2} & \frac{1}{3} & \frac{1}{4} & \frac{1}{5} \\
\end{array} \right)
\end{equation*}


\textbf{d:} The running time of $\proc{Randomized-Select}$ is dominated by the 
time spent in the $\proc{Randomized-Partition}$ procedure. There are at most $n$ 
calls to $\proc{Randomized-Partition}$ procedure. Each call takes $O(1)$ time.  
So the calls of $\proc{Randomized-Partition}$ takes $O(n)$. The number of 
comparisons made $\proc{Randomized-Partition}$'s expected value is $4n$. So 
$\proc{Randomized-Select}$ runs in expected time $O(n)$.

\part{Data Structures}

\chapter{Elementary Data Structures}

\subsection*{10.1-2} One stack grows upwards from $S[1]$. The other stack grows 
downwards from $S[n]$.

\subsection*{10.1-4}
\begin{codebox}
\Procname{$\proc{Enqueue}(Q,x)$}
\li \If $Q.head \isequal Q.tail+1$ or ($Q.head \isequal 1$ and $Q.tail \isequal 
n$)
\li \Then \Error "overflow"
\li \Else $Q[Q.tail] \gets x$
\li \If $Q.tail \isequal Q.length$
\li \Then $Q.tail = 1$
\li \Else $Q.tail \gets Q.tail + 1$
\End
\end{codebox}

\begin{codebox}
\Procname{$\proc{Dequeue}(Q)$}
\li \If $Q.head \isequal Q.tail$
\li \Then \Error "underflow"
\li \Else $x \gets Q[Q.head]$
\li \If $Q.head \isequal Q.length$
\li \Then $Q.head \gets 1$
\li \Else $Q.head \gets Q.head + 1$
\End
\li \Return $x$
\End
\end{codebox}

\subsection*{10.1-5}

Operations at the big end:
\begin{codebox}
\Procname{$\proc{Add-Last}(D, x)$}
\li $D[D.tail] \gets x $
\li \If $D.tail \isequal D.length$
\li \Then $D.tail \gets 1$
\li \Else $D.tail = D.tail + 1$
\End
\end{codebox}

\begin{codebox}
\Procname{$\proc{Remove-Last}(D)$}
\li \If $D.tail \isequal 1$
\li \Then $D.tail \gets D.length$
\li \Else $D.tail = D.tail - 1$
\End
\li \Return $D[D.tail]$
\end{codebox}

The operations at the small end:
\begin{codebox}
\Procname{$\proc{Add-First}(D,x)$}
\li \If $D.head \isequal 1$
\li \Then $D.head \gets D.length$
\li \Else $D.head = D.head - 1$
\End
\li $D[D.head] = x$
\end{codebox}

\begin{codebox}
\Procname{$\proc{Remove-First}(D)$}
\li $x \gets D[D.head]$
\li \If $D.head \isequal D.length$
\li \Then $D.head \gets 1$
\li \Else $D.head \gets D.head + 1$
\End
\li \Return $x$
\end{codebox}

\subsection*{10.1-6}

\begin{description}
\item[Enqueue(Q,x)] Push x onto Stack 1.
\item[Dequeue(Q)] Pop all items from Stack 1 while pushing onto Stack 2.  Now 
the top item in Stack 2 is the one to be dequeued(Pop Stack 2). Pop all the 
remaining items from Stack 2 while pushing onto Stack 1.
\end{description} 

Enqueue takes $O(1)$ time. Dequeue takes $O(n)$ time.

\subsection*{10.1-7}
\begin{description}
\item[Push(S, x)] Dequeue all the items in Queue 1 while enqueuing onto Queue 2.  
Enqueue x onto Queue 1. Dequeue all the items in Queue2 while enqueuing onto 
Queue 1.
\item[Pop(S)] Dequeue from Queue 1. \end{description}

Push takes $O(n)$ time. Pop takes $O(1)$ time.
Another solution:
\begin{description}
\item[Push(item)]	Enqueue item onto Queue 1.
\item[Pop()] Dequeue all items but one from Queue 1 while enqueuing onto Queue 
2. The last item in Queue 1 is what is popped (dequeue Queue 1). Now think of 
   Queue 2 as Queue 1 and vice versa.
\end{description}
Push $O(1)$.	Pop	$O(n)$

\subsection*{10.2-1}
\begin{codebox}
\Procname{$\proc{List-Insert}(L, x)$}
\li $x.next \gets L.head$
\li $L.head \gets x$
\end{codebox}
DELETE can't be implemented in $O(1)$ time since $x$'s predecessor can't be 
accessed in $O(1)$ time.

\subsection*{10.2-2}

\begin{codebox}
\Procname{$\proc{Push}(S, x)$}
\li $\proc{List-Insert}(S, x)$
\end{codebox}

\begin{codebox}
\Procname{$\proc{Pop}(S)$}
\li $x \gets S.head$
\li \If $x \ne $ Nil
\li \Then $S.head \gets x.next$
\End
\li $x.next \gets $ Nil
\li \Return $x$
\end{codebox}

\subsection*{10.2-3}
The singly linked list $L$ needs a additional pointer pointing to the tail of 
the list. Enqueue adds item at the tail of the list. Dequeue removes item at the 
head of the list.

\subsection*{10.2-4}
Assign $L.nil.key$ a value which is impossible for any valid $key$. For example, 
$\infty$.

\subsection*{10.2-5}

\begin{codebox}
\Procname{$\proc{Search}(S, k)$}
\li $x \gets L.head$
\li \If $x \ne nil$ and $x.key \ne k$
\li \Then $x \gets x.next$
\li \While $x \ne L.head$ and $x.key \ne k$
\li \Do $x \gets x.next$
\End
\End
\li \Return $x$
\end{codebox}

\begin{codebox}
\Procname{$\proc{Insert}(S, k)$}
\li $x.next \gets L.head$
\li \If $L.head \isequal \const{nil}$
\li \Then $x.next \gets x$
\li \Else $y \gets L.head$
\li \While $y.next \ne L.head$
\li \Do $y \gets y.next$
\End
\li $y.next \gets x$
\End
\li $L.head \gets x$
\end{codebox}

\begin{codebox}
\Procname{$\proc{Delete}(S, x)$}

\li $p \gets x$
\li \While $p.next \ne x$
\li \Do $p \gets p.next$
\End

\li \If $p \isequal x$
\li \Then $L.head \gets \const{nil}$
\li \Else $p.next \gets x.next$
\li \If $L.head \isequal x$
\li \Then $L.head \gets x.next$
\End
\End
\end{codebox}
SEARCH takes $O(n)$ time. INSERT takes $O(n)$ time. DELETE takes $O(n)$.

\subsection*{10.2-6} Use a singly linked list with a head pointer and a tail 
pointer.
\begin{codebox}
\Procname{$\proc{Union}(S_1, S_2)$}
\li $S_1.tail.next \gets S_2.head$
\li $S.head \gets S_1.head$
\li $S.tail \gets S_2.tail$
\end{codebox}
This procedure takes $O(1)$. It is obvious.

\subsection*{10.2-7}

\begin{codebox}
\Procname{$\proc{Reverse}(L)$}
\li \If $L.head \isequal \const{nil}$
\li \Then \Return
\End

\li $p \gets L.head$
\li $L.head \gets \const{nil}$

\li \While $p \ne \const{nil}$
\li \Do  $x \gets p.next$
\li $p.next \gets L.head$
\li $L.head \gets p$
\li $p \gets x$
\End
\end{codebox}

\subsection*{10.2-8}

\begin{codebox}
\Procname{$\proc{Search}(L,k)$}
\li $y \gets \const{nil}$
\li $x \gets L.head$
\li \While $x.np \ne y$ and $x.key \ne k$
\li \Do $z \gets x$
\li $x \gets y $ xor $x.np$
\li $y \gets z$
\End
\li \If $x.key \isequal k$
\li \Then \Return $x$
\li \Else \Return $\const{nil}$
\End
\end{codebox}

\begin{codebox}
\Procname{$\proc{Insert}(L, x)$}
\li \If $L.head \ne \const{nil}$
\li \Then $L.head.np \gets L.head.np$ xor $x$
\li $x.np \gets L.head$
\End
\li $L.head \gets x$
\end{codebox}

\begin{codebox}
\Procname{$\proc{Delete}(L, x)$}
\li \If $x.np \isequal \const{nil}$
\li \Then $L.head \gets \const{nil}$
\End
\li $q \gets \const{nil}$
\li $p \gets L.head$
\li \While $p \ne x$
\li \Do $t \gets q$
\li $q \gets p$
\li $p \gets t$ xor $p.np$
\End
\li $r \gets x.np $ xor $q$
\li \If $q \ne \const{nil}$
\li \Then $q.np \gets q.np \text{ xor } x \text{ xor } r$
\End
\li \If $r \ne \const{ni}$
\li \Then $r.np \gets r.np \text{ xor } x \text{ xor } q$
\End
\end{codebox}
To reverse such a list, we only need to point $L.head$ to the end of the list.  
So it takes $O(1)$.

\subsection*{10.3-2}

\begin{codebox}

\Procname{$\proc{Allocate-Object}$}
\li \If $free \isequal \const{nil}$
\li \Then \Error "out of space"
\li \Else $x \gets free$
\li $free \gets A[x+1]$
\li \Return $x$
\End
\end{codebox}

\begin{codebox}
\Procname{$\proc{Free-Object}(x)$}
\li $A[x+1] \gets free$
\li $free \gets x$
\end{codebox}



\subsection*{10.3-3}
We don't need to set or reset the $prev$ attributes of objects in the 
implementation of the ALLOCATE-OBJECT and FREE-OBJECT procedure since the INSERT 
method of doubly linked list will handle it.

\subsection*{10.3-4}
Treat the free list of multiple-array representation as a stack growing towards 
the start of the multiple-array.  Initially, $S.top$ is 1.  ALLOCATE-OBJECT 
procedure just calls $POP(S)$ procedure. If an element of the doubly-linked list 
is freed.  First, the freed element $x$ is move to the end of the list while the 
structure of the doubly-linked list is kept.  Then, $PUSH(S, x)$ is called.

\subsection*{10.3-5}

\begin{codebox}
\Procname{$\proc{Compactify-List}(L, F)$}
\li $h \gets 1$
\li $x \gets L$
\li \While $x \ne \const{nil}$
\li \Do \If $x \ne h$
\li \Then \If $h \isequal F$
\li \Then $F \gets x$
\End
\li $\proc{Exchange}(h, x)$
\End
\li $x \gets h.next$
\li $h \gets h + 1$
\End
\li $L \gets 1$
\end{codebox}

\begin{codebox}
\Procname{$\proc{Exchange}(h, x)$}
\li \If $x.next \isequal h$
\li \Then \If $x.prev \ne \const{nil}$
\li \Then $x.prev.next \gets h$
\End
\li \If $h.next \ne \const{nil}$
\li \Then $h.next.prev \gets x$
\End
\li $\proc{Swap-Key}(h, x)$
\li $x.next \gets h.next$
\li $h.prev \gets x.prev$
\li $x.prev \gets h$
\li $h.next \gets x$
\li \Else \li \If $h.prev \ne \const{nil}$
\li \Then $h.prev.next \gets x$
\End
\li \If $h.next \ne \const{nil}$
\li \Then $h.next.prev \gets x$
\End
\li \If $x.prev \ne \const{nil}$
\li \Then $x.prev.next \gets h$
\End
\li \If $x.next \ne \const{nil}$
\li \Then $x.next.prev \gets h$
\End
\li $\proc{Swap}(h,x)$
\End
\end{codebox}

\begin{codebox}
\Procname{$\proc{Swap-Key}(a, b)$}
\li $temp \gets a.key$
\li $a.key \gets b.key$
\li $b.key \gets temp$
\end{codebox}

\begin{codebox}
\Procname{$\proc{Swap-Next}(a, b)$}
\li $temp \gets a.next$
\li $a.next \gets b.next$
\li $b.next \gets temp$
\end{codebox}

\begin{codebox}
\Procname{$\proc{Swap-Prev}(a, b)$}
\li $temp \gets a.prev$
\li $a.prev \gets b.prev$
\li $b.prev \gets temp$
\end{codebox}

\begin{codebox}
\Procname{$\proc{Swap}(a,b)$}
\li $\proc{Swap-Next}(a,b)$
\li $\proc{Swap-Key}(a,b)$
\li $\proc{Swap-Prev}(a,b)$
\end{codebox}

Since $\proc{Exchange}$ takes $\Uptheta(1)$ time, $\proc{Compactify-List}$ takes 
$\Uptheta(n)$ time.

\subsection*{10.4-2}
Recursive pre-root traversal.
\begin{codebox}
\Procname{$\proc{Traverse}(node)$}
\li \If $node \ne \const{nil}$
\li \Then print $node.key$
\li $\proc{Traverse}(node.left)$
\li $\proc{Traverse}(node.right)$
\End
\end{codebox}

\subsection*{10.4-3}
Non-recursive pre-root traversal.
\begin{codebox}
\Procname{$\proc{Traverse}(node)$}
\li \If $node \isequal \const{nil}$
\li \Then \Return
\End
\li let S be a n-element stack
\li $\proc{Push}(S,x)$
\li \While not $\proc{Stack-Empty}(S)$
\li \Do $y \gets \proc{Pop}(S)$
\li print y.key
\li \If $y.right \ne \const{nil}$
\li \Then $\proc{Push}(S, y.right)$
\End
\li \If $y.left \ne \const{nil}$
\li \Then $\proc{Push}(S, y.left)$
\End
\end{codebox}

\subsection*{10.4-4}
\begin{codebox}
\Procname{$\proc{Traverse}(node)$}
\li \If $node \ne \const{nil}$
\li \Then print $node.key$
\li $\proc{Traverse}(node.left\text{-}child)$
\li $\proc{Traverse}(node.right\text{-}sibling)$
\End
\end{codebox}

\subsection*{10.4-5}
When visiting the tree, the status is either TRAVERSE or BACKTRACK. And 
different statuses are handle differently.
%\lstinputlisting[language=C]{/data/code/algorithm/prog/n-traverse.c}

\subsection*{10.4-6}
$p$ points to either parent or right-sibling. If $flag$ is PARENT, then $p$ 
points to parent. Otherwise, it points to right-sibling.
%\lstinputlisting[language=C]{/data/code/algorithm/prog/navigate.c}

\subsection*{10.1}
\begin{tabular}{p{3.5cm}|p{1.5cm}|p{1.5cm}|p{1.5cm}|p{1.5cm}|}
& unsorted, singly linked & sorted, singly linked & unsorted, doubly linked & 
sorted, doubly linked \\ \hline
$\proc{Search}(L,k)$ & $O(n)$ & $O(n)$ & $O(n)$ & $O(n)$ \\ \hline
$\proc{Insert}(L,x)$ & $O(1)$ & $O(n)$ & $O(1)$ & $O(n)$ \\ \hline
$\proc{Delete}(L,x)$ & $O(n)$ & $O(n)$ & $O(1)$ & $O(1)$ \\ \hline
$\proc{Successor}(L,x)$ & $O(1)$ & $O(1)$ & $O(1)$ & $O(1)$ \\\hline
$\proc{Predecessor}(L,x)$ & $O(n)$ & $O(n)$ & $O(1)$ & $O(1)$ \\ \hline
$\proc{Minimum}(L)$ & $O(n)$ & $O(1)$ & $O(n)$ & $O(1)$ \\ \hline
$\proc{Maximum}(L)$ & $O(n)$ & $O(n)$ & $O(n)$ & $O(n)$ \\ \hline
\end{tabular}

\subsection*{10.2}
I don't quite understand the author's meaning. And I get no relevant information 
after searching the web.

\subsection*{10.3}

\textbf{a:}
There are 2 cases. One is that $\proc{Compact-List-Search}(L,n,k)$ returns from 
line 7. In this case, \textbf{for} loop of 
$\proc{Compact-List-Search'}(L,n,k,t)$ runs $t$ times. And the latter returns at 
line-7. The other is that the former returns at line-10 or line-11. In this 
case, the integers returned by $\proc{Random}(1,n)$ don't contain $p(key[p]=k)$.  
So the later will run $t$ iterations of the \textbf{for} loop. And it will 
continue at line 8. So the latter will run $t$ iterations of both \textbf{for} 
loop and \textbf{while} loop.

\textbf{b:}
There are two parts of the running time. One part is the \textbf{while} loop. It 
has $O(t)$ iterations. The other part is the \textbf{for} loop. It has $X_t$ 
iterations. So the expected running time is $O(t+\text{E}[X_t])$.

\textbf{c:}
Assume that $key[p] \isequal k$. $X_t$'s sample space is $\{0, 1, 2, \dots, 
p-1\}$. $X_t \ge r$ means that in all the $t$ iterations, not an element between 
position $p-r+1$ (inclusive) and $p$ (inclusive) is chosen. In every iteration, 
there are $n-r$ such elements which can be chosen. So we have:

\begin{align*}
Pr\{X_t\ge r\} = \big(\frac{n-r}{n}\big)^t
\end{align*}
\begin{align*}
\text{E}[X_t] & = \sum_{r=0}^{p-1}r\cdot Pr\{X_t = r\} \\
& = \sum_{r=0}^{p-1}r\cdot\Big(Pr\{X_t \ge r\}-Pr\{X_t \ge r+1\}\Big) \\
& = \sum_{r=1}^{p-1}Pr\{X_t \ge r\} - (p-1)Pr\{X_t = p\} \\
& = \sum_{r=1}^{p-1}Pr\{X_t \ge r\} \\
& = \sum_{r=1}^{p-1}(\frac{n-r}{n})^t \\
& \le \sum_{r=1}^{n}(\frac{n-r}{n})^t \\
\end{align*}

\textbf{d:}
\begin{align*}
\sum_{r=0}^{n-1}r^t & \le \int_{0}^{n} r^t {\mathrm{d}r} \\
& = \Big[\frac{r^{t+1}}{t+1}\Big]_0^n \\
& = n^{t+1}/(t+1)
\end{align*}

\textbf{e:}
\begin{align*}
\text{E}[X_t] & \le \sum_{r=1}^{n}(1-r/n)^t \\
& = \sum_{r=1}^{n}(\frac{n-r}{n})^t \\
& = \frac{1}{n^t}\sum_{r=1}^{n}(n-r)^t \\
& = \frac{1}{n^t}\sum_{r=0}^{n-1}r^t \\
& \le \frac{1}{n^t}\frac{n^{t+1}}{t+1} \\
& = \frac{n}{t+1}
\end{align*}

\textbf{f:}
\begin{align*}
T(n) & = O(t+\text{E}[X_t]) \\
& \le O(t+n/(t+1)) \\
& \le O(t+n/t) \\
\end{align*}

\textbf{g:}

The running time of $\proc{Compact-List-Search}$ is the expected number of 
\textbf{while} loop. Both the \textbf{for} loop logic and \textbf{while} loop 
($i = next[i]$) in $\proc{Compact-List-Search'}$ runs the same number of 
iterations in $\proc{Compact-List-Search}$. So we have $O(t) = O(n/t)$. Solve 
it, we have $\sqrt{n}$. So $\proc{Compact-List-Search}$ runs in $O(\sqrt{n})$ 
expected time.


\textbf{h:}
If there are some duplicated keys in the list, the effect of random skips will 
be reduced. For example, if all keys before $k$ are the same, the random skips 
does not have any effect.

\chapter{Hash Tables}
\subsection*{11.1-1}

\begin{codebox}
\Procname{$\proc{Search-Max}(T)$}
\li \For $i \gets m-1$ \To $0$
\li \Do \If $T[i] \ne \const{nil}$
\li \Then \Return $T[i]$
\End
\End
\li \Return $\const{nil}$
\end{codebox}
The worst-case performance is $\Uptheta(n)$.

\subsection*{11.1-2}
%\lstinputlisting[language=C]{/data/code/algorithm/prog/bit-vector.c}

\subsection*{11.1-3}
For insert, it means to create a doubly linked list and let the direct-address 
table element points to it. For search, return the head to the doubly linked 
list. For delete, just delete the element from the doubly linked list. Since the 
list is doubly linked, it can be done in $O(1)$.

\subsection*{11.1-4}
The answer is in "Instructor's Manual".
C program is as follows:
%\lstinputlisting[language=C]{/data/code/algorithm/prog/huge.c}

\subsection*{11.2-3}
Average-case running time:
\begin{description}
\item[successful searches] $O(1+\alpha)$
\item[unsuccessful searches] $O(1+\alpha)$
\item[insertions] $O(1+\alpha)$
\item[deletions] $O(1)$
\end{description}
For the analysis of successful searches, we assume that $k_i$ and $k_j$ ($h(k_i) 
= h(k_j)$). $k_j$ is inserted before $k_i$. This assumption does not influence 
the running time of successful searches. So the analysis in the book applies.

\subsection*{11.2-5}
Assume that no more that $n-1$ keys map to a slot. We have
\begin{align*}
|U| &= \sum_{i=0}^{m-1}k_i \\
&\le \sum_{i=0}^{m-1} (n-1) \\
& = m(n-1)
\end{align*}

$k_i$ is the number of keys which maps to $i$ slot. We have $|U|\le m(n-1)$ 
which contradicts with $|U|>nm$. So $U$ has a subset of size $n$ consisting of 
keys that all hash to the same slot.

\subsection*{11.2-6}
The C program:

%\lstinputlisting[language=C]{/data/code/algorithm/prog/random-hash-element.c}

The sample space of the random number is $\{0,1, \cdot, Lm-1\}$. The probability 
of selecting a key is $1/(Lm)$. There are two parts of the running time $X$.  
$X=Y+Z$.  $Y$ is the number of $\proc{select\_hash\_element}$ function 
invocations. $Z$ is the number of linked list elements visited. $X$ is a random 
variable for geometric distribution. 

\begin{align*}
\text{E}[Y] &= \sum_{i=1}^{\infty}i\cdot Pr\{X=i\} \\
& = \sum_{i=1}^{\infty}i\cdot\frac{n}{Lm} \\
& = \frac{Lm}{n}
\end{align*}

\begin{align*}
\text{E}[Z] \le L
\end{align*}

\begin{align*}
\text{E}[X] &= \text{E}[Y+Z] \\
&=\text{E}[Y] + \text{E}[Z] \\
& \le \frac{Lm}{n}+L \\
& = L\cdot(1+1/\alpha)
\end{align*}

So the expected running time is $O(L\cdot(1+1/\alpha))$.

\subsection*{11.3-1}
Let $a$ denotes the given key. When traversing the linked list, compare $h(a)$ 
against $h(k)$. Only if $h(a) \isequal h(k)$, do the comparsion between $a$ and 
$k$.

\subsection*{11.3-2}
The solution is based on the following properties of modular addition and 
multiplication:
\begin{align*}
a + b & \equiv a' + b' & (\mbox{mod}\;n), \\
ab & \equiv a'b' & (\mbox{mod}\;n).
\end{align*}
\begin{codebox}
\Procname{$\proc{Hash}(A)$}
\li $power \gets 1$
\li $hash \gets 0$
\li \For $i \gets 0$ \To $r$
\li \Do $hash \gets (hash + (A[i] \cdot power) \mod m) \mod m$
\li $power \gets (power \cdot 128) \mod m$
\End
\li \Return $hash$
\end{codebox}

\subsection*{11.3-3}
Given a string $a_0 a_1 \dots a_{r-1}$. Let's compute its hash value. $k$ is 
$\sum_{i=0}^{r-1}a_i 2^{ip}$.

hash value:
\begin{align*}
hash(k) & = k \mod m \\
& = (\sum_{i=0}^{r-1}a_i 2^{ip}) \mod (2^p-1) \\
& = \sum_{i=0}^{r-1}(a_i 2^{ip}) \mod (2^p-1) \\
& = \sum_{i=0}^{r-1}(a_i ((2^{p}-1)+1)^i) \mod (2^p-1) \\
& = \sum_{i=0}^{r-1}(a_i (\sum_{j=0}^{i}b_j(2^p-1)^j) \mod (2^p-1) \\
& = \sum_{i=0}^{r-1}(a_i + (a_i\sum_{j=1}^{i}b_j(2^p-1)^j) \mod (2^p-1) \\
& = \sum_{i=0}^{r-1}\Big(a_i\mod (2^p-1) + (a_i\sum_{j=1}^{i}b_j(2^p-1)^j) \mod 
(2^p-1) \Big)\\
& = \sum_{i=0}^{r-1}\Big(a_i\mod (2^p-1)\Big)
\end{align*}

We can see that the hash value of $a_0 a_1 \dots a_{r-1}$ does not relate to the 
positions of individual characters. So if we can derive string $x$ from string 
$y$ by permuting its characters, then $x$ and $y$ hash to
the same value.

In Insturctor's Manual, there is another solution.

\subsection*{11.3-4}
$$A=(\sqrt{5}-1)/2=0.6180339887498949$$
$$h(61) = \lfloor 1000 (61A \mod 1)\rfloor = 700$$
$$h(62) = \lfloor 1000 (62A \mod 1)\rfloor = 318$$
$$h(63) = \lfloor 1000 (63A \mod 1)\rfloor = 936$$
$$h(64) = \lfloor 1000 (64A \mod 1)\rfloor = 554$$
$$h(65) = \lfloor 1000 (64A \mod 1)\rfloor = 172$$

\subsection*{11.3-5}
In Insturctor's manual, there is a solution. The key of proof is the following 
lemma:

The total number of collisions is minimized when $u_j = u/b$ for each $j \in B$.

But the proof does not apply if $b$ is not a divisor of $u$. 


We will prove that for $x_i, (1 \le i \le n)$. If $\sum_{i=1}^{n}x_i=k$, then 
$\sum_{i=1}^{n}x_i(x_i-1)$ is minimum when $x_i=k/n$. There are two ways to 
prove it. One is to use Lagrange multipliers. Here we show the other way. 

\begin{align*}
\sum_{i=1}^{n}x_i(x_i-1) & = 
\sum_{i=1}^{n}\Big((x_i-\frac{1}{2})^2-\frac{1}{4}\Big) \\
& = \Big(\sum_{i=1}^{n}(x_i-\frac{1}{2})^2\Big)-\frac{n}{4}
\end{align*}

Let $t_i=x_i-1/2, (1\le i \le n)$. We have $\sum_{i=1}^{n}t_i=k-n/2$. Let 
$nc=k-n/2$. So:

\begin{align*}
\sum_{i=1}^{n}x_i(x_i-1) & = \sum_{i=1}^{n}t_i^2-\frac{n}{4} \\
& = \sum_{i=1}^{n}(c+t_i-c)^2 \\
& = c^2n-2c\sum_{i=1}^{n}(t_i-c)+\sum_{i=1}^{n}(t_i-c)^2 \\
& = c^2n+\sum_{i=1}^{n}(t_i-c)^2 \\
\end{align*}
The right hand is obviously minimized when $t_i=c$. Since $t_i=x_i-1/2$. We have 
$x_i=c+1/2=(k-n/2)n+1/2=k/n$. For a detailed discussion of this solution, refer 
to 
\href{http://math.stackexchange.com/questions/67192/how-to-prove-the-sum-of-squares-is-minimum}{How 
to prove the sum of squares is minimum?}.

\chapter{Binary Search Tree}

$successor$ and $predecessor$ are symmetic. The case of $successor$ w/ right 
subtree is the case of $predecessor$ w/o left subtree. The case of $predecessor$ 
w/ left subtree is the case of $successor$ w/o right subtree. So to analyze the 
case of $successor$ w/o right subtree uses the case of $predecessor$ w/ left 
subtree.

$\proc{Tree-Delete}$ handle case c and d in right subtree. It can also handle
them in left subtree. The method is symmetric.

\subsection*{MyProblem-1}
Given a sorted array. Design an algorithm that creates an AVL using array 
elements.

\begin{codebox}
\Procname{$\proc{To-Avl}(A, l, h)$}
\li \If $l > h$
\li \Then \Return Nil
\End
\li $m = \lfloor(l + h)/2 \rfloor $
\li $root = \proc{Allocate-Node}()$
\li $root.key = A[m]$
\li $root.left = \proc{To-Avl}(A, l, m-1)$
\li $root.right = \proc{To-Avl}(A, m+1, h)$
\li \Return $root$
\end{codebox}

Now let's prove that the resulted tree returned from $\proc{To-Avl}$ is a AVL. 
Let $n = h-l+1$ and use $f(n)$ represent $\proc{To-Avl}(A, n)$. We will prove 
the following \textbf{Statement}:

$f(i)$ ($i = 0...n$, $n \ge 1$) is an AVL. And $f(i)$'s and $f(i-1)$'s heights 
differs at most $1$.

\begin{proof} \textbf{Base case:} For $n = 1$, $f(0)$ is empty which is 
  a AVL. $f(1)$ is a single node which is also an AVL. $f(0)$''s height is -1.
  And $f(1)$'s height is 1. Their heights differs with 1. So for $n = 1$, the 
  statement is $true$. 

\textbf{Induction step:} 
If $n$ is odd, $f(n)$ is constructed as $f(n).key = A[\frac{n-1}{2}]$, 
$f(n).left = f(\frac{n-1}{2})$ and $f(n).right = f(\frac{n-1}{2})$. Because 
$\frac{n-1}{2} < n$, $f(\frac{n-1}{2})$ is AVL from induction.
Since $f(n).left$ and $f(n).right$ have the same shape, $f(n)$ is a AVL. 
$f(n-1)$ is constructed as $f(n).key = A[\frac{n-1}{2}-1]$, $f(n).left = 
f(\frac{n-1}{2}-1)$, and $f(n).right = f(\frac{n-1}{2})$. The proof for $f(n-1)$ 
being an AVL is similar to $f(n)$. The subtrees for $f(n)$ and $f(n-1)$ are 
$f(\frac{n-1}{2}-1)$ or $f(\frac{n-1}{2})$. By induction, we know that they are 
AVL whose heights differs at most 1. So $f(n)$ and $f(n-1)$ are AVL trees whose 
heights differ at most 1.

If $n$ is even, it can be proved that $f(n)$ and $f(n-1)$ both are AVLs. And their 
subtrees are $f(\frac{n}{2}-1)$ and $f(\frac{n}{2})$. So $f(n)$ and $f(n-1)$ 
are AVLs whose heights differs at most 1.
\end{proof}


\chapter{Red-Black Trees}
\chapter{Augmenting Data Structures}

\subsection*{14.3-4}
\url{http://fileadmin.cs.lth.se/cs/Personal/Rolf_Karlsson/tut2.pdf} solves with
the use of the solution to 14.3.-3. It is complicated. Actually, searchAll method in
\url{http://algs4.cs.princeton.edu/93intersection/IntervalST.java.html} does the
same thing, but it is much simpler.

\begin{codebox}
\Procname{$\proc{Interval-Search-All}(x, i, list)$}
\li $foundInLeft = False$
\li $foundInX = False$
\li $foundInRight = False$
\li \If $x.left \ne nil$ and $x.left.max \ge i.low$
\li \Then $foundInLeft = \proc{Interval-Search-All}(x.left, i, list)$
\End
\li \If $i$ overlaps $x.int$
\li \Then $\proc{Add}(list, x.int)$
\li $foundInX = True$
\End
\li \If $foundInLeft$ or $x.left = nil$ or $x.left.max < i.low$
\li \Then $foundInRight = \proc{Interval-Search-All}(x.right, i, list)$
\End
\li \Return $foundInLeft$ or $foundInX$ or $foundInRight$
\End
\end{codebox}

The above method is an inorder walk of the interval tree, it is obvious that the time complexity is less than $O(n)$.
The first interval found by this method is the min interval. And it costs $\lg(n)$. After it, $foundInLeft$ is true. The
method will contiue to find the second min interval since the method performs an inorder walk of the tree. And it wil
also cost $\lg(n)$. The method will find all $k$ intervals in this way. And the whole mehod will cost $(k+1)\lg(n) =
O(k\lg(n))$. So the conclusion is $O(min(n, k \lg n))$.

\url{https://godoc.org/github.com/biogo/store/interval} augments a LLRB with an additional $min$. Its $Get$ method is
equivalent to $\proc{Interval-Search-All}$.  And its time complexity is also $O(min(n, k \lg n))$. It does not need to
check $foundInLeft$ because of the existence of $min$. $foundInLeft = True$ means that $left.min > i.high$. Figure
14.5(b) in CLRS shows such a case. And $foundInLeft = False$ means: $x.left \ne null$ and $x.left.max \ge i.low$ and $foundInLeft =
   False$.

\part{Advanced Design and Analysis Techniques}
\part{Advanced Data Structures}
\chapter{B-Trees}
CLLS and Algorithms has a different definition of B-Tree. A B-Tree with 
$t=3$ in CLRS has $3$, $4$, $5$ and $6$ children. A B-Tree with $M=6$ in Algorithms 
can has $3$, $4$ and $5$ children. And Algorithms B-Tree is a B+-Tree since it only 
has data in leaf nodes.

\subsection*{18.3-2}

\begin{codebox}
\Procname{$\proc{B-Tree-Delete}(T, k)$}
\li $i = x.n$
\li \While $i \ge 1$ and $k < x.key_i$
\li \Do $i = i -1$
\End
\li \If $x.leaf$
\li \Then \If $i \ge 1$ and $x.key_i = k$
\li \Then \Comment case 1
\li $\proc{Delete-Key}(x, i)$
\li \Return $True$
\li \Else
\li \Return $False$
\End
\End
\li \If $x.key_i = k$
\li \Then \If $x.c_i.n \ge t$
\li \Then \Comment case 2a
\li $x.key_i = Min(x.c_i)$
\li \ElseIf $x.c_{i+1}.n \ge t$
\li \Then \Comment case 2b
\li $x.key_i = Max(x.c_{i+1})$
\li \Else \Comment case 2c
\li $\proc{Merge-With-Right}(x,i)$
\End
\li \Return $true$
\End
\li $i = i + 1$
\li \If $x.c_{i}.n < t$
\li \Then \Comment case 3a and 3b
\li \If $i > 1$ and $x.c_{i-1}.n \ge t$
\li \Then $\proc{Borrow-From-Left}(x, i)$
\li \ElseIf $i \le x.n$ and $x.c_{i+1}.n \ge t$
\li \Then $\proc{Borrow-From-Right}(x, i)$
\li \Else $\proc{Merge}(x, i)$
\End
\End
\li $\proc{B-Tree-Delete}(x.c_i, k)$
\End
\end{codebox}

\begin{codebox}
\Procname{$\proc{Min}(x)$}
\li $y = x$
\li \While $y.leaf = false$
\li \Do $y = y.c_{y.n}$
\End
\li $y.n = y.n - 1$
\li \Return $y.key_{y.n+1}$
\end{codebox}

\begin{codebox}
\Procname{$\proc{Max}(x)$}
\li $y = x$
\li \While $y.leaf = false$
\li \Do $y = y.c_1$
\End
\li $k = y.key_1$
\li \For $j = 1$ to $y.n-1$
\li \Do $y.key_j = y.key_{j+1}$
\End
\li $y.n = y.n - 1$
\li \Return $k$
\end{codebox}

\begin{codebox}
\Procname{$\proc{Delete-Key}(x, i)$}
\li \For $j = i$ to $x.n-1$
\li \Do $x.key_i = x.key_{i+1}$
\End
\end{codebox}

\begin{codebox}
\Procname{$\proc{Borrow-From-Right}(x, i)$}
\li $y = x.c_i$
\li $z = x.c_{i+1}$
\li $y.n = y.n + 1$
\li $y.key_n = x.key_i$
\li $y.c_{y.n+1} = z.c_1$
\li $x.key_i = z.key_1$
\li \For $j = 1$ to $z.n - 1$
\li \Do $z.key_i = z.key_{i+1}$
\End
\li \For $j = 1$ to $z.n$
\li \Do $z.c_i = z.c_{i+1}$
\End
\end{codebox}

\begin{codebox}
\Procname{$\proc{Borrow-From-Left}(x, i)$}
\li $z = x.c_{i-1}$
\li $y = x.c_i$
\li $y.n = y.n + 1$
\li \For $j = x.n - 1$ downto $1$
\li \Do $y.key_{j+1} = y.key_j$
\End
\li \For $j = x.n$ downto $1$
\li \Do $y.c_{j+1} = y.c_j$
\End
\li $y.key_1 = x.key_{i-1}$
\li $y.c_1 = z.c_{z.n+1}$
\li $x.key_{i-1} = z.key_1$
\li $z.n = z.n - 1$
\end{codebox}

\begin{codebox}
\Procname{$\proc{Merge}(x,i)$}
\li \If $i > 1$
\li \Then $\proc{Merge-With-Left}(x, i)$
\li \Else $\proc{Merge-With-Right}(x, i)$
\End
\end{codebox}

\begin{codebox}
\Procname{$\proc{Merge-With-Left}(x, i)$}
\li $y = x.c_i$
\li $z = c.c_{i-1}$
\li \For $j = y.n$ downto $1$
\li \Do $y.key_{j+z.n+1} = y.key_j$
\End
\li \For $j = y.n+1$ downto $1$
\li \Do $y.c_{j+z.n+1} = y.c_j$
\End
\li $y.key_{z.n + 1} = x.key_{i-1}$
\li \For $j = 1$ to $z.n$
\li \Do $y.key_j = z.key_j$
\End
\li \For $j = 1$ to $z.n + 1$
\li \Do $y.c_j = z.c_j$
\End
\li \For $j = i-1$ to $x.n-1$
\li \Do $x.key_{j} = x.key_{j+1}$
\End
\li \For $j = i$ to $x.n$
\li \Do $x.c_j = x.c_{j+1}$
\End
\li $x.n = x.n - 1$
\li $y.n = y.n + z.n + 1$
\end{codebox}

\begin{codebox}
\Procname{$\proc{Merge-With-Right}(x, i)$}
\li $y = x.c_i$
\li $z = x.c_{i+1}$
\li $y.key_{x.n + 1} = x.key_i$
\li \For $j = 1$ to $z.n$
\li \Do $y.key_{x.n + 1 + j} = z.key_j$ 
\End
\li \For $j = 1$ to $z.n + 1$
\li \Do $y.key_{x.n + 1 + j} = z.c_j$
\End
\li \For $j = i$ to $x.n-1$
\li \Do $x.key_i = x.key_{i+1}$
\End
\li \For $j = i+1$ to $x.n$
\li \Do $x.c_j = x.c_{j+1}$
\End
\li $x.n = x.n - 1$
\li $y.n = y.n + z.n + 1$
\end{codebox}


\part{Graph Algorithms}
\chapter{Elementary Graph Theory}

\subsection*{MyProblem-1}
For directed graph $G = (V, E)$, if every vertex's \textit{\textbf{in-degree}} 
is bigger than $0$, then $G$ has a cycle.

\begin{proof}
  Remove all vertices whose \textit\textbf{out-degree} is $0$. Then each 
  remaining vertex has at lease one incoming edge and at lease one outgoing 
  edge. Choose any vertex $v_1$. Follow one outgoing edge from $v_1$ to another 
  vertex $v_2$.  Again follow one outgoing edge from $v_2$ to another vertex 
  $v_3$ and so on.  We can get an infinite sequence $v_1, v_2, ...$. Since there 
  is only a finite number of vertices. Some vertices must appear more than once 
  in the sequence.  So $G$ has a cycle.
\end{proof}

\part{Selected Topics}

\chapter{Number-Theoretic Algorithms}

\subsection*{31.1-5}
From $n \divides ab$, we have $ab=kn$. From gcd$(a, n)= 1$, we have $ax + ny = 
1$.  Multiply it by $b$, we have $abx + nby = b$. $b$ is a linear combination of 
$ab$ and $n$. Both of them are divisible by $n$. So $n\divides b$.

\subsection*{31.1-11}

\part{Appendix: Mathematical Background}

\chapter{B Sets, Etc.}
\subsection*{B.5-3}
Make the following definitions:
\begin{itemize}
  \item$n$: the total number of nodes
  \item$B$: the number of branches
  \item$n_{0}$: the number of nodes with no children (degree-0 nodes)
  \item$n_{1}$: the number of nodes with a single child (degree-1 nodes)
  \item$n_{2}$: the number of nodes with two children (degree-2 nodes)
\end{itemize}
$B = n -1 $ (since all nodes except the root node come from a single branch).  
$B = n_{1} + 2\cdot n_{2}$ (sincel all branches are from degree-$1$ nodes and 
degree-$2$ nodes). From the above two equations, we have $n = n_{1} + 2\cdot 
n_{2} + 1$. And by $n$'s defintion, we have $n = n_{0} + n_{1} + n_{2}$. From $n 
= n_{1} + 2 \cdot n_{2} + 1$ and $n = n_{0} + n_{1} + n_{2}$, we have $n_{2} = 
n_{0} - 1$.  So the number of degree-2 nodes in any nonempty binary tree is 1 
fewer than the number of leaves.

The above conclusion can also be proved by induction. If the binary tree has 
only $1$ node, $n_{0} = n_{2} + 1$ holds since $n_{0} = 1$ and $n_{2} = 0$.  
Let's assume $n_{0} = n_{2} + 1$ (The binary tree has $n-1$ nodes. And $n-1 \geq 
1$). Now add a new node to this binary tree. If the node node is added to a 
degree-$0$ node, there is a new degree-{0} node. But one existing degree-$0$ 
node becomes degree-$1$ node.  So $n_{0}$ does not change. And $n_{2}$ does not 
change. So the equantion still nodes. If the node is added to a degree-$1$ node, 
there is a new degree-$0$ node and a new degree-$2$ node. So the equation also 
holds. Therefore, the equation holds for a $n$-node binary tree when adding a 
new node. Now we can claim the orignal conclustion.

For a fully binary, there are only degree-0 nodes and degree-2 nodes. So $n_2$ 
is the number of all internal nodes. And $n_{0} = n_{2} + 1$. So the number of 
internal nodes in a full binary tree is 1 fewer than the number of leaves.

\end{CJK}
\end{document}

