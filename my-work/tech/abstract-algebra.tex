\documentclass{book}

\usepackage{amsmath}
\usepackage{enumitem}
\usepackage{amsthm}
\usepackage{amssymb}
\usepackage{extarrows}
\usepackage{graphicx}
\usepackage{tikz}
\usetikzlibrary{
	cd,
	math,
	decorations.markings,
	decorations.pathreplacing,
	positioning,
	arrows.meta,
	circuits.logic.US,
	shapes,
	calc,
	fit,
	quotes}


\begin{document}
If $a \ne 0$, $b = qa + c$, then $a|b$ iff $a|c$.

\begin{proof}
  If $a|c$, we have $c = aq'$. $b = qa + c = qa + aq' = a(q + q')$. So $a|b$. If
  $a|b$, we have $b = as$. We already have $b = qa +c$. Combine these two
  equatiosn to have $as = qa +c$. $c = a(s -q)$. $a|c$.
\end{proof}

$a \equiv b \pmod{m}$ iff $a = q_1 m + r_1, 0 \le r_1 < m$ and $b = q_2 m + r_2,
0 \le r_2 < m$, and $r_1 = r_2$.

\begin{proof}
  From the left side, $a - b = (q_1 - q_2) m + (r_1 - r_2)$. From $m|a-b$, we
  have $m|r_1 - r_2$. And $ 0 \le |r_1 - r_2| < m$, we can have $r1 - r2 = 0$,
  i.e. $r_1 = r_2$.\\
  From the right side, we have $a - b = (q_1 - q_2) m$. So $m|a-b$. $a \equiv b
  \pmod{m}$.
\end{proof}

\end{document}
