\documentclass[a4paper]{article}
\usepackage{fixltx2e} % LaTeX patches, \textsubscript
\usepackage{cmap} % fix search and cut-and-paste in Acrobat
\usepackage{ifthen}
\usepackage[hidelinks]{hyperref}
\usepackage[T1]{fontenc}
\usepackage[utf8]{inputenc}

\usepackage{fancyhdr}
\pagestyle{fancy}

%%% Custom LaTeX preamble
% PDF Standard Fonts
\usepackage{mathptmx} % Times
\usepackage[scaled=.90]{helvet}
\usepackage{courier}
\usepackage{CJKutf8}


%%% User specified packages and stylesheets
%%% Fallback definitions for Docutils-specific commands

% providelength (provide a length variable and set default, if it is new)
\providecommand*{\DUprovidelength}[2]{
  \ifthenelse{\isundefined{#1}}{\newlength{#1}\setlength{#1}{#2}}{}
}

% lineblock environment
\DUprovidelength{\DUlineblockindent}{2.5em}
\ifthenelse{\isundefined{\DUlineblock}}{
  \newenvironment{DUlineblock}[1]{%
    \list{}{\setlength{\partopsep}{\parskip}
            \addtolength{\partopsep}{\baselineskip}
            \setlength{\topsep}{0pt}
            \setlength{\itemsep}{0.15\baselineskip}
            \setlength{\parsep}{0pt}
            \setlength{\leftmargin}{#1}}
    \raggedright
  }
  {\endlist}
}{}

% hyperlinks:
\ifthenelse{\isundefined{\hypersetup}}{
  \usepackage[colorlinks=true,linkcolor=blue,urlcolor=blue]{hyperref}
  \urlstyle{same} % normal text font (alternatives: tt, rm, sf)
}{}


%%% Body
\begin{document}
\begin{CJK}{UTF8}{gkai}

\section*{个人信息}

\begin{DUlineblock}{0em}
\item[] 姓名:么敬国
\item[] 性别:男
\item[] 手机:(86)139-1033-4380
\item[] Email:yaojingguo@gmail.com
\item[] Web site: \url{http://yaojingguo.github.io/},
\url{https://github.com/yaojingguo},
\item[] 出生日期: 1976年
\end{DUlineblock}


\section*{自我评价}
\begin{itemize}
\item 12年软件研发和管理经验。
\item 能熟练使用Java,C/C++,Go, Python和Lisp。熟悉Scala和Ruby。
\item 掌握分布式数据库系统设计和开发相关理论。
\item 熟练使用主流大数据系统和掌握数据分析技术。丰富的Hadoop平台上数据分析应用开发经验。
\end{itemize}

\section*{英语}
CET6优秀,口语流利,多年和海外团队合作经验。

\section*{教育,认证和培训}
\begin{itemize}
\item 北京理工大学       计算机应用技术            硕士,2004年4月
\item 北京理工大学       计算机科学                学士,1999年7月
\item MIT-6.033 Computer Systems Engineering
\item MIT-6.830 Database Systems
\item MIT-6.828 Operating System Engineering
\item MIT-6.824 Distributed Systems
\item Stanford-CS143 Compilers
\end{itemize}

\section*{工作经历}
\subsection*{北京新东方教育科技集团 (2014/05 -{}- 至今)}
\subsubsection*{信息管理部}
\begin{DUlineblock}{0em}
\item[] 首席架构师
\item[] 2014/05 -{}- 至今
\end{DUlineblock}
\begin{itemize}
\item 新东方集团核心系统的架构,开发和调优:优能一对一系统和我学系统等。
\item 基于Hadoop/Spark/MemSQL建设新东方大数据平台。
\item 基于Kafka设计开发新东方数据总线系统。
\item 参与数据挖掘系统的开发: 新东方游学学员推荐,搜课推荐和新东方前途出国重复文书识别。
\item 参与CockroachDB开发,主要PR:
\begin{itemize}
  \item \href{https://github.com/cockroachdb/cockroach/pull/8867}{WIP perf: replace LLRB-tree with btree in interval.Tree}
  \item \href{https://github.com/cockroachdb/cockroach/pull/3912}{sql: allow user priority to be set via SQL}
  \item \href{https://github.com/cockroachdb/cockroach/pull/3783}{make MakePriority function more math precise}
\end{itemize}
\end{itemize}

\subsection*{海南天涯在线网络科技有限公司 (2010/08 -{}- 2014/04)}

\subsubsection*{天涯大数据部}
\begin{DUlineblock}{0em}
\item[] 总监
\item[] 09/2012 -{}- 2014/04
\end{DUlineblock}
\begin{itemize}
\item 数据挖掘,信息过滤,搜索,BI和会员团队管理。
\end{itemize}

\subsubsection*{天涯Hadoop数据分析平台}
\begin{DUlineblock}{0em}
\item[] 技术经理
\item[] Java, Clojure, Hadoop
\item[] 02/2011 -{}- 08/2012
\end{DUlineblock}
%
\begin{itemize}
\item 用Hadoop/Hive/Pig开发天涯的日志分析平台,并向Apache提交Patch。
\item 用Hadoop实现协同过滤推荐算法。
\item 自主开发和维护天涯BI系统。
\end{itemize}

\subsubsection*{天涯虚拟化平台}
\begin{DUlineblock}{0em}
\item[] 资深系统架构师
\item[] 12/2010 -{}- 01/2011
\end{DUlineblock}
天涯虚拟化平台的前期调研和设计工作。

\subsubsection*{Memlink Project}
\begin{DUlineblock}{0em}
\item[] Committer
\item[] C Linux System Programming, libevent
\item[] 09/2010 -{}- 11/2010
\end{DUlineblock}
\url{http://code.google.com/p/memlink}/是一个列表数据存储引擎。我设计实现了主从同步部分。

\subsection*{IBM中国软件开发中心 (2004/11 -{}- 2010/08)}

\subsubsection*{ECM/BPM CoE}
\begin{DUlineblock}{0em}
\item[] COE Leader
\item[] Enterprise Content Mangement/Business Process Management
\item[] 06/2010 -{}- 08/2010
\end{DUlineblock}
ECM/BPM COE的职责是促进IBM ECM/BPM软件产品在中国的成长。我们的主要任务是为ECM/BPM建立重要的样板客户。
我作为COE Leader参加了银行业,保险,制造和核电行业的售前和方案设计工作。其中保险分析项目使用SPSS进行交叉
销售,实用了关联规则和决策树。

\subsubsection*{大数据查询语言Jaql}
\begin{DUlineblock}{0em}
\item[] Technical Leader
\item[] 07/2009 -{}- 05/2010
\item[] Java/C++/Hadoop/SPSS Statistics/SPSS Data Modler
\end{DUlineblock}
Jaql (\url{http://code.google.com/p/jaql})是IBM Hadoop大数据平台上的查询语言, 实现类似Hadoop Pig的功能。在这个
项目中,我和IBM Almaden Research Center的科研人员合作。 我是中国方面的technial leader, 同时是
Jaql项目的committer。我设计实现了Jaql catalog server,SPSS与Jaql的集成和JSON与CSV的转换。
Catalog Server提供一个key-value存储机制。SPSS与Jaql的集成集成的目标是结合SPSS的统计数据挖掘能力和Hadoop
平台的扩展性和并行计算机制。通过这个集成,可以平行的运行某些SPSS的算法。我还实现了JSON和CSV的转换,并做了
系统性能优化,达到了和Pig相当的性能。


\subsubsection*{泰康人寿保全工作流项目}
\begin{DUlineblock}{0em}
\item[] Development Lead/Architect
\item[] 08/2008 -{}- 06/2009
\item[] Java/JSF/IBM ECM/FileNet BPM
\end{DUlineblock}
我领导了泰康人寿保全工作流系统的开发。开发团队有IBM实验室,IBM服务部门,IBM合作伙伴和泰康IT人员组成。
我领导团队按时高标准的完成了泰康个人保险保全项目,并和泰康建立了深厚的客户关系。在项目中,
我设计实现了IBM Content Manager和Filenet BPM的集成。我出差到美国IBM IOD大会上演示了这个集成。
这个集成被当作最主要的功能点放入了IBM CM 8.4.1发行版。

\subsubsection*{Rapid Industry Solution Enabler}
\begin{DUlineblock}{0em}
\item[] Technical Leader
\item[] 06/2006 -{}- 07/2008
\end{DUlineblock}
Rapid Industry Solution Enabler是一个代码生成工具,其目标是加速IBM ECM和BPM平台上的解决方
案开发。这个工具从UML数据模型和流程模型生成数据库schema,流程定义, Java代码和UI代码。所有这些构成了
一个可以运行的Web应用。这是一个IBM CDL和IBM Waston Research Center的合作项目。项目的技术
领导者是一个IBM杰出软件工程师。

\subsubsection*{ECM (企业内容管理)}
\begin{DUlineblock}{0em}
\item[] Software Engineer/Team Lead
\item[] 11/2004 – 05/2006
\item[] Java/J2EE
\end{DUlineblock}
这是一个IBM中国实验室和IBM Silicon Valley Lab的合作项目,为IBM ECM产品做开发和测试。我最初用Rational
Robot和Rational Functional Tester做产品界面测试,用ANT和JUnit做非界面测试。我后来加入了
Records Manager Enabler(一个ECM组件)的开发团队,设计实现了FIPS功能点。

\subsection*{NEC-CAS (05/2004 -{}- 10/2004)}
\begin{DUlineblock}{0em}
\item[] Software Engineer
\end{DUlineblock}
我参与了几个文档管理系统的应用软件开发。我主要做了bug fixing,功能测试和性能测试的工作。

\subsection*{北京理工大学 (09/2001 – 04/2004)}
研究方向包括Semantic Web和知识推理。我参加了几个基于J2EE的项目。


\subsection*{Yuanfull Co., Ltd (09/2000 – 08/2001 )}
\begin{DUlineblock}{0em}
\item[] Software Engineer/Project Manager
\item[] Visual Baisc/SQL Server
\end{DUlineblock}
我领导了中国海油的建设成本管理系统和建设项目投标系统的开发。

\subsection*{Lexotech Co., Ltd (08/1999 – 08/2000 )}
\begin{DUlineblock}{0em}
\item[] Software Engineer
\item[] Visual Baisc/ASP/SQL Server
\end{DUlineblock}
我参加了湘财证券的投资管理和在线资讯系统的开发工作。另外我参加了一些垂直搜索引擎的简单开发工作。


\section*{文章}
\begin{itemize}
\item Effective Java GUI automation on multiple platforms. IBM developerWorks.
\url{http://www-128.ibm.com/developerworks/rational/library/05/1004_yao/?ca=dgr-lnxw06JavaGUI4Win-Linux}.
\item Integrate FileNet BPM with IBM Content Manager,
Part 3: Implement the Component Integrator-based work performers.
IBM developerworks.
\url{http://www.ibm.com/developerworks/data/library/techarticle/dm-0805yao/index.html}.
\item 一个语义Web应用研究:旅游信息系统{[}J{]}。 北京理工大学学报(Ei核心刊物)。
\item 语义Web系统及其实现{[}J{]}。 北京理工大学学报(Ei核心刊物)。 2004,24(2):145-149。
\item 基于奇异值分解的中文Onotology自动学习技术{[}J{]}。 计算机工程(Ei Page One刊物),2003,29(9):137-139。
\item CRL:对语义Web上的Ontology表示语言DAML-OIL的一种扩充方案{[}J{]}。 计算机工程与应用(核心期刊),2003-23。
\item 一个语义Web架构及其实现{[}J{]}。 计算机工程与应用(核心期),2003-15。
\end{itemize}

\newpage

%___________________________________________________________________________

\section*{Personal Information}
\begin{DUlineblock}{0em}
\item[] Name:   Jingguo Yao
\item[] Gender: Male
\item[] Mobile: (86)139-1033-4380
\item[] Email:  \href{mailto:yaojingguo@gmail.com}{yaojingguo@gmail.com}
\item[] Blog:   \url{http://yaojingguo.javaeye.com}
\url{https://github.com/yaojingguo}
\end{DUlineblock}


\section*{Summary}
\begin{itemize}
\item 11-year experience of software development and project management
experience.
\item Fluent in Java, C/C++, Python, Lisp and Bash. Familiar with Haskell and Ruby.
\item Deep knowledge in computer system construction. Rich experience on building
mission-critical systems.
\item Experienced with Big Data system and data mining technology. Rich experience
on building data analysis systems on Hadoop platform.
\item Familiar with cloud virtualization technology.
\end{itemize}

\section*{English}
CET6 Excellent, fluent spoken English, many years of experience to collaborate
with oversea teams.

\section*{Education, Certification and Training}
\begin{itemize}
\item Beijing Institute of Technology M.S. of Computer Application, April 2004
\item Beijing Institute of Technology B.Sc. of Computer Science, July 1999
\item IBM Certified Solution Designer DB2 Content Manager V8
\item MIT-6.830 Database Systems
\item MIT-6.828 Operating System Engineering
\item MIT-6.824 Distributed Systems
\end{itemize}

\section*{Professional Experience}

\subsection*{www.tianya.cn (2010/08 -{}- present)}

\subsubsection*{Tianya Big Data Department}
\begin{DUlineblock}{0em}
\item[] Director
\item[] 09/2012 -{}- present
\end{DUlineblock}
\begin{itemize}
\item Manage data mining, information filtering, search and membership teams.
\item Draft Tianya Big Data strategy.
\item Follow Big Data research.
\end{itemize}

\subsubsection*{Tianya Data Analysis Platform}
\begin{DUlineblock}{0em}
\item[] Technical manager
\item[] Java, Clojure
\item[] 02/2011 -{}- 08/2012
\end{DUlineblock}
\begin{itemize}
\item Develop log analysis platform with Hadoop/Hive.
\item Develop recommendation algorithm with Hadoop.
\item Develop a data integration system with Clojure to replace an existing
commercial system. Leveraging clojure's DSL power, I was able to develop a
powerful and extensible system with a small code base.
\item Develop and maintain BI system.
\end{itemize}

\subsubsection*{Tianya Virtualization Platform}
\begin{DUlineblock}{0em}
\item[] senior system architect
\item[] 12/2010 -{}- 01/2011
\end{DUlineblock}
Technique investigation and design.

\subsubsection*{Memlink Project}
\begin{DUlineblock}{0em}
\item[] Committer
\item[] C Linux System Programming, libevent
\item[] 09/2010 -{}- 11/2010
\end{DUlineblock}
\url{http://code.google.com/p/memlink/} is a list data storage engine. I designed and
implemented the master-slave synchronization part.

\subsection*{IBM CDL Beijing (2004/11 -{}- 2010/08)}

\subsubsection*{ECM/BPM CoE}
\begin{DUlineblock}{0em}
\item[] COE Leader
\item[] Enterprise Content Management/Business Process Management
\item[] 06/2010 -{}- 08/2010
\end{DUlineblock}
The goal of ECM CoE is to grow China market for IBM ECM/BPM products. Our task
is to help build strategic reference accounts for IBM ECM/BPM products. As the
CoE leader, I was involved in pre-sales and solution design for banking, insurance
, manufacture and nuclear-power industries. Insurance analysis project uses SPSS to
do cross-selling, association rule and decision tree algorithms are used.


\subsubsection*{Big Datg Query Language Jaql}
\begin{DUlineblock}{0em}
\item[] Jaql Committer/Technical Leader
\item[] 07/2009 -{}- 05/2010
\item[] Java/C++/SPSS Statistics
\end{DUlineblock}
Jaql (\url{http://code.google.com/p/jaql/}) is a query language on top of IBM Hadoop Big Data platform.
It is similar to Hadoop Pig. In this project, I worked together with researchers
from IBM Almaden Research Center. I was the technical leader at China side and Jaql
committer. I designed and implemented Jaql catalog server, the integration between
SPSS and Jaql, and the conversion between JSON and CSV. Catalog server provides a
simple key-value storage. The SPSS and Jaql integration is aimed to combine the
statistics and data mining power of SPSS and the scalability and parallelism of
Hadoop platform. With this integration, some SPSS algorithms can be executed in
parallel on Hadoop. After the design and implementation of the conversion between
JSON and CSV, I did performance improvement to make it have the same performance
characteristics as Pig.

\subsubsection*{ECM (Enterprise Content Management) Advanced Engagement}
\begin{DUlineblock}{0em}
\item[] Development Lead
\item[] 08/2008 -{}- 06/2009
\end{DUlineblock}
I led the development of policy conservation system for Taikang Life. The
development team consists of IBM Dev/Service team, IBM business partner team
and Taikang IT personnel. The system uses IBM ECM and FileNet BPM products. I
led the team to finish the project with a high quality on a tight schedule. In
this project, I did the integration between IBM Content Manager and FileNet BPM
which has been put into product in IBM Content Manager 8.4.1. And I travelled
to US to demonstrate the integration in IOD 2008.

\subsubsection*{Rapid Industry Solution Enabler}
\begin{DUlineblock}{0em}
\item[] Technical Leader
\item[] 06/2006 -{}- 07/2008
\end{DUlineblock}
Rapid Industry Solution Enabler is a code generation tool to accelerate the
solution development based on IBM ECM and BPM products. Fed with a UML data
model and a process model, the tool can generated database schema, process
definition, Java code and UI code which are parts of a runnable web application.
In this project, I worked together with an IBM distinguished engineer and IBM Waston
Research Center.

\subsubsection*{ECM}
\begin{DUlineblock}{0em}
\item[] Software Engineer/Team Lead
\item[] 11/2004 – 05/2006
\end{DUlineblock}
I worked together with IBM Silicon Valley Lab. At first, I did various test
automation work for ECM products.  I use Rational Robot and Ration Functional
Tester to test GUI.  And I use ANT and JUnit for non-GUI test. Later on, I did
the development work for Records Manager Enabler (One ECM component). I
designed and implemented FIPS feature. The related techniques are Java and
J2EE.

\subsection*{NEC-CAS (05/2004 -{}- 10/2004)}
\begin{DUlineblock}{0em}
\item[] Software Engineer
\end{DUlineblock}
I took part in the development and test of several document management
software.  I did bug fixing, functional testing and performance testing.

\subsection*{Beijing Institute of Technology (09/2001 – 04/2004)}
Research interests are Semantic Web and knowledge reasoning.
I took part in some J2EE-based projects.

\subsection*{Yuanfull Co., Ltd (09/2000 – 08/2001)}
\begin{DUlineblock}{0em}
\item[] Software Engineer/Project Manager
\item[] Visual Baisc/SQL Server
\end{DUlineblock}
I led the development of construction cost management system and construction
project bidding system for CNOOC. These two systems employed C/S architecture.

\subsection*{Lexotech Co., Ltd (08/1999 – 08/2000)}
\begin{DUlineblock}{0em}
\item[] Software Engineer
\item[] Visual Baisc/ASP/SQL Server
\end{DUlineblock}
I developed investment management system  and on-line consultation system for
Xiangcai Securities Co. I also took part in the development of a vertical
search engine.

\section*{Publications}
\begin{itemize}

\item Effective Java GUI automation on multiple platforms. IBM developerWorks.
\url{http://www-128.ibm.com/developerworks/rational/library/05/1004_yao/?ca=dgr-lnxw06JavaGUI4Win-Linux}.
\item Integrate FileNet BPM with IBM Content Manager, Part 3: Implement the
Component Integrator-based work performers. IBM
developerworks. \url{http://www.ibm.com/developerworks/data/library/techarticle/dm-0805yao/index.html}.
\item A Case Study of Semantic Web Application: Tourism Information System {[}J{]}.
Transactions of Beijing Institute of Technology (Ei core journal).
\item A Semantic Web System Architecture and its Implementation {[}J{]}. Transactions
of Beijing Institute of Technology (Ei core journal). 2004, 24(2):145-149.
\item Automated Chinese Ontology Learning Technology Based on Singular Value
Decomposition {[}J{]}. Computer Engineering (Ei Page One journal), 2003,
29(9):137-139.
\item CRL: an Extension of DAML+OIL-Ontology Language for the Semantic Web {[}J{]}.
Computer Engineering and Application (core journal), 2003-23.
\item A Semantic Web Architecture and its Implementation {[}J{]}. Computer Engineering
and Application (core journal), 2003-15.
\end{itemize}
\end{CJK}
\end{document}
